\chapter{Dielectric and Magnetic Media}\label{dielectric}
\section{Introduction}
In this chapter, we shall use Maxwell's equations to investigate the interaction
of dielectric and magnetic media with quasi-static electric and magnetic fields.

\section{Polarization}\label{spolz}
The terrestrial environment is characterized by 
dielectric media ({\em e.g.},
air, water) which are, for the most part, electrically neutral, since  they
are made up  of neutral atoms and molecules. However, if these
atoms and
molecules are placed in an external electric field then they tend to {\em polarize}: {\em i.e.}, their positively and negatively charged
components displace with respect to one another.
Suppose that if a given neutral molecule is placed in an external  electric
field ${\bf E}$ then the centre of charge of its 
constituent electrons, whose total
charge is (say) $q$, displaces by ${\bf d}$ with respect
to the center of charge of its constituent
atomic nuclii. The {\em dipole moment}\/
of the molecule is defined as ${\bf p} = q \,{\bf d}$. If a dielectric medium is made up of 
$N$ such molecules per unit volume then the {\em electric polarization},
${\bf P}$, of the medium ({\em i.e.}, the dipole moment per unit volume) is
given by ${\bf P} = N \,{\bf p}$. More generally,
\begin{equation}
{\bf P}({\bf r}) = \sum_i N_i \,\langle {\bf p}_i\rangle,
\end{equation}
where $\langle {\bf p}_i\rangle$ is the
average  dipole moment of the $i$th type of molecule making up the  medium,
and $N_i$ the average number of such molecules per unit volume, in the vicinity of point ${\bf r}$.

Now, we saw previously,  in Exercise~2.4, that the scalar electric potential field generated by an
electric dipole of  moment ${\bf p}$ situated at the origin is
\begin{equation}
\phi({\bf r}) = \frac{{\bf p}\cdot{\bf r}}{4\pi\epsilon_0\,r^3}.
\end{equation}
Hence, from the principle of superposition, the scalar potential field
generated by a dielectric medium of dipole moment per unit volume ${\bf  P}({\bf r})$ is
\begin{equation}
\phi({\bf r}) = \frac{1}{4\pi\epsilon_0}\int\frac{{\bf P}({\bf r}')\cdot({\bf r}-{\bf r}')}{|{\bf r}-{\bf r}'|^3}\,d^3{\bf r}',
\end{equation}
where the volume integral is over all space. However, it follows from Equations~(\ref{e315}) and (\ref{e2.145h}) that
\begin{equation}
\phi({\bf r}) = \frac{1}{4\pi\epsilon_0}\int{\bf P}({\bf r}')\cdot\nabla' \left(\frac{1}{|{\bf r}-{\bf r}'|}\right)\,d^3{\bf r}'.
\end{equation}
Finally, making use of Equation~(\ref{e2.152x}), we obtain
\begin{equation}
\phi({\bf r}) = \frac{1}{4\pi\epsilon_0}\int \frac{\rho_b({\bf r}')}{|{\bf r}-{\bf r}'|}\,d^3{\bf r}',
\end{equation}
assuming that $|{\bf P}|/r\rightarrow 0$ as $r \rightarrow\infty$,
where $\rho_b =- \nabla\!\cdot\!{\bf P}$.
Thus, by comparison with Equation~(\ref{e317}), we can see that minus the divergence
of the polarization field  is equivalent to a charge density.

As explained above, any divergence of the polarization
field ${\bf P}({\bf r})$ of a dielectric medium gives rise to an effective charge density
$\rho_b({\bf r})$, where
\begin{equation}\label{e6.2}
\rho_b = -\nabla\!\cdot\!{\bf P}.
\end{equation}
This charge density is attributable to {\em bound charges} ({\em i.e.},
charges which arise from the polarization of neutral atoms), and
is usually distinguished from the charge density  $\rho_f({\bf r})$ due to
{\em free charges}, which typically represents a net surplus or
deficit of electrons in the medium. Thus, the total
charge density $\rho$ in the medium is
\begin{equation}
\rho = \rho_f + \rho_b.
\end{equation}
It must be emphasized that both terms in this equation represent real
physical charge. Nevertheless, it is useful to make the distinction
between bound and free charges, especially when it comes to working out
the energy associated with electric fields in dielectric media. 

Gauss' law takes the differential form
\begin{equation}
\nabla\!\cdot\!{\bf E} = \frac{\rho}{\epsilon_0} =
\frac{\rho_f + \rho_b}{\epsilon_0}.
\end{equation}
This expression can be rearranged to give
\begin{equation}\label{e6.5}
\nabla\!\cdot\!{\bf D} = \rho_f,
\end{equation}
where
\begin{equation}\label{e6.6}
{\bf D} = \epsilon_0 \,{\bf E} + {\bf P}
\end{equation}
is termed the {\em electric displacement}, and has the same dimensions
as ${\bf P}$ ({\em i.e.}, dipole moment
per unit volume). Gauss' theorem tells us that
\begin{equation}\label{e6.7}
\oint_S {\bf D}\!\cdot\!d{\bf S} = \int_V \rho_f\,dV.
\end{equation}
In other words, the flux of ${\bf D}$ out of some closed surface $S$
is equal to the total {\em free charge}\/ enclosed within that surface. Unlike the
electric field ${\bf E}$ (which is the electric force acting on a unit charge),
or the polarization ${\bf P}$ (which is the dipole moment per unit volume),
the electric displacement ${\bf D}$ has no clear physical interpretation. 
In fact, the only reason for introducing this quantity is that it enables us to
calculate electric fields in the presence of dielectric media  without 
having to know the distribution of bound charges beforehand. However,
this is only possible if we have a {\em constitutive
relation}\/ connecting  ${\bf E}$ and  ${\bf D}$.

\section{Electric Susceptibility and Permittivity}
In a large class of dielectric materials, there exits an approximately
 linear relationship between ${\bf P}$ and ${\bf E}$. If the material
 is isotropic then
\begin{equation}\label{e6.8}
{\bf P} = \epsilon_0\,\chi_e  \,{\bf E},
\end{equation}
where $\chi_e$ is termed the {\em electric susceptibility}. It follows that
\begin{equation}\label{e6.9}
{\bf D} = \epsilon_0\,\epsilon\,{\bf E},
\end{equation}
where
\begin{equation}\label{e6.10}
\epsilon= 1 + \chi_e
\end{equation}
is termed the {\em relative dielectric constant}\/ or
 {\em relative permittivity}\/ of the medium. 
(Likewise, $\epsilon_0$ is termed the {\em permittivity of free space}.)
Note that $\epsilon$ is dimensionless. Values of $\epsilon$ for some
common dielectric materials are given in Table~\ref{tdie}. It can be seen
that dielectric constants are generally greater than unity, and can be significantly greater than unity for liquids and solids.

\begin{table}\centering
\begin{tabular}{ll}\hline
{\bf Material} & {\bf Dielectric constant}\\[0.5ex]\hline
Air (1 atm)& $1.00059$\\[0.5ex]
Paper & $3.5$\\[0.5ex]
Concrete&$4.5$\\[0.5ex]
Glass &5--10\\[0.5ex]
Silicon & $11.68$\\[0.5ex]
Water &80.4
\end{tabular}
\caption{\em Low-frequency dielectric constants of some common materials.}\label{tdie}
\end{table}


It  follows from Equations~(\ref{e6.5}) and (\ref{e6.9})
that
\begin{equation}\label{e6.11}
\nabla\!\cdot\!{\bf E} = \frac{\rho_f}{\epsilon_0\,\epsilon}.
\end{equation}
Thus, the electric fields produced by free charges in a {\em uniform}\/ dielectric medium are analogous
to those produced by the same charges in a vacuum,  except that they are all reduced by
a factor $\epsilon$. This reduction can be understood in terms of a 
polarization of the atoms or molecules in  the dielectric medium which produces
electric fields which oppose those generated by the free charges. One immediate 
consequence of this effect is that the capacitance of a capacitor is increased by a factor
$\epsilon$ if the empty space between its electrodes is filled with a
dielectric medium of dielectric constant $\epsilon$ (assuming that
fringing fields can be neglected). 

It must be understood that Equations~(\ref{e6.8})--(\ref{e6.11}) merely represent an {\em approximation}\/
which is generally found to hold under terrestrial conditions  in {\em isotropic}\/  media (provided that
the electric field intensity is not too large). For anisotropic
media ({\em e.g.}, crystals), Equation~(\ref{e6.9}) generalizes to
\begin{equation}
{\bf D} = \epsilon_0 \,\mbox{\boldmath$ \epsilon$}\!\cdot\!{\bf E},
\end{equation}
where \mbox{\boldmath$\epsilon$} is a second-rank tensor known as the
{\em dielectric tensor}. For strong electric fields,
${\bf D}$ ceases to vary linearly with ${\bf E}$.


\section{Boundary Conditions for ${\bf E}$ and ${\bf D}$}
When the space surrounding a set of charges contains dielectric material
of {\em non-uniform} dielectric constant then the electric field no longer
has the same functional form as in  a vacuum. Suppose, for example, that 
space is occupied by  two dielectric media, labeled 1 and 2, whose uniform dielectric
constants are $\epsilon_1$ and $\epsilon_2$, respectively. What are the boundary
conditions on ${\bf E}$ and ${\bf D}$ at the interface between the
two media?


Let us apply Equation~\ref{e6.7} to  a Gaussian pill-box $S$ enclosing part of the interface---see Figure~\ref{f43a}. The thickness of the pill-box
is allowed to tend towards zero, so that the only contribution to
the outward flux of ${\bf D}$ comes from the flat faces of the box, which are
parallel to the interface.  Assuming that there
is no free charge inside the pill-box  (which is reasonable in the limit
in which the volume of the box tends to zero), then Equation~(\ref{e6.7})
yields
\begin{equation}\label{e6.13}
D_{\perp\,1}-D_{\perp\,2} = 0,
\end{equation}
where $D_{\perp\,1}$ is the component of the electric displacement in medium 1 which is normal to the interface, {\em etc}. 
According to Equation~(\ref{e6.9}), the boundary condition on the
normal component of the electric field is
\begin{equation}
\epsilon_1\,E_{\perp\,1}- \epsilon_2\,E_{\perp\,2} = 0.
\end{equation}
Integrating Faraday's law,
\begin{equation}
\nabla\times {\bf E} = -\frac{\partial {\bf B}}{\partial t},
\end{equation}
around a narrow rectangular loop $C$ which straddles the interface---see Figure~\ref{f43a}---yields
\begin{equation}
(E_{\parallel\,1} - E_{\parallel\,2})\,l = -A\,\frac{\partial B_\perp}{\partial t},
\end{equation}
where $l$ is the length of the long side of the loop, $A$  the area of the loop, and $B_\perp$ the magnetic field normal to the loop. In the limit in which the length of the short side of the loop tends to zero, $A$ also goes to zero, and we obtain
the familiar boundary condition
\begin{equation}
E_{\parallel\,1}-E_{\parallel\,2} = 0.
\end{equation}
Generally speaking, there is a {\em bound charge sheet}\/ on the interface between two dielectric media. The charge density of this sheet follows from Gauss' law:
\begin{equation}
\sigma_b = \epsilon_0\,(E_{\perp\,1}-E_{\perp\,2}) = (1/\epsilon_1-1/\epsilon_2)\,D_{\perp}.
\end{equation}
This can also be written
\begin{equation}\label{e5.21a}
\sigma_b = ({\bf P}_1-{\bf P}_2)\cdot {\bf n},
\end{equation}
where ${\bf n}$ is the unit normal at the interface (pointing from medium 1 to medium 2), and ${\bf P}_{1,2}$ are the electric polarizations in the two media.
\begin{figure}
\epsfysize=1.75in
\centerline{\epsffile{chapter6/fig6.1.eps}}
\caption{\em The boundary between two different dielectric media.}\label{f43a}
\end{figure}

\section{Boundary Value Problems with Dielectrics}
Consider a point charge $q$ embedded in a semi-infinite dielectric medium of uniform dielectric constant
$\epsilon_1$, and located a distance $d$ away from a plane interface which
separates this medium from another semi-infinite dielectric medium of
dielectric constant
$\epsilon_2$. Let the interface  coincide with the plane $z=0$, and
let the point charge lie on the positive $z$-axis. In order to solve this  problem, 
we need to find solutions to the equations
\begin{equation}\label{e6.16}
\epsilon_1\, \nabla\!\cdot\!{\bf E} = \frac{q\,\delta({\bf r}-{\bf r}_0)}{\epsilon_0},
\end{equation}
where ${\bf r}_0=(0,\,0,\,d)$, for $z>0$,
\begin{equation}\label{e6.17}
\epsilon_2\,\nabla\!\cdot\!{\bf E} = 0
\end{equation}
for $z<0$, and
\begin{equation}
\nabla\times{\bf E} = {\bf 0}
\end{equation}
everywhere, subject to the boundary conditions 
\begin{eqnarray}\label{e6.19a}
\epsilon_1\, E_z(x,y,0_+) &=& \epsilon_2 \,E_z (x,y,0_-),\\[0.5ex]\label{e6.19b}
E_x (x,y,0_+) &=& E_x (x,y,0_-),\\[0.5ex]
E_y (z,y,0_+) &=& E_y(x,y,0_-).\label{e6.19c}
\end{eqnarray}

We can  solve this problem by employing a slightly modified form of
the  method of images---see Sect~\ref{s5.10}. Since $\nabla\times{\bf E} ={\bf  0}$ everywhere,
the electric field can be written in terms of a scalar potential: {\em i.e.}, ${\bf E} =-\nabla \phi$. Consider the region $z>0$. 
We shall assume that the scalar potential in this region is the same as
that in an analog problem in which the whole of space is filled with a dielectric medium of dielectric constant
$\epsilon_1$, and, in addition to the real charge $q$ at position $(0,\,0,\,d)$,
there is a second charge $q'$ at the image position $(0,\,0,\,-d)$---see Figure~\ref{fimagex}. 
If this is the case, then the potential at some general point 
in the region $z>0$ is given by
\begin{equation}\label{e6.20}
\phi(r,z) = \frac{1}{4\pi\epsilon_0\, \epsilon_1}\left(\frac{q}{R_1}
+ \frac{q'}{R_2}\right),
\end{equation}
where $R_1= \sqrt{r^2+(d-z)^2}$, $R_2= \sqrt{r^2+(d+z)^2}$,  and
$r=\sqrt{x^2+y^2}$. 
Note that the potential (\ref{e6.20})  is clearly a solution of Equation~(\ref{e6.16}) in
the region $z>0$: {\em i.e.}, it gives $\nabla \cdot {\bf E}  = 0$, with the
appropriate singularity at the position of the point charge $q$. 
\begin{figure}
 \epsfysize=2.25in
\centerline{\epsffile{chapter6/fig6.2.eps}}
\caption{\em The method of images for a charge near the plane interface
between two dielectric media.}\label{fimagex}
\end{figure}


Consider the region $z<0$. Let us assume that the scalar potential in this
region is the same as that in an analog problem in which the whole of space is filled
with a dielectric medium of dielectric constant $\epsilon_2$, and a charge $q''$ is located at  $(0,\,0,\,d)$---see Figure~\ref{fimagex}. If this is the case, then the potential in the region $z<0$ is
given by
\begin{equation}
\phi(r,z) = \frac{1}{4\pi\epsilon_0\,\epsilon_2} \frac{q''}{R_1}.
\end{equation}
The above potential is clearly a solution of Equation~(\ref{e6.17}) in the region
$z<0$: {\em i.e.}, it gives $\nabla\!\cdot\!{\bf E}  = 0$, with 
no singularities. 

It now remains to choose $q'$ and $q''$ in such a manner that the boundary
conditions (\ref{e6.19a})--(\ref{e6.19c}) are satisfied. The boundary conditions (\ref{e6.19b}) and
(\ref{e6.19c}) are obviously satisfied if the scalar potential is continuous
at the interface between the two dielectric media: {\em i.e.},  if
\begin{equation}
\phi(r,0_+) = \phi(r,0_-).
\end{equation}
The boundary condition (\ref{e6.19a}) implies a jump in the normal derivative
of the scalar potential across the interface: {\em i.e.}, 
\begin{equation}
\epsilon_1 \,\frac{\partial\phi(r,0_+)}{\partial z} = \epsilon_2\,
\frac{\partial \phi(r,0_-)}{\partial z}.
\end{equation}
The first matching condition yields
\begin{equation}\label{e6.24}
\frac{q+q'}{\epsilon_1} = \frac{q''}{\epsilon_2},
\end{equation}
whereas the second gives
\begin{equation}\label{e6.25}
q-q' = q''.
\end{equation}
Here, use has been made of
\begin{equation}
\frac{\partial}{\partial z}\!\left(\frac{1}{R_1}\right)_{z=0}
=- \frac{\partial}{\partial z}\!\left(\frac{1}{R_2}\right)_{z=0}
= \frac{d}{(r^2+d^2)^{3/2}}.
\end{equation}
Equations~(\ref{e6.24}) and (\ref{e6.25}) imply that
\begin{eqnarray}
q' &=& -\left(\frac{\epsilon_2-\epsilon_1}{\epsilon_2 + \epsilon_1}
\right) q,\\[0.5ex]
q''&=& \left(\frac{2\,\epsilon_2}{\epsilon_2+\epsilon_1}\right) q.
\end{eqnarray}

Now, the bound charge density is given by $\rho_b = - \nabla\!\cdot\!
{\bf P}$, However,  we have ${\bf P}=\epsilon_0\,
\chi_e \,{\bf E}$ inside both dielectric media. Hence, $\nabla \cdot {\bf P} = \epsilon_0\,
\chi_e\, \nabla {\cdot} {\bf E} = 0$, except at the location of the original point charge. 
We conclude that there is zero  bound charge density in either dielectric
medium.  However,  
there is a bound charge sheet on the interface
between the two  media. 
In fact, the density of this sheet is given by
\begin{equation}
\sigma_b(r) = \epsilon_0\,(E_{z\,1}-E_{z\,2})_{z=0}.
\end{equation}
Hence,
\begin{equation}
\sigma_b(r) = \epsilon_0\,\,\frac{\partial\phi(r,0_-)}{\partial z}-
\epsilon_0\,\,\frac{\partial\phi(r,0_+)}{\partial z}
=-\frac{q}{2\pi} \frac{(\epsilon_2-\epsilon_1)}
{\epsilon_1(\epsilon_2+\epsilon_1)} \frac{d}{(r^2+d^2)^{3/2}}.
\end{equation}
Incidentally, it is easily demonstrated that the net charge on the interface
is $q'/\epsilon_1$. 
In the limit $\epsilon_2\gg \epsilon_1$, the dielectric with dielectric constant  $\epsilon_2$
behaves like a conductor ({\em i.e.}, ${\bf E}\rightarrow
{\bf 0}$ in the region $z<0$), 
and the bound surface charge  density 
on the interface approaches that obtained in the case where the plane
$z=0$ coincides with a conducting surface---see Section~\ref{s5.10}.

As a second example, consider a dielectric sphere of radius $a$, and
uniform dielectric constant $\epsilon$, placed in a uniform
$z$-directed electric field of magnitude $E_0$. Suppose that the
sphere is centered on the origin. Now, we
can always write ${\bf E}=-\nabla\phi$ for an electrostatic problem. In the present case, $\nabla\cdot{\bf E}=0$ both inside and outside the sphere, since there are no free charges, and the bound  charge density is zero in a uniform 
dielectric medium (or a vacuum). Hence, the scalar potential satisfies Laplace's equation, $\nabla^2\phi=0$, throughout space. Adopting spherical polar coordinates,
$(r, \theta, \varphi)$, aligned along the $z$-axis, the boundary conditions are that $\phi\rightarrow
- E_0\,r\,\cos\theta$ as $r\rightarrow\infty$, and that $\phi$ is well-behaved at
$r=0$. At the surface of the sphere, $r=a$, the continuity of $E_\parallel$
implies that $\phi$ is  continuous. Furthermore, the
continuity of $D_\perp=\epsilon_0\,\epsilon\,E_\perp$ leads to the matching condition
\begin{equation}\label{emat}
\left.\frac{\partial \phi}{\partial r}\right|_{r=a+} = \left.\epsilon\,
\frac{\partial\phi}{\partial r}\right|_{r=a-}.
\end{equation}

Let us try axisymmetric separable solutions of the form $r^m\,\cos\theta$. It is
easily demonstrated that such solutions satisfy Laplace's equation
provided that $m=1$ or $m=-2$. Hence, the most general solution to Laplace's equation outside
the sphere, which satisfies the boundary condition at $r\rightarrow\infty$, is
\begin{equation}
\phi(r,\theta) = - E_0\,r\,\cos\theta + E_0\,\alpha \,\frac{a^3\,\cos\theta}{r^2}.
\end{equation}
Likewise, the most general solution inside the sphere, which satisfies
the boundary condition at $r=0$, is
\begin{equation}
\phi(r,\theta) = - E_1\,r\,\cos\theta.
\end{equation}
The continuity of $\phi$ at $r=a$ yields
\begin{equation}
E_0 - E_0\,\alpha = E_1.
\end{equation}
Likewise, the matching condition (\ref{emat}) gives
\begin{equation}
E_0 + 2\,E_0\,\alpha = \epsilon\,E_1.
\end{equation}
Hence, we obtain
\begin{eqnarray}
\alpha &=& \frac{\epsilon-1}{\epsilon+2},\\[0.5ex]
E_1 &=& \frac{3\,E_0}{\epsilon+2}.\label{e5.48x}
\end{eqnarray}
Note that the electric field inside the sphere is {\em uniform}, parallel
to the external electric field outside the sphere, and of magnitude $E_1$. Moreover, $E_1 < E_0$ , provided that $\epsilon>1$---see Figure~\ref{fex6}.
The density of the bound charge sheet on the surface of the sphere
is
\begin{equation}
\sigma_b(\theta) = -\epsilon_0\left(\left.\frac{\partial\phi}{\partial r}\right|_{r=a+}
-\left.\frac{\partial\phi}{\partial r}\right|_{r=a-}\right) = 
3\,\epsilon_0\left(\frac{\epsilon-1}{\epsilon+2}\right)E_0\,\cos\theta.
\end{equation}
Finally, the electric field outside the sphere consists of the original
uniform field, plus the field of an electric dipole of moment
\begin{equation}
{\bf p} = 4\pi\,a^3\,\epsilon_0\left(\frac{\epsilon-1}{\epsilon+2}\right)E_0\,{\bf e}_z.
\end{equation}
This is simply the net induced  dipole moment, ${\bf p} = (4/3)\,\pi\,a^3\,{\bf P}$, of
the sphere, where ${\bf P} = \epsilon_0\,(\epsilon-1)\,E_1\,{\bf e}_z$. 
\begin{figure}
 \epsfysize=3.in
\centerline{\epsffile{chapter6/fig6.3.eps}}
\caption{\em Equally spaced coutours of the axisymmetric potential
$\phi(r,\theta)$ generated by a dielectric sphere of unit radius and dielectric constant $\epsilon=3$ placed in a uniform $z$-directed electric field.}\label{fex6}
\end{figure}

As a final example, consider a spherical cavity, of radius $a$, inside a
uniform dielectric medium of dielectric constant $\epsilon$, in the presence of a
$z$-directed electric field of magnitude $E_0$. This problem is analogous
to the previous one, except that the matching condition
(\ref{emat}) becomes
\begin{equation}
\left.\epsilon\,\frac{\partial \phi}{\partial r}\right|_{r=a+} = \left.
\frac{\partial\phi}{\partial r}\right|_{r=a-}.
\end{equation}
Hence, $\epsilon\rightarrow1/\epsilon$, and we obtain
\begin{eqnarray}
\alpha &=&- \frac{\epsilon-1}{1+2\,\epsilon},\\[0.5ex]
E_1 &=& \frac{3\,\epsilon\,E_0}{1+2\,\epsilon}.
\end{eqnarray}
Note that the field inside the cavity is  {\em uniform}, parallel
to the external electric field outside the sphere, and of magnitude $E_1$. Moreover, $E_1 > E_0$,  provided that $\epsilon>1$---see Figure~\ref{fex7}. The density
of the bound charge sheet on the inside surface of the cavity
is
\begin{equation}
\sigma_b(\theta) = -\epsilon_0\left(\left.\frac{\partial\phi}{\partial r}\right|_{r=a+}
-\left.\frac{\partial\phi}{\partial r}\right|_{r=a-}\right) = 
-3\,\epsilon_0\left(\frac{\epsilon-1}{1+2\,\epsilon}\right)E_0\,\cos\theta.
\end{equation}
Hence, it follows from Equation~(\ref{e5.21a}) that the polarization immediately
outside the cavity is
\begin{equation}
{\bf P} = 3\,\epsilon_0\left(\frac{\epsilon-1}{1+2\,\epsilon}\right)E_0\,{\bf e}_z.
\end{equation}
This is less than the polarization field a long way from the cavity by a factor
$3/(1+2\,\epsilon)$. In other words, the cavity induces a slight depolarization
of the dielectric medium in its immediate vicinity. The electric field
inside the cavity can be written
\begin{equation}\label{e5.yyy}
{\bf E}_1 = {\bf E}_0 + \frac{\bf P}{3\epsilon_0},
\end{equation}
where ${\bf E}_0$ is the external field, and ${\bf P}$ the polarization field
immediately outside the cavity.
\begin{figure}
 \epsfysize=3.in
\centerline{\epsffile{chapter6/fig6.4.eps}}
\caption{\em Equally spaced coutours of the axisymmetric potential
$\phi(r,\theta)$ generated by a cavity of unit radius inside a dielectric medium of dielectric constant $\epsilon=3$ placed in a uniform $z$-directed electric field.}\label{fex7}
\end{figure}


\section{Energy Density Within a Dielectric Medium}
Consider a system of free charges embedded in a dielectric
medium. The increase  in the total energy when a small
amount  of free  charge $\delta\rho_f$ is added to the system
is given by
\begin{equation}
\delta W = \int \phi \,\delta\rho_f \,d^3{\bf r},
\end{equation}
where the integral is taken over all space,
 and $\phi({\bf r})$ is the
electrostatic potential. 
Here, it is assumed that both the original charges and the dielectric medium are
held fixed, so that no mechanical work is performed.
It follows from Equation~(\ref{e6.5}) that
\begin{equation}\label{e6.35}
\delta W = \int \phi\,\nabla\!\cdot\!\delta {\bf D}\,d^3{\bf r},
\end{equation}
where $\delta{\bf D}$ is the change in the electric displacement associated
with the charge increment. Now the above equation can also be written
\begin{equation}
\delta W = \int \nabla\!\cdot\!(\phi\,\delta{\bf D})\,d^3{\bf r}
- \int \nabla\phi\!\cdot\!\delta{\bf D}\,d^3{\bf r},
\end{equation}
giving
\begin{equation}
\delta W = \oint_S \phi \,\delta{\bf D}\!\cdot\!d{\bf S} -
\int_V \nabla\phi\!\cdot\!\delta{\bf D}\,d^3{\bf r},
\end{equation}
where use has been made of Gauss' theorem. Here, $V$ is some
volume bounded by the closed surface $S$. 
If the dielectric medium
is of finite
spatial extent then the surface term is eliminated by integrating over all space. We thus obtain
\begin{equation}\label{e6.37}
\delta W = -\int\nabla\phi  \!\cdot\!\delta{\bf D}\,d^3{\bf r}
= \int {\bf E}\!\cdot\! \delta {\bf D}\,d^3{\bf r}.
\end{equation}
Now, this energy increment cannot be integrated unless ${\bf E}$ is a known
function of ${\bf D}$. Let us adopt the conventional approach, and assume that
${\bf D} =\epsilon_0\, \epsilon\, {\bf E}$, where the dielectric constant
$\epsilon$ is {\em independent}\/ of the electric field. The
change in energy associated with taking the displacement field from
zero to ${\bf D}({\bf r})$ at all points in space is given by
\begin{equation}
W = \int_{\bf 0}^{\bf D} \delta W = \int_{\bf 0}^{\bf D }\int {\bf E}\!\cdot\!
\delta{\bf D}\,d^3{\bf r},
\end{equation}
or
\begin{equation}
W = \int \int_0^E \frac{\epsilon_0\,\epsilon \,\delta (E^2)}{2}\,d^3{\bf r}
= \frac{1}{2} \int \epsilon_0\,\epsilon \,E^2\,d^{3} {\bf r},
\end{equation}
which reduces to 
\begin{equation}\label{e6.41}
W = \frac{1}{2} \int {\bf E}\!\cdot\!{\bf D}\,d^3{\bf r}.
\end{equation}
Thus, the electrostatic energy density inside a dielectric medium is given by
\begin{equation}\label{eed}
U = \frac{1}{2}\, {\bf E}\!\cdot\!{\bf D}.
\end{equation}
This is a standard result, and is often quoted in textbooks. Nevertheless,
it is important to realize that the above formula is only valid for dielectric
media
in which
the electric displacement ${\bf D}$ varies {\em linearly}\/ with the
electric field ${\bf E}$. Note, finally, that Equation~(\ref{eed}) is consistent with
the previously obtained expression (\ref{eeu}).

\section{Force Density Within a Dielectric Medium}
Equation~(\ref{e6.41}) was derived by considering a virtual process in which
true charges are added to a system of charges and dielectrics which are
held fixed, so that no mechanical work is done against physical
displacements. Let us now consider a different virtual process in which
the physical coordinates of the charges and dielectric are given a virtual
displacement $\delta{\bf r}$ at each point in space, but no free charges
are added to the system. Since we are dealing with a conservative system,
the energy expression (\ref{e6.41}) can still be employed, despite the fact that
it was derived in terms of another virtual process. The variation in
the total electrostatic energy $\delta W$ when the system undergoes a virtual
displacement $\delta{\bf r}$ is related to the electrostatic 
force density ${\bf f}$
acting within the dielectric medium via
\begin{equation}
\delta W = - \int {\bf f}\!\cdot\! \delta {\bf r}\,d^3{\bf r}.
\end{equation}
So, if the medium  is moving, and has   a  velocity field ${\bf u}$, then 
the rate at which electrostatic energy is drained from the
${\bf E}$ and ${\bf D}$ fields is given
by
\begin{equation}\label{e5.64x}
\frac{dW}{dt} = - \int {\bf f}\!\cdot\! {\bf u}\,d^3{\bf r}.
\end{equation}

Let us now consider the electrostatic energy increment due to  a change $\delta\rho_f$
in the free charge distribution, and a change $\delta\epsilon$ in the
dielectric constant,  both of which are caused by the virtual displacement. From Equation~(\ref{e6.41}),
\begin{equation}
\delta W = \frac{1}{2\epsilon_0}\int \left[D^2\,\delta(1/\epsilon)
+ 2 \,{\bf D}\!\cdot\!\delta {\bf D}/\epsilon\right]d^3{\bf r},
\end{equation}
or 
\begin{equation}\label{e5.66x}
\delta W  = -\frac{\epsilon_0}{2}\int E^2\,\delta\epsilon \,
d^3{\bf r} + \int {\bf E} \!\cdot\!\delta{\bf D}\,d^3{\bf r}.
\end{equation}
Here, the first term on the right-hand side represents the energy increment due to the change
in dielectric constant associated with the virtual displacement, whereas
the second term corresponds to the energy increment caused by the displacement
of  free charges. The second term can be written
\begin{equation}
\int{\bf E}\!\cdot\! \delta{\bf D}\,d^3{\bf r} = -
\int\nabla\phi \!\cdot\! \delta {\bf D}\, d^3{\bf r} =\int \phi\,
\nabla\!\cdot\!\delta {\bf D}\,d^3{\bf r} = \int \phi\,\delta\rho_f\,d^3{\bf r},
\end{equation}
where surface terms have been neglected. Thus, Equation~(\ref{e5.66x}) implies
that
\begin{equation}\label{e5.68x}
\frac{dW}{dt} = \int\left(\phi\,\frac{\partial\rho_f}{\partial t} - \frac{\epsilon_0}{2} \,E^2\, \frac{\partial\epsilon}{\partial t}\right)d^3{\bf r}.
\end{equation}

In order to arrive at an expression for the force density ${\bf f}$, we
need to express the time derivatives $\partial\rho/\partial t$ and
$\partial\epsilon/\partial t$ in terms of the velocity field ${\bf u}$. 
This can be achieved by adopting a dielectric equation of state:
{\em i.e.}, a relation which gives the dependence of the dielectric
constant $\epsilon$  on the mass density $\rho_m$. Let us assume that 
$\epsilon(\rho_m)$ is a known function. It follows that
\begin{equation}
\frac{D\epsilon}{Dt} = \frac{d\epsilon}{d\rho_m}
\frac{D\rho_m}{Dt},
\end{equation}
where 
\begin{equation}
\frac{D}{Dt} \equiv \frac{\partial}{\partial t} + {\bf u}\!\cdot\!\nabla
\end{equation}
is the total time derivative ({\em i.e.}, the time
derivative in a frame of reference which
is locally co-moving with the dielectric.) 
The hydrodynamic equation of continuity of the dielectric is [see Equation~(\ref{econt})]
\begin{equation}
\frac{\partial \rho_m}{\partial t} + \nabla\!\cdot\!(\rho_m\, {\bf u}) = 0,
\end{equation}
which implies that
\begin{equation}
\frac{D\rho_m}{Dt} = - \rho_m \nabla\!\cdot\!{\bf u}.
\end{equation}
Hence, it follows that
\begin{equation}
\frac{\partial \epsilon}{\partial t} = - \frac{d\epsilon}{d\rho_m}\,
\rho_m \nabla\!\cdot\!{\bf u} - {\bf u} \!\cdot\!\nabla\epsilon.
\end{equation}
The conservation equation for the free charges is written
\begin{equation}
\frac{\partial \rho_f}{\partial t} + \nabla\!\cdot\!(\rho_f\,
{\bf u}) = 0.
\end{equation}
Thus, we can express Equation~(\ref{e5.68x}) in the form
\begin{equation}\label{e5.75x}
\frac{dW}{dt} = \int\left[-\phi\,\nabla\!\cdot\!(\rho_f\,{\bf u}) + \frac{\epsilon_0}{2}\,E^2\,\frac{d\epsilon}{d\rho_m} \,\rho_m\,\nabla\!\cdot\!{\bf u} 
+ \left(\frac{\epsilon_0}{2} \,E^2\,\nabla\epsilon\right)
\!\cdot\! {\bf u}\right]d^3{\bf r}.
\end{equation}
Integrating the first term on the right-hand side by parts, and neglecting any surface contributions,
we obtain
\begin{equation}
-\int\phi\,\nabla\!\cdot\!(\rho_f\,{\bf u}) \,d^3{\bf r} = 
\int \rho_f \,\nabla\phi\!\cdot\! {\bf u}\,d^3{\bf r}.
\end{equation}
Likewise,
\begin{equation}
\int\frac{\epsilon_0}{2}\,E^2\,\frac{d\epsilon}{d\rho_m} \,\rho_m\,\nabla\!\cdot\!{\bf u}\,d^3{\bf r} = - \int\frac{\epsilon_0}{2}\,
\nabla\!\left(E^2 \frac{d\epsilon}{d\rho_m}\,\rho_m\right)\!\cdot\!
{\bf u}\,d^3{\bf r}.
\end{equation}
Hence, Equation~(\ref{e5.75x}) becomes
\begin{equation}
\frac{dW}{dt} = \int\left[-\rho_f \,{\bf E} +\frac{\epsilon_0}{2}\,
E^2\,\nabla\epsilon - \frac{\epsilon_0}{2}\,
\nabla\!\left(E^2\,\frac{d\epsilon}{d\rho_m}\,\rho_m\right)
\right]\!\cdot\! {\bf u}\, d^3{\bf r}.
\end{equation}
Comparing with Equation~(\ref{e5.64x}), we can see that the  force density inside the
dielectric medium is
given by
\begin{equation}\label{e5.79x}
{\bf f} = \rho_f \,{\bf E} - \frac{\epsilon_0}{2} \,E^2\,\nabla\epsilon
+ \frac{\epsilon_0}{2} \,\nabla\!\left(E^2 \frac{d\epsilon}{d\rho_m}\,\rho_m
\right).
\end{equation}
The first term on the right-hand side of the above expression is the standard electrostatic
force density. The second term represents a force which appears whenever
an inhomogeneous dielectric is placed in an electric field. The last 
term, which is known as the {\em electrostriction}\/ term, corresponds to a force acting
on a dielectric placed in an inhomogeneous electric field. Note that the
magnitude of the electrostriction force depends explicitly on the
dielectric equation of state of the material, through $d\epsilon/d\rho_m$. 
The electrostriction  term gives zero net force acting on any finite region
of dielectric, if we can integrate over a large enough portion of the dielectric that its extremities lie in a field-free region. For this reason,
the term is frequently neglected, since  it usually does not contribute to the
total force acting on a dielectric body. 
Note, however, that if the electrostriction
 term is omitted then we obtain an incorrect pressure
variation within the dielectric, despite the fact that the total force is correct.

\section{Clausius-Mossotti Relation}
Let us now investigate  what a dielectric equation of state actually looks like.
Suppose that a dielectric medium is made up of identical molecules which
develop a dipole moment 
\begin{equation}
{\bf p} = \alpha\,\epsilon_0 \,{\bf E} 
\end{equation}
when placed in an electric field ${\bf E}$. The constant $\alpha$ (which has units of volume) is called
the {\em molecular polarizability}. Note that $\alpha$, which is solely a property of the molecule, is typically of order the
molecular volume.
If $N$ is the number density of
 molecules
then the polarization of the medium is
\begin{equation}
{\bf P} = N \,{\bf p} = N\,\alpha \,\epsilon_0 \,{\bf E},
\end{equation}
or
\begin{equation}\label{e3.82x}
{\bf P} = \frac{N_A\, \rho_m\, \alpha}{M} \,\epsilon_0 \,{\bf E},
\end{equation}
where $\rho_m$ is the mass density, $N_A$ is Avogadro's number, and $M$ is 
the molecular weight. But, how does the electric field experienced
by an individual molecule relate to the average electric field in the
medium? This is not a trivial question, since we expect the electric
field to vary strongly (on atomic length-scales) inside the medium.

Suppose that the dielectric is polarized with a uniform mean electric field
${\bf E}_0 = E_0\,{\bf e}_z$. Consider one of the molecules which constitute
the dielectric. Let us draw a sphere of radius $a$ about this particular
molecule. This is intended to represent the boundary between the microscopic
and the macroscopic range of phenomena affecting the molecule. We shall
treat the dielectric outside the sphere as a continuous medium, and the
dielectric inside the sphere as a collection of polarized molecules. 
Note that, unlike the case of a spherical cavity in a dielectric medium,
there is no depolarization of the dielectric immediately outside the sphere,
since there is dielectric material inside the sphere.
Thus, from Equation~(\ref{e5.yyy}), the total field inside the sphere is 
\begin{equation}
{\bf E} = {\bf E}_0 + \frac{\bf P}{3\epsilon_0},
\end{equation}
where ${\bf P}$ is given by Equation~(\ref{e3.82x}).  The second term on the right-hand side of the above equation is the field
at the molecule due to the surface charge on the inside of the sphere.

The field due to the individual molecules within the sphere is
obtained by summing over the dipole fields of these molecules.
The electric field at a distance ${\bf r}$ from a dipole
of moment ${\bf p}$ is (see Exercise~2.4)
\begin{equation}
{\bf E} = -\frac{1}{4\pi\epsilon_0}\left[\frac{\bf p}{r^3}
- \frac{3\,({\bf p}\!\cdot\!{\bf r})\,{\bf r}}{r^5}\right].
\end{equation}
It is assumed that the dipole moment of each molecule within the
sphere is the same, and also that the molecules are evenly distributed
throughout the sphere. This being the case, the value
of $E_z$ at the molecule due to all of the other molecules
within in the sphere,
\begin{equation}
E_z = -\frac{1}{4\pi\epsilon_0}\sum\left[\frac{ p_z\,r^2
- 3\,(p_x\, x\,z + p_y\, y\,z + p_z\, z^2)}{r^5}\right],
\end{equation}
is zero, since
\begin{equation}
\sum z^2 = \frac{1}{3} \sum r^2
\end{equation}
and
\begin{equation}
\sum x\,z = \sum y\,z = 0.
\end{equation}
Here, the sum is over all of the molecules in the sphere.
Furthermore, it is easily demonstrated that $E_\theta = E_\varphi=0$ (where $\theta$ and $\varphi$ are spherical polar coordinates).
Hence, the electric field at the molecule due to the other molecules within
the sphere vanishes.
 
It is clear that the net electric field seen by an individual molecule is
\begin{equation}
{\bf E} = {\bf E}_0 + \frac{\bf P}{3\epsilon_0}.
\end{equation}
This is {\em larger} than the average electric field 
${\bf E}_0$ in the dielectric. The above  analysis indicates
that this effect is ascribable to the long range (rather than the short range)
interactions of the molecule with the other molecules in the medium. 
Making use of Equation~(\ref{e3.82x}), and the definition ${\bf P} =\epsilon_0\,
(\epsilon-1)\, {\bf E}_0$, we obtain
\begin{equation}
\frac{\epsilon -1}{\epsilon+2} = \frac{N_A\,\rho_m\, \alpha}{3 \,M}.
\end{equation}
This formula is called the {\em Clausius-Mossotti}\/ relation, and  is found
to work fairly well for relatively dilute dielectric media whose dielectric constants are close to unity. Incidentally, the right-hand side of this expression is approximately the volume
fraction of space occupied by the molecules making up the medium in question (and, should, therefore, be less than unity).
Finally, the Clausius-Mossotti relation yields
\begin{equation}\label{e5.93x}
\frac{d\epsilon}{d\rho_m} = \frac{(\epsilon-1)\,(\epsilon+2)}{3\,\rho_m}.
\end{equation}

\section{Dielectric Liquids in Electrostatic Fields}
Consider the behaviour of an uncharged dielectric liquid placed in
an electrostatic field. If $p({\bf r})$ is the pressure in the liquid when it is
in equilibrium with the electrostatic force density ${\bf f}({\bf r})$, then
force balance requires  that
\begin{equation}\label{e5.94x}
\nabla p = {\bf f}.
\end{equation}
It follows from Equation~(\ref{e5.79x}) that
\begin{equation}
\nabla p = - \frac{\epsilon_0}{2} \,E^2\,\nabla\epsilon
+ \frac{\epsilon_0}{2} \,\nabla\!\left(E^2 \frac{d\epsilon}{d\rho_m}\,\rho_m
\right)= \frac{\epsilon_0\,\rho_m}{2}\, \nabla\!\left(E^2 \frac{d\epsilon}{d\rho_m}
\right),
\end{equation}
since $\nabla \epsilon = (d\epsilon/d\rho_m)\,\nabla\rho_m$. 
We can integrate  this  equation to  give
\begin{equation}
\int_{p_1}^{p_2} \frac{dp}{\rho_m} =\frac{\epsilon_0}{2}
\left(\left[E^2\frac{d\epsilon}{d\rho_m}\right]_2 - 
\left[E^2\frac{d\epsilon}{d\rho_m}\right]_1\right),
\end{equation}
where 1 and 2 refer to two general points within the liquid. Here, it is
assumed that the liquid possesses an equation of state, so that
$p=p(\rho_m)$. If the liquid is essentially incompressible
({\em i.e.}, $\rho_m\simeq$ constant) then 
\begin{equation}
p_2 - p_1 = \frac{\epsilon_0\,\rho_m}{2} \left[E^2\,\frac{d\epsilon}{d\rho_m}
\right]_1^2.
\end{equation}
Moreover, if the liquid obeys the Clausius-Mossotti relation then
\begin{equation}\label{e5.98x}
p_2 - p_1 = \left[ \frac{\epsilon_0\, E^2}{2} \frac{(\epsilon-1)\,(\epsilon+2)}{3}
\right]_{1}^2.
\end{equation}

According to Equations~(\ref{e5.48x}) and (\ref{e5.98x}), if a sphere of  dielectric
liquid is placed in a uniform electric field ${\bf E}_0$ then the
pressure inside the liquid  takes the constant value
\begin{equation}
p = \frac{3}{2}\, \epsilon_0 \left(\frac{\epsilon-1}{\epsilon + 2}\right) E_0^{\,2}.
\end{equation}
Now, it is fairly clear that the electrostatic
forces acting on the dielectric  are all concentrated at the
edge of the sphere, and are directed radially inwards: {\em i.e.}, the
dielectric is {\em compressed}\/ by the external electric field. This
is a somewhat surprising result, since the electrostatic forces acting on
a rigid conducting sphere are also concentrated at the edge of the sphere, but
are directed radially outwards. We might expect these two cases to give the
same result in the limit $\epsilon\rightarrow \infty$. 
The reason that this does not occur is 
because a dielectric liquid is 
 slightly {\em compressible},
and is,  therefore, subject to an electrostriction force. There is no
electrostriction force for the case of a completely rigid body. 
In fact, the force
density inside a rigid dielectric (for which $\nabla \cdot {\bf u} =0$)
is given by Equation~(\ref{e5.79x}) with the third term on the right-hand side (the electrostriction term)
missing. It is easily seen that the force exerted by an electric
field on a rigid dielectric is directed outwards, and approaches that exerted
on a rigid conductor in the limit $\epsilon\rightarrow 0$.

As is well-known, when a pair of charged (parallel plane) capacitor plates are dipped into
a dielectric liquid, the liquid is drawn up between the plates to
some extent. Let us examine this effect. We can, without loss of
generality, assume that the transition from dielectric to vacuum takes
place in a continuous manner. Consider the electrostatic
pressure difference
between a point $A$ lying just above the surface of the liquid in the region between the
plates, and a point $B$ lying just above the surface of the liquid some region well
away from the capacitor where $E\simeq 0$---see Figure~\ref{fcapt}. The pressure difference is
given by
\begin{equation}\label{e5.100x}
p_A - p_B = -\int_A^B {\bf f}\!\cdot\!d{\bf l},
\end{equation}
where $d{\bf l}$ is an element of some path linking points $A$ and $B$.
Note, however, that the Clausius-Mossotti
relation yields $d\epsilon/d\rho_m=0$ at both $A$ and $B$,  since $\epsilon=1$
in a vacuum [see Equation~(\ref{e5.93x})]. Thus, it is clear from Equation~(\ref{e5.79x})
that the electrostriction term makes no contribution to the   line
integral (\ref{e5.100x}). It follows that
\begin{equation}\label{e5.101x}
p_A - p_B = \frac{\epsilon_0}{2}\int_A^B E^2\, \nabla\epsilon\!\cdot\!d{\bf l}.
\end{equation}
The only contribution to this integral comes from the vacuum/dielectric
interface in the vicinity of point $A$ (since $\epsilon$ is 
constant inside the
liquid, and $E\simeq 0$ in the vicinity of
 point $B$). Suppose that the electric field
at point $A$ has normal and tangential (to the surface) components
$E_\perp$ and $E_\parallel$, respectively. Making use of the
boundary conditions that $D_\perp$ and $E_\parallel$ are constant across
a vacuum/dielectric interface, we obtain
\begin{equation}
p_A - p_B = \frac{\epsilon_0}{2}\left[ E_\parallel^{\,2} \,(\epsilon -1)
+ \frac{D_\perp^{\,2}}{\epsilon_0^{\,2}}\, \int_1^\epsilon\, \frac{d\epsilon}{\epsilon^2}\right],
\end{equation}
giving
\begin{equation}\label{e5.103x}
p_A - p_B = \frac{\epsilon_0\,(\epsilon-1)}{2} \left[E_\parallel^{\,2} +\frac{D_\perp^{\,2}}{\epsilon_0^{\,2}\,\epsilon}\right].
\end{equation}
This electrostatic
pressure difference can be equated to the hydrostatic pressure
difference $\rho_m\, g\, h$ to determine the height $h$ that the liquid
rises between the plates.
At first sight,  the above analysis appears  to suggest that
the dielectric liquid is drawn upward by a  surface force
acting on the vacuum/dielectric interface in the
region between the plates. In fact, this is far from being the
case. A brief examination of Equations~(\ref{e5.94x}) and (\ref{e5.98x}) shows that this surface force 
is actually directed {\em downwards}. 
Indeed, according to Equation~(\ref{e5.79x}), the force which causes the liquid to
rise between the plates
is a volume force which develops in the region of non-uniform electric
field  at the base of the capacitor, where the field splays out  between
the plates. Thus, although we can determine the height to which the fluid
rises between the plates without reference to the electrostriction force,
it is, somewhat paradoxically, this force which is actually
responsible for supporting the liquid against gravity. 
\begin{figure}
 \epsfysize=2.in
\centerline{\epsffile{chapter6/fig6.5.eps}}
\caption{\em Two capacitor plates dipped in a dielectric liquid.}\label{fcapt}
\end{figure}

Let us consider another paradox concerning the electrostatic forces
exerted in a dielectric medium. Suppose that we have two charges
embedded in a uniform dielectric medium of dielectric constant $\epsilon$. The electric field generated
by each charge is the same as that in vacuum, except that it is reduced
by a factor $\epsilon$. Therefore, we would expect that the force exerted by 
one charge on another is the same as that in vacuum, except that it
is also reduced by a factor $\epsilon$. Let us examine how
this reduction in force comes about. Consider a simple example.
Suppose that we take a parallel plate capacitor, and insert a block
of solid dielectric between the plates. Suppose, further, that there is
a small vacuum gap between the faces of the block and each of
the capacitor plates. Let $\pm\sigma$ be the surface charge densities
on each of the capacitor plates, and let $\pm\sigma_b$ be the bound
surface charge densities  which develop on the outer faces of the intervening dielectric
block. The two layers of polarization charge produce equal and opposite 
electric fields on each plate, and their effects therefore cancel each other.
Thus, from the point of view of electrical interaction alone, there would 
appear to be no change in the force exerted by one capacitor plate on the
 other
when a dielectric slab is placed between them (assuming that $\sigma$ remains
constant during this process). That is, the force per unit
area (which is attractive) remains
\begin{equation}\label{e5.104x}
f = \frac{\sigma^2}{2\epsilon_0}
\end{equation}
---see Equation~(\ref{eforcc}).
However, in experiments in which a capacitor is submerged
in a dielectric liquid, the force per unit area exerted by one plate
on another is observed to decrease to
\begin{equation}
f = \frac{\sigma^2}{2\epsilon_0\,\epsilon}.
\end{equation}


This apparent paradox can be explained by taking into account the
difference in liquid pressure in the field-filled space
between the plates, and the field-free region outside the capacitor. 
This pressure difference is balanced by internal elastic forces in the case
of a solid dielectric, but is transmitted to the
plates in the case of the liquid. We can compute the pressure difference
between a point $C$ on the inside surface of one of the capacitor
plates, and a point $D$ on the outside surface of the same plate using
Equation~(\ref{e5.101x})---see Figure~\ref{fcapt}. If we neglect end effects, then the electric field is normal
to the plates in the region between the plates, and is zero
everywhere else. Thus, the only contribution to the line integral
(\ref{e5.101x}) comes from the plate/dielectric interface in the vicinity of
point $C$. Adopting Equation~(\ref{e5.103x}), we find that
\begin{equation}
p_C- p_D = \frac{\epsilon_0}{2}\left(1-\frac{1}{\epsilon}\right)
E_\perp^{\,2} = \frac{\sigma^2}{2\epsilon_0}\left(1-\frac{1}{\epsilon}\right),
\end{equation}
where $E_\perp=\sigma/\epsilon_0$ is the normal field strength between the plates in
the absence of the dielectric. The sum of this pressure force (which is repulsive) and
the attractive electrostatic force per unit area (\ref{e5.104x}) yields a net attractive
force per unit area of
\begin{equation}
f= \frac{\sigma^2}{2\epsilon_0\,\epsilon}
\end{equation}
acting between the plates. Thus, any decrease in the forces exerted
by charges on one another
when they are immersed, or embedded, in a dielectric medium
can only be understood in terms of mechanical forces transmitted
between the charges by the medium itself. 

\section{Polarization Current}
We have seen that the bound charge density is related to the polarization field
via
\begin{equation}
\rho_b = - \nabla\cdot{\bf P}.
\end{equation}
Now, it is clear, from this equation, that if the polarization field  inside
some dielectric material changes in time then the distribution of bound charges will
also change. Hence, in order to conserve charge, a net current must flow.
This current is known as the {\em polarization current}. Charge conservation
implies that
\begin{equation}
\nabla\cdot {\bf j}_p + \frac{\partial \rho_b}{\partial t} = 0,
\end{equation}
where ${\bf j}_p$ is the polarization current density. It follows from the previous two equations that
\begin{equation}\label{e6.44x}
{\bf j}_p = \frac{\partial {\bf P}}{\partial t}.
\end{equation}
Note that the polarization current is a real current, despite the fact that
it is generated by the rearrangement of bound charges. There is, however,
no drifting of real charges over length-scales longer than atomic or molecular
length-scales associated with this current.

\section{Magnetization}
All matter is built up out of  molecules, and each molecule consists of
electrons in motion around stationary nuclii. The currents
associated with this type of electron motion 
are termed {\em molecular currents}. Each molecular current is
a tiny closed circuit of molecular dimensions, and may therefore be 
appropriately described as a magnetic dipole. Suppose that a given
molecule has a magnetic dipole moment ${\bf m}$. 
If there are $N$ such  molecules per unit volume then the {\em magnetization}\/ ${\bf M}$
({\em i.e.}, the magnetic dipole moment per unit volume) is given
by ${\bf M} = N\, {\bf m}$. More generally,
\begin{equation}
{\bf M} ({\bf r}) = \sum_i N_i\, \langle {\bf m}_i\rangle,
\end{equation}
where $\langle {\bf m}_i\rangle$ is the average magnetic dipole moment
of the $i$th type of molecule,
and $N_i$ is the average number of such molecules per unit volume, in the vicinity of point ${\bf r}$. 

Now, we saw earlier, in Exercise~2.20, that the vector potential field generated by a
magnetic dipole of moment ${\bf m}$ situated at the origin is
\begin{equation}
{\bf A}({\bf r}) = \frac{\mu_0}{4\pi}\,\frac{{\bf m}\times {\bf r}}{r^3}.
\end{equation}
Hence, from the principle of superposition, the vector potential field
generated by a magnetic medium of magnetic moment per unit volume
${\bf M}({\bf r})$ is
\begin{equation}
{\bf A}({\bf r}) = \frac{\mu_0}{4\pi}\int \frac{{\bf M}({\bf r}')\times ({\bf r}-{\bf r}')}{|{\bf r}-{\bf r}'|^3}\,d^3{\bf r}',
\end{equation}
where the volume integral is taken over all space.
 However, it follows from Equations~(\ref{e315}) and (\ref{e2.145h}) that
\begin{equation}
{\bf A}({\bf r}) = \frac{\mu_0}{4\pi}\int {\bf M}({\bf r}')\times \nabla'\left(\frac{1}{|{\bf r}-{\bf r}'|}\right)\,d^3{\bf r}'.
\end{equation}
Now, it is easily demonstrated that
\begin{equation}
\int {\bf f}\times \nabla g\,d^3{\bf r}= \int g\,\nabla\times {\bf f}\,\,d^3{\bf r},
\end{equation}
provided that the integral is over all space, and $g\,|{\bf f}|\rightarrow 0$ as
$|{\bf r}|\rightarrow\infty$. Hence, for a magnetization field of
finite extent, we can write
\begin{equation}
{\bf A}({\bf r}) = \frac{\mu_0}{4\pi}\int \frac{{\bf j}_m({\bf r}')}{|{\bf r}-{\bf r}'|}\,d^3{\bf r}',
\end{equation}
where
\begin{equation}
{\bf j}_m = \nabla\times {\bf M}.
\end{equation}
It follows, by comparison with Equation~(\ref{e3169b}), that the curl
of the magnetization field constitutes a current density. The associated
current is known as the {\em magnetization current}.

The total current density, ${\bf j}$, in a general medium takes the form
\begin{equation}\label{ejmag}
{\bf j} = {\bf j}_t + \nabla\times{\bf M} + \frac{\partial{\bf P}}{\partial t},
\end{equation}
where the three terms on the right-hand side represent the {\em true current density}\/
({\em i.e.}, that part of the current density which is due to the movement of free charges), the magnetization current density, and the polarization current density,
respectively.
It must be emphasized that all three terms represent
real physical currents, although only the first term is due to the motion
of real charges (over more than molecular dimensions). 

Now, the differential form of Amp\`{e}re's law is
\begin{equation}
\nabla\times{\bf B} = \mu_0\,{\bf j} + \mu_0\epsilon_0\,\frac{\partial {\bf E}}
{\partial t},
\end{equation}
which can also be written
\begin{equation}
\nabla\times{\bf B} = \mu_0\,{\bf j}_t +\mu_0\,\nabla\times{\bf M}
+ \mu_0\,\frac{\partial{\bf D}}{\partial t},
\end{equation}
where use has been made of Equation~(\ref{ejmag}) and the definition ${\bf D} = \epsilon_0\,{\bf E} +
{\bf P}$. The above expression can be rearranged to give
\begin{equation}\label{e5.121x}
\nabla\times{\bf H} = {\bf j}_t + \frac{\partial{\bf D}}{\partial t},
\end{equation}
where
\begin{equation}
{\bf H} = \frac{\bf B}{\mu_0} - {\bf M}
\end{equation}
is termed the {\em magnetic intensity}, and has the same dimensions
as ${\bf M}$ ({\em i.e.}, magnetic dipole moment per unit volume). 
In a steady-state situation, Stokes' theorem tell us that
\begin{equation}
\oint_C {\bf H}\!\cdot\!d{\bf l} = \int_S {\bf j}_t\!\cdot\!d{\bf S}.
\end{equation}
In other words, the line integral of ${\bf H}$ around some closed loop
is equal to the flux of the true  current through any surface attached to that
loop. Unlike the magnetic field ${\bf B}$ (which specifies
the magnetic force $q \,{\bf v}\times {\bf B}$ acting on a charge $q$ moving
with velocity ${\bf v}$),
or the magnetization ${\bf M}$ (which is the magnetic dipole moment
per unit volume), the magnetic intensity ${\bf H}$ has no clear physical
interpretation. The only reason for introducing this quantity is that it enables us to
calculate magnetic fields in the presence of magnetic materials without
having to know the distribution of magnetization currents beforehand. 
However, this is only possible if we possess a constitutive relation
connecting ${\bf B}$ and ${\bf H}$. 

\section{Magnetic Susceptibility and Permeability}
In a large class of magnetic materials, there exists an approximately linear
relationship between ${\bf M}$ and ${\bf H}$. If the material
is isotropic then
\begin{equation}\label{e6.124}
{\bf M} = \chi_m\, {\bf H},
\end{equation}
where $\chi_m$ is called the {\em magnetic susceptibility}. If $\chi_m$ is
positive then the material is called {\em paramagnetic}, and the magnetic field
is strengthened by the presence of the material. On the other hand, if $\chi_m$ is
negative then the material is {\em diamagnetic}, and the magnetic field
is weakened in the presence of the material. The magnetic
susceptibilities of paramagnetic and diamagnetic materials are
generally extremely small. A few example values of $\chi_m$ are given in Table~\ref{tab1}.

\begin{table}\centering
\begin{tabular}{ll}\hline
{\bf Material} & {\bf Magnetic susceptibility}\\ \hline
Aluminium & $+2.3\times 10^{-5}$ \\ 
Copper & $-9.8\times 10^{-6}$ \\
Diamond & $-2.2\times 10^{-5}$ \\
Tungsten & $+6.8\times 10^{-5}$ \\
Hydrogen (1 atm) & $-2.1\times 10^{-9}$ \\
Oxygen (1 atm) & $+2.1\times 10^{-6}$ \\
Nitrogen (1 atm) & $-5.0\times 10^{-9}$\\ \hline
\end{tabular}
\caption{\em Low-frequency magnetic susceptibilities of some common materials.}\label{tab1}
\end{table}

A linear relationship between ${\bf M}$ and ${\bf H}$ also
implies a linear relationship between ${\bf  B}$ and ${\bf H}$.
In fact, we can write 
\begin{equation}\label{e6.125}
{\bf B} = \mu_0\,\mu\,{\bf H},
\end{equation}
where
\begin{equation}
\mu = 1+ \chi_m
\end{equation}
is termed the {\em relative magnetic permeability}\/ of the material
in question.
(Likewise,
$\mu_0$ is termed the {\em permeability of free space}.) Note that $\mu$
is dimensionless.
It is clear from
Table~\ref{tab1} that the relative permeabilities of common diamagnetic and paramagnetic
materials do not differ substantially from unity. In fact,
to all intents and purposes, the magnetic properties of such materials
can be safely neglected. 

\section{Ferromagnetism}
There exists, however, a third class of magnetic materials called
{\em ferromagnetic}\/ materials. Such materials are characterized by a
possible permanent magnetization, and generally have a profound effect
on magnetic fields ({\em i.e.}, $\mu\gg 1$). 
Unfortunately, ferromagnetic materials {\em do not}\/ generally exhibit a linear
dependence between ${\bf M}$ and ${\bf H}$, or between ${\bf B}$ and ${\bf H}$,
so that we cannot employ Equations~(\ref{e6.124}) and (\ref{e6.125}) with constant values
of $\chi_m$ and $\mu$. It is still  expedient to use  Equation~(\ref{e6.125}) as the
definition of  $\mu$, with $\mu = \mu({\bf H})$. However,
this practice can lead to difficulties under certain circumstances.
The permeability of a ferromagnetic material, as defined by
Equation~(\ref{e6.125}), can vary through the entire range of possible values
from zero to infinity, and may be either positive or negative. The most
sensible  
approach is to consider each problem involving ferromagnetic materials
separately, try to determine which region of the ${\bf B}$--${\bf H}$ diagram
is important for the particular case in hand, and then
make approximations appropriate
to this region. 



First, let us consider an unmagnetized sample of ferromagnetic material.
If the magnetic intensity, which is initially zero, is increased
{\em monotonically}, then the ${\bf B}$--${\bf H}$ relationship
traces out a curve such as that shown schematially in Figure~\ref{fbh}. This is called a
{\em magnetization curve}. It is evident that the permeabilities
$\mu$ derived from the curve (according to the rule $\mu = \mu_0^{-1}\,B/H$) are
always positive, and show a wide range of values. The maximum permeability
occurs at the ``knee'' of the curve. In some materials, this
maximum permeability is as large as $10^5$. The reason for
the knee is that the magnetization ${\bf M}$ reaches
a maximum value in the material, so that
\begin{equation}\label{e5.127x}
{\bf B} = \mu_0\,({\bf H} + {\bf M})
\end{equation}
continues to increase at large ${\bf H}$ only because of the
$\mu_0\,{\bf H}$ term. The maximum value of ${\bf M}$ is
called the {\em saturation magnetization} of the material. Incidentally, it is clear
from the above equation that $\mu=1$ for a fully saturated magnetic
material in which $|{\bf H}|\gg |{\bf M}|$.
\begin{figure}
\epsfysize=2.5in
\centerline{\epsffile{chapter6/fig6.6.eps}}
\caption{\em A typical magnetization curve for a ferromagnet.}\label{fbh}
\end{figure}

Next, consider a ferromagnetic sample magnetized by the above procedure.
If the magnetic intensity ${\bf H}$ is decreased then the ${\bf B}$--${\bf H}$
relation does not return along the curve shown in Figure~\ref{fbh}, but instead
moves along a new curve, which is sketched in Figure~\ref{fhst}, to the point $R$. 
Thus, the magnetization, once established, does not disappear with the removal
of ${\bf H}$. In fact, it takes a reversed magnetic intensity to
reduce the magnetization to zero. If ${\bf H}$ continues to
build up in the reversed direction, then ${\bf M}$ (and,
hence, ${\bf B}$) becomes increasingly negative. Finally,
when ${\bf H}$ increases again the operating point follows the lower
curve in Figure~\ref{fhst}. Thus, the ${\bf B}$--${\bf H}$ curve for
increasing ${\bf H}$ is quite different to that for decreasing
${\bf H}$. This phenomenon is known as {\em hysteresis}.
 

The curve sketched in Figure~\ref{fhst} called the {\em hysteresis loop}\/ of the ferromagnetic material
in question. The value of ${\bf B}$ at the point $R$ is called the
{\em retentivity}\/ or {\em remanence}. The magnitude of ${\bf H}$ at
the point $C$ is called the {\em coercivity}. It is
evident that $\mu$ is negative in the second and fourth quadrants
of the loop, and positive in the first and third quadrants. The shape
of the hysteresis loop depends not only on the nature of
the ferromagnetic material, but also on the maximum value of $|{\bf H}|$
to which the material has been subjected. However, once this maximum
value, $|{\bf H}|_{\rm max}$, becomes sufficiently large to produce saturation in the material, the hysteresis loop does not change shape with any further
increase in $|{\bf H}|_{\rm max}$. 
\begin{figure}
\epsfysize=2.in
\centerline{\epsffile{chapter6/fig6.7.eps}}
\caption{\em A typical hysteresis loop for a ferromagnet.}\label{fhst}
\end{figure}


Ferromagnetic materials are used either to channel magnetic flux
({\em e.g.}, around transformer circuits) or as sources of magnetic
field ({\em e.g.}, permanent magnets). For use as a permanent magnet, the
material is first magnetized by placing it in a strong magnetic
field. However, once the magnet is removed from the external field
it is subject to a demagnetizing ${\bf H}$. Thus, it is vitally important
that a permanent magnet should possess both a large remanence and a large
coercivity. As will become
clear later on, it is generally a good idea for the ferromagnetic materials
used to channel magnetic flux around transformer circuits to
possess small remanences and small coercivities. 

\section{Boundary Conditions for ${\bf B}$ and ${\bf H}$}
What are the boundary conditions for ${\bf B}$ and ${\bf H}$ at
the interface between two magnetic media? Well, the governing equations for a steady-state situation are
\begin{equation}\label{e6.128}
\nabla\!\cdot\!{\bf B} = 0,
\end{equation}
and
\begin{equation}\label{e6.129}
\nabla\times{\bf H} = {\bf j}_t.
\end{equation}
Integrating Equation~(\ref{e6.128}) over a thin Gaussian pill-box $S$ enclosing part of the
interface between the two media gives
\begin{equation}\label{e5.130x}
B_{\perp\,1}-B_{\perp\,2}= 0,
\end{equation}
where $B_{\perp\,1}$ denotes the component of ${\bf B}$ perpendicular to
the interface in medium 1, {\em etc.}---see Figure~\ref{f43b}.
 Integrating Equation~(\ref{e6.129}) around a narrow loop $C$ which
straddles the interface yields
\begin{equation}\label{e5.131x}
H_{\parallel\,1}-H_{\parallel\,2} = 0,
\end{equation}
assuming that there is no true  current sheet flowing at the interface---see Figure~\ref{f43b}.
Here, $H_{\parallel\,1}$ denotes the component of ${\bf H}$ parallel to the
interface in medium 1, {\em etc}.
In general, there is a magnetization current sheet flowing
at the interface between two magnetic materials whose density is 
\begin{equation}
{\bf J}_m= ({\bf M}_{1}-{\bf M}_{2})\times {\bf n},
\end{equation}
where ${\bf n}$ is the unit normal to the interface (pointing from medium 1 to medium 2), and ${\bf M}_{1,2}$ are the magnetizations in the two
media.
\begin{figure}
\epsfysize=1.75in
\centerline{\epsffile{chapter6/fig6.8.eps}}
\caption{\em The boundary between two different magnetic media.}\label{f43b}
\end{figure}

\section{Boundary Value Problems with Ferromagnets}
Consider a ferromagnetic sphere of permanent magnetization ${\bf M} = M\,{\bf e}_z$, where $M$ is a constant. What is the magnetic field generated by such a sphere? Suppose
that the sphere is of radius $a$, and is centered on the origin. From Equation~(\ref{e5.121x}), we
have 
\begin{equation}
\nabla\times {\bf H} = {\bf 0},
\end{equation}
since this is a time independent problem with no true currents. It follows that
\begin{equation}
{\bf H} = - \nabla\phi_m,
\end{equation}
where $\phi_m$ is termed the {\em magnetic scalar potential}. Now,
\begin{equation}
{\bf H} = \frac{\bf B}{\mu_0} - {\bf M},
\end{equation}
which implies that
\begin{equation}
\nabla\cdot {\bf H} = 0
\end{equation}
everywhere (apart from on the surface of the sphere), since $\nabla\cdot{\bf B} = 0$, and ${\bf M}$ is constant inside the sphere, and zero outside.
It follows that the magnetic scalar potential satisfies Laplace's equation,
\begin{equation}
\nabla^2\phi_m=0,
\end{equation}
both inside and outside the sphere.

Adopting spherical polar coordinates,
$(r, \theta, \varphi)$, aligned along the $z$-axis, the boundary conditions are that $\phi_m$ is well-behaved at $r=0$, and
$\phi_m\rightarrow 0$ as $r\rightarrow\infty$. Moreover, Equation~(\ref{e5.131x})
implies that $\phi_m$ must be continuous at $r=a$, whereas Equations~(\ref{e5.127x}) and (\ref{e5.130x}) yield
\begin{equation}
-\left[\frac{\partial \phi_m}{\partial r}\right]_{a-}^{a+} = {\bf M}\cdot{\bf e}_r,
\end{equation}
or
\begin{equation}\label{e5.139}
\left.\frac{\partial\phi_m}{\partial r}\right|_{r=a+}  - \left.\frac{\partial\phi_m}{\partial r}\right|_{r=a-} = -M\,\cos\theta.
\end{equation}

Let us try separable solutions of the form $r^m\,\cos\theta$. It is
easily demonstrated that such solutions satisfy Laplace's equation
provided that $m=1$ or $m=-2$. Hence, the most general solution to Laplace's equation outside
the sphere, which satisfies the boundary condition at $r\rightarrow\infty$, is
\begin{equation}
\phi_m(r,\theta) = C\,\frac{a^3\,\cos\theta}{r^2}.
\end{equation}
Likewise, the most general solution inside the sphere, which satisfies
the boundary condition at $r=0$, is
\begin{equation}
\phi_m(r,\theta) = D\,r\,\cos\theta.
\end{equation}

The continuity of $\phi_m$ at $r=a$ gives
$C = D$,
whereas the boundary condition (\ref{e5.139}) yields $C=M/3$. 
Hence, the magnetic scalar potential takes the form
\begin{equation}
\phi_m = \frac{M}{3} \,\frac{a^3}{r^2}\,\cos\theta
\end{equation}
outside the sphere, and
\begin{equation}
\phi_m = \frac{M}{3}\,r\,\cos\theta
\end{equation}
inside the sphere. It follows that
\begin{eqnarray}
{\bf H} &=& - {\bf M}/3,\\[0.5ex]
{\bf B}/\mu_0 &=& 2\,{\bf M}/3
\end{eqnarray}
inside the sphere. Hence, both the magnetic field and the magnetic intensity
are uniform and parallel or anti-parallel to the permanent magnetization within the sphere---see Figure~\ref{fex8}. Note, however, that the sphere is subject to a {\em demagnetizing}\/ magnetic intensity ({\em i.e.}, ${\bf H}\propto - {\bf M}$).
\begin{figure}
 \epsfysize=3.in
\centerline{\epsffile{chapter6/fig6.9.eps}}
\caption{\em Equally spaced coutours of the axisymmetric potential
$\phi_m(r,\theta)$ generated by a ferromagnetic sphere of unit radius uniformly magnetized in the $z$-direction.}\label{fex8}
\end{figure}


It is easily demonstrated that the scalar magnetic potential due to a magnetic
dipole of moment ${\bf m}$ at the origin is
\begin{equation}
\phi_m({\bf r}) = \frac{1}{4\pi}\,\frac{{\bf m}\cdot{\bf r}}{r^3}.
\end{equation}
Thus, it is clear that the magnetic field outside the sphere is the same as
that of a magnetic dipole of moment
\begin{equation}
{\bf m} = \frac{4}{3}\,\pi\,a^3\,{\bf M}
\end{equation}
at the origin. This, of course, is the permanent magnetic dipole moment of the sphere.
Finally, the magnetization sheet current density at the surface of the sphere is given by
\begin{equation}
{\bf J}_m = {\bf M} \times {\bf e}_r = M\,\sin\theta\,{\bf e}_\varphi.
\end{equation}

Consider a ferromagnetic sphere, of uniform permeability $\mu$, placed in
a uniform $z$-directed magnetic field of magnitude $B_0$. Suppose
that the sphere is centred on the origin. In the absence of any true currents,
we have $\nabla\times{\bf H} = {\bf 0}$. Hence, we can again write ${\bf H} = -\nabla\phi_m$. Given that $\nabla\cdot{\bf B}=0$, and ${\bf B} = \mu_0\,\mu\,{\bf H}$, it follows that $\nabla^2\phi_m=0$ in any uniform magnetic medium
(or a vacuum). Thus, $\nabla^2\phi_m=0$ throughout space. Adopting spherical polar coordinates,
$(r, \theta, \varphi)$, aligned along the $z$-axis, the boundary
conditions are that $\phi_m \rightarrow - (B_0/\mu_0)\,r\,\cos\theta$ as $r\rightarrow\infty$, and that $\phi_m$ is well-behaved at $r=0$.  At the surface of the sphere, $r=a$, the continuity of $H_\parallel$
 implies that $\phi_m$ is continuous. Furthermore, the
continuity of $B_\perp=\mu_0\,\mu\,H_\perp$  leads to the matching condition
\begin{equation}\label{emat1}
\left.\frac{\partial \phi_m}{\partial r}\right|_{r=a+} = \left.\mu\,
\frac{\partial\phi_m}{\partial r}\right|_{r=a-}.
\end{equation}

Let us again try separable solutions of the form $r^m\,\cos\theta$. The most general solution to Laplace's equation outside
the sphere, which satisfies the boundary condition at $r\rightarrow\infty$, is
\begin{equation}
\phi_m(r,\theta) = - (B_0/\mu_0)\,r\,\cos\theta + (B_0/\mu_0)\,\alpha \,\frac{a^3\,\cos\theta}{r^2}.
\end{equation}
Likewise, the most general solution inside the sphere, which satisfies
the boundary condition at $r=0$, is
\begin{equation}
\phi_m(r,\theta) = - (B_1/\mu_0\,\mu)\,r\,\cos\theta.
\end{equation}
The continuity of $\phi_m$ at $r=a$ yields
\begin{equation}
B_0 - B_0\,\alpha = B_1/\mu.
\end{equation}
Likewise, the matching condition (\ref{emat1}) gives
\begin{equation}
B_0 + 2\,B_0\,\alpha = B_1.
\end{equation}
Hence,
\begin{eqnarray}
\alpha &=& \frac{\mu-1}{\mu+2},\\[0.5ex]
B_1 &=& \frac{3\,\mu\,B_0}{\mu+2}.
\end{eqnarray}
Note that the magnetic field inside the sphere is {\em uniform}, parallel
to the external magnetic field outside the sphere, and of magnitude $B_1$. Moreover, $B_1 > B_0$,  provided that $\mu>1$. The magnetization inside
the sphere is also uniform and parallel to the external magnetic field. In fact,
\begin{equation}
{\bf M} = \frac{3\,(\mu-1)}{\mu+2}\,\frac{B_0}{\mu_0}\,{\bf e}_z.
\end{equation}
The magnetic field outside the sphere is that due to the external field plus
the field of a magnetic dipole of moment ${\bf m} = (4/3)\,\pi\,a^3\,{\bf M}$.
This is, of course, the induced magnetic dipole moment of the sphere.
Finally, the magnetization sheet current density at the surface of the
sphere is ${\bf J}_m = {\bf M}\times {\bf e}_r  = M\,\sin\theta\,{\bf e}_\varphi$.


As a final example, consider an electromagnet of the form sketched  in Figure~\ref{mc}. A wire, carrying a current $I$, is wrapped $N$ times
around a thin toroidal iron core of radius $a$ and permeability $\mu\gg 1$. The core contains
a thin gap of width $d$. What is the magnetic field induced in the
gap? 
Let us neglect any leakage of magnetic flux from the core, which is
reasonable if $\mu\gg 1$. We expect the magnetic field, $B_c$,
and the magnetic intensity, $H_c$, in the core to both be toroidal and essentially
uniform. It is also reasonable to suppose that the magnetic field, $B_g$, and the
magnetic intensity, $H_g$, in the gap are toroidal and uniform, since
$d\ll a$. We have $B_c = \mu_0\,\mu\,H_c$ and $B_g=\mu_0\,H_g$. 
Moreover, since the magnetic field is normal to the interface between the
core and the gap, the continuity of $B_\perp$ implies that
\begin{equation}
B_c = B_g.
\end{equation}
Thus, the magnetic field-strength in the core is the same as that in the
gap. However, the magnetic intensities in the core and the gap are
quite different: $H_c = B_c/\mu_0\,\mu = B_g/\mu_0\,\mu = H_g/\mu$.
Integration of Equation~(\ref{e6.129}) around the torus yields
\begin{equation}
\oint {\bf H}\cdot d{\bf l} = \int{\bf j}_t\cdot d{\bf S} = N\,I.
\end{equation}
Hence,
\begin{equation}
(2\pi\,a-d)\,H_c + d\,H_g = N\,I.
\end{equation}
It follows that
\begin{equation}
B_g = \frac{N\,\mu_0\,I}{(2\pi\,a-d)/\mu + d}.
\end{equation}
Note that if $\mu\gg 1$ and $d\ll 2\pi\,a$ then the magnetic field in the
gap is considerably larger than that which would be obtained if the
core of the electromagnet were not ferromagnetic.
\begin{figure}
\epsfysize=2.in
\centerline{\epsffile{chapter6/fig6.10.eps}}
\caption{\em An electromagnet.}\label{mc}
\end{figure}

\section{Magnetic Energy}
Consider an electrical conductor. Suppose that a battery with an
electromotive field ${\bf E}'$ is feeding energy into this conductor.
The energy is either dissipated as heat, or is used to generate a
magnetic field. Ohm's law inside the conductor gives
\begin{equation}
{\bf j}_t = \sigma\, ({\bf E} + {\bf E}'),
\end{equation}
where ${\bf j}_t$ is the true current density, $\sigma$ is the
conductivity, and ${\bf E}$ is the inductive electric field. Taking
the scalar product with ${\bf j}_t$, we obtain
\begin{equation}\label{e6.186}
{\bf E}'\!\cdot\!{\bf j}_t =  \frac{j_t^{\,2}}{\sigma}
 -{\bf E}\cdot{\bf j}_t.
\end{equation}
The left-hand side of this equation represents the rate at which the
battery does work on the conductor. The first term on the right-hand
side is the rate of ohmic heating inside the conductor. Thus,  the remaining  term must represent the rate at which energy is fed into
the magnetic field. If all fields are quasi-static
({\em i.e.}, slowly varying) then the displacement current can be neglected,
and the differential form of Amp\`{e}re's law reduces to $\nabla\times{\bf H}
= {\bf j}_t$. Substituting this expression into Equation~(\ref{e6.186}),
and integrating over all space, we get
\begin{equation}
\int {\bf E}'\!\cdot\!(\nabla\times{\bf H})\,d^3{\bf r}
= \int\frac{(\nabla\times{\bf H})^2}{\sigma}
\,d^3{\bf r} - \int {\bf E}\!\cdot\!(\nabla\times{\bf H})\,
d^3{\bf r}.
\end{equation}
The last term can be integrated by parts using the vector identity
\begin{equation}
\nabla\!\cdot\!({\bf E}\times{\bf H}) \equiv {\bf H}\!\cdot\!(\nabla\times
{\bf E}) -  {\bf E}\!\cdot\!(\nabla\times{\bf H}).
\end{equation}
Gauss' theorem plus the differential form of Faraday's law yield
\begin{equation}
\int{\bf E}\!\cdot\!(\nabla\times{\bf H})\,d^3{\bf r} 
= -\int {\bf H}\!\cdot\!\frac{\partial{\bf B}}{\partial t}\,d^3{\bf r}
- \int({\bf E}\times{\bf H})\!\cdot\!d{\bf S}.
\end{equation}
Since ${\bf E}\times{\bf H}$ falls off at least  as fast as $1/r^5$ in
quasi-static electric and magnetic fields ($1/r^2$ comes from electric
monopole fields, and $1/r^3$ from magnetic dipole fields), the surface
integral in the above expression can be neglected. Of course, this is
not the case for radiation fields, for which ${\bf E}$ and ${\bf H}$
both fall off like $1/r$. Thus, the  ``quasi-static'' constraint effectively
means that the fields vary sufficiently slowly that any radiation fields
can be neglected. 

The total power expended by the battery can now be written
\begin{equation}
\int {\bf E}'\!\cdot\!(\nabla\times{\bf H})\,d^3{\bf r}
= \int\frac{(\nabla\times{\bf H})^2}{\sigma}
\,d^3{\bf r} + \int {\bf H}\!\cdot\!\frac{\partial{\bf B}}{\partial t}
\,d^3{\bf r}.
\end{equation}
The first term on the right-hand side has already been identified as the
energy 
loss rate due to ohmic  heating, and the second as  the
rate at which energy is fed into the magnetic field. The variation
$\delta W$ in the magnetic field energy can therefore be written
\begin{equation}\label{e6.191}
\delta W = \int {\bf H}\!\cdot\!\delta {\bf B}\,d^3{\bf r}.
\end{equation}
This result is analogous to the result (\ref{e6.37}) for the variation in the
energy of an electrostatic field. 

In order to make Equation~(\ref{e6.191}) integrable, we must assume a functional
relationship between ${\bf H}$ and ${\bf B}$. For a medium which
magnetizes linearly, the integration can be carried out in an analogous manner to that used to derive Equation~(\ref{e6.41}), to give
\begin{equation}
W  = \frac{1}{2} \int {\bf H}\!\cdot\!{\bf B} \,d^3{\bf r}.
\end{equation}
Thus, the magnetostatic energy density inside a linear magnetic
material is given by
\begin{equation}\label{e6en}
U=\frac{1}{2}\, {\bf H}\!\cdot{\bf B}.
\end{equation}
Unfortunately, most interesting magnetic materials, such as ferromagnets,
exhibit a nonlinear relationship between ${\bf H}$ and ${\bf B}$. 
For such materials, Equation~(\ref{e6.191}) can only be integrated between
definite states, and the result, in general, depends on the past
history of the sample. For ferromagnets, the integral of Equation~(\ref{e6.191}) has
a finite, non-zero value when ${\bf B}$ is integrated around a
complete magnetization  cycle. This cyclic energy loss
is given by
\begin{equation}
\Delta W = \int \oint {\bf H}\!\cdot d{\bf B} \,d^3{\bf r}.
\end{equation}
In other words, the energy expended per unit volume when a magnetic
material is carried around a magnetization cycle is equal to the {\em area} of
its hysteresis loop, as plotted in a graph of $B$ against $H$. Thus,
it is particularly important to ensure that the magnetic
materials used to form
transformer cores possess hysteresis loops with comparatively small
areas, otherwise the transformers are  likely to be extremely lossy. 

{\small
\section{Exercises}
\renewcommand{\theenumi}{6.\arabic{enumi}}
\begin{enumerate}
\item An infinite slab of dielectric of uniform dielectric constant $\epsilon$ lies
between the planes $z=-a$ and $z=a$. Suppose that the slab contains
free charge of uniform charge density $\rho_f$. Find ${\bf E}$, ${\bf D}$,
and ${\bf P}$ as functions of $z$. What is the bound charge sheet
density on the two faces of the slab?
\item An infinite slab of uncharged dielectric of uniform dielectric constant $\epsilon$ lies
between the planes $z=-a$ and $z=a$, and is placed in a uniform
electric field ${\bf E}$ whose field-lines make an angle $\theta$ with the $z$-axis. What is the bound charge sheet
density on the two faces of the slab?
\item An uncharged  dielectric sphere of radius $a$, centered on the origin, possesses a polarization field ${\bf P} = p\,{\bf r}$, where $p$ is a constant. Find the bound charge density inside the sphere, and the bound charge sheet
density on the surface of the sphere.
Find ${\bf E}$ and ${\bf D}$ both inside and outside the sphere.
\item An infinite dielectric of dielectric constant $\epsilon$ contains
a uniform electric field ${\bf E}_0$.  Find the electric field inside
a needle-shaped cavity running parallel to ${\bf E}_0$. Find the field
inside a wafer-shaped cavity aligned perpendicular to ${\bf E}_0$. Neglect
end effects.
\item Consider a plane interface between two uniform dielectrics of dielectric
constants $\epsilon_1$ and $\epsilon_2$. A straight electric field-line 
which passes across the interface is bent at an angle. Demonstrate that
$$
\epsilon_1\,\tan\theta_2= \epsilon_2\,\tan\theta_1,
$$
where $\theta_1$ is the angle the field-line makes with the
normal to the interface in medium 1, {\em etc.}
\item A charge $q$ lies at the center of an otherwise uncharged dielectric sphere of radius $a$
and uniform dielectric constant $\epsilon$. Find ${\bf D}$ and ${\bf E}$
throughout space. Find the bound charge sheet density on the surface of the
sphere.
\item A cylindrical coaxial cable consists of an inner conductor of radius $a$,
surrounded by a dielectric sheath of dielectric constant $\epsilon_1$ and outer radius $b$, surrounded  by a second dielectric sheath of
dielectric constant $\epsilon_2$ and outer radius $c$, surrounded
by an outer conductor. All components of the cable are touching. What
is the capacitance per unit length of the cable?
\item A long dielectric cylinder of radius $a$ and uniform dielectric constant
$\epsilon$ is placed in a uniform electric field ${\bf E}_0$ which
runs perpendicular to the axis of the cylinder. Find the electric
field both inside and outside the cylinder. Find the bound charge sheet
density on the surface of the cylinder. Hint: Use separation of variables.
\item An electric dipole of moment ${\bf p} = p\,{\bf e}_z$ lies at the center of an  uncharged dielectric sphere of radius $a$
and uniform dielectric constant $\epsilon$. Find ${\bf D}$ and ${\bf E}$
throughout space. Find the bound charge sheet density on the surface of the
sphere. Hint: Use the separation of variables.
\item A parallel plate capacitor has plates of area $A$ and spacing $d$.
Half of the region between the plates is filled with a dielectric
of uniform dielectric constant $\epsilon_1$, and the other half is
filled with a dielectric of uniform dielectric constant $\epsilon_2$. 
If the interface between the two dielectric media is a plane parallel to
the two plates, lying half way between them, what is the capacitance of
the capacitor? If the interface is perpendicular to the two plates, and
bisects them, what is the capacitance of the capacitor?
\item Consider a parallel plate capacitor whose plates are of
area $A$ and spacing $d$. Find the force of attraction per unit area between
the plates when:
\begin{enumerate}
\item The region between the plates is empty and the capacitor is
connected to a battery of voltage $V$.
\item The capacitor is disconnected from the battery (but remains charged),
and then fully immersed in a dielectric liquid of uniform dielectric constant $\epsilon$.
\item The dielectric liquid is replaced by a slab of
solid dielectric of uniform dielectric constant $\epsilon$ which fills the
region between the plates, but does not touch the plates.
\item The uncharged capacitor is fully immersed in a dielectric liquid of uniform dielectric constant $\epsilon$, and then charged to a voltage $V$.
\item The region between the plates of the uncharged capacitor is
filled by a solid dielectric of uniform dielectric constant $\epsilon$, which does not touch the plates, and the capacitor is then charged to a voltage $V$.
\end{enumerate}
\item A parallel plate capacitor has the region between its electrodes
completely filled with a dielectric slab of uniform dielectric constant $\epsilon$. The plates
are of length $l$, width $w$, and spacing $d$. The capacitor is
charged until its plates are at a potential difference $V$, and then disconnected. The dielectric slab is then partially withdrawn in the
$l$ dimension until only a length $x$ remains between the plates.
What is the potential difference between the plates? What is the
force acting to pull the slab back towards its initial position? Neglect
end effects. Hint: Use an energy argument to calculate the force.
\item A parallel plate capacitor with electrodes of area $A$, spacing
$d$, which carry the fixed charges $\pm Q$, is dipped vertically into a large vat of dielectric liquid of uniform dielectric
constant $\epsilon$ and 
mass density $\rho_m$. What height $h$ does the liquid rise between
the plates (relative to the liquid level outside the plates)? Neglect end effects.
\item A solenoid consists of a wire wrapped uniformly around a long solid cylindrical ferromagnetic core
of radius $a$ and uniform permeability $\mu$. Suppose that there are $N$
turns of the wire per unit length. What is the magnetic field-strength inside
the core when a current $I$ flows through the wire? Suppose that the core
is replaced by a cylindrical annulus of the same material. What is
the magnetic field-strength in the cylindrical cavity inside the annulus when a current $I$ flows
through the wire?
\item A long straight wire carries a current $I$ and is surrounded by
a ferromagnetic co-axial cylindrical annulus of uniform permeability $\mu$, inner radius $a$, and
outer radius $b$. Find the magnetic field everywhere. Find the magnetization
current density both within the annulus and on the surfaces of the annulus.
\item An infinitely long  cylinder of radius $a$, which is coaxial with the $z$-axis, has a uniform magnetization ${\bf M} = M\,{\bf e}_z$. Find the induced magnetic
field both inside and outside the cylinder. Find the magnetization current
density on the surface of the cylinder.
\item A very large piece of magnetic material  of constant permeability  $\mu$ contains
a uniform magnetic  field ${\bf B}_0$.  Find the magnetic field inside
a needle-shaped cavity running parallel to ${\bf B}_0$. Find the field
inside a wafer-shaped cavity aligned perpendicular to ${\bf B}_0$. Neglect
end effects.
\item A spherical annulus of magnetic material of inner radius $a$,  outer
radius $b$, and uniform permeability $\mu$ is placed in the
uniform magnetic field ${\bf B} = B_0\,{\bf e}_z$. Find the magnetic
scalar potential everywhere. What is the magnetic field inside the shell?
Hint: Use the separation of variables.
\item A long ferromagnetic cylinder of radius $a$ and uniform permeability  
$\mu$ is placed in a uniform magnetic field ${\bf B}_0$ which
runs perpendicular to the axis of the cylinder. Find the magnetic
field both inside and outside the cylinder. Find the magnetization current
 on the surface of the cylinder. Hint: Use separation of variables.
\item An magnetic dipole of moment ${\bf m} = m\,{\bf e}_z$ lies at the center of a ferromagnetic sphere of radius $a$
and uniform permeability $\mu$. Find ${\bf H}$ and ${\bf B}$
throughout space. Find the magnetization current density on the surface of the
sphere. Hint: Use the separation of variables.
\item An magnet consists of a thin ring of magnetic material of radius
$a$ containing a narrow gap of width $d$, where $d\ll a$ ({\em i.e.}, the magnet has the
same shape as the core of the electromagnet shown in Figure~\ref{mc}).
The magnetic material possesses a
uniform permanent magnetization ${\bf M} = M\,{\bf e}_\varphi$, where
${\bf e}_\varphi$ is a unit vector which runs toroidally around the ring.
What is the strength of the magnetic field in the gap? If $A$ is the cross-sectional area of the ring, how much magnetostatic energy does the magnet possess? Neglect field leakage.
\item Suppose that half of the permanently magnetized material making up the
ring in the previous question is replaced by material of uniform permeability
$\mu$. What is the magnetic field-strength in the gap? How much magnetostatic energy does the magnet possess? Neglect field
leakage.
\end{enumerate}
\renewcommand{\theenumi}{arabic{enumi}}
}
