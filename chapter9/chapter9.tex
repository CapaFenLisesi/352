\chapter{Electromagnetic Radiation}\label{waves}
\section{Introduction}
In this chapter, we shall employ Maxwell's equations to investigate
the emission, scattering, propagation, absorption, reflection, and refraction of electromagnetic radiation.

\section{Hertzian Dipole}\label{s9.1}
Consider two small spherical conductors connected by a wire. Suppose that  electric
charge flows periodically back and forth between the spheres. Let $q(t)$ be the {\em instantaneous} charge on
one of the conductors. The system is assumed to have zero net charge, so that 
the charge on the other conductor is $-q(t)$. Finally, let
\begin{equation} 
q(t) = q_0\, \sin \,(\omega\, t).
\end{equation}
Now, we expect the oscillating current flowing in the wire connecting the two spheres to
generate {\em electromagnetic radiation}\/ (see Section~\ref{sret}). Let us consider the simple
case in which the length of the wire is {\em small}\/ compared to the wavelength of the emitted
radiation. If this is the case then the current $I$ flowing between the conductors has the
same phase along the whole length of the wire. It  follows that
\begin{equation}
I(t) = \frac{dq}{dt} = I_0\, \cos(\omega\, t),
\end{equation}
where $I_0 = \omega \,q_0$. This type of antenna is called a {\em Hertzian dipole}, after
the German physicist Heinrich Hertz.

The magnetic vector potential generated by a current distribution ${\bf j}({\bf r})$ is given by
the well-known formula (see Section~\ref{s4.12})
\begin{equation}
{\bf A}({\bf r}, t) = \frac{\mu_0}{4\pi} \int \frac{[{\bf j}]}{|{\bf r} - {\bf r}'|}\,
d^3{\bf r}',
\end{equation}
where
\begin{equation}
[f] \equiv f({\bf r}', t - |{\bf r} - {\bf r}'|/c).
\end{equation}
Suppose that the wire is aligned along the $z$-axis, and extends from $z=-l/2$ to $z=l/2$. 
For a wire of negligible thickness, we can replace ${\bf  j}({\bf r}', t- |{\bf r} - {\bf r}'|/c)
\,d^3{\bf r}'$ by $I({\bf r}', t - |{\bf r} - {\bf r}'|/c)\,dz' \,{\bf e}_z$. 
Thus, ${\bf A}({\bf r}, t) = A_z({\bf r}, t)\, {\bf e}_z$, and
\begin{equation}
A_z ({\bf r}, t) = \frac{\mu_0}{4\pi} \int_{-l/2}^{l/2} 
\frac{I(z', t - |{\bf r} - z'\, {\bf e}_z |/c)}{|{\bf r} - z'\, {\bf e}_z|}\,dz'.
\end{equation}

In the region $r\gg l$,
\begin{equation}
|{\bf r} - z' \,{\bf e}_z| \simeq r,
\end{equation}
and
\begin{equation}
t - |{\bf r} - z' \,{\bf e}_z|/c \simeq t - r/c.
\end{equation}
The maximum error involved in the latter approximation is $\Delta t \sim l/c$. This error (which is
a time) must be much less than a period of oscillation of the emitted radiation,
otherwise the phase of the radiation will be wrong. So we require that
\begin{equation}
\frac{l}{c} \ll \frac{2\pi}{\omega},
\end{equation}
which implies that $l\ll \lambda$,
 where $\lambda = 2\pi\, c/\omega$ is the wavelength of the emitted radiation.
However, 
we have already assumed that the length of the wire $l$ is much less than the wavelength of the
radiation, and so the above inequality is automatically
 satisfied. Thus, in the {\em far field}\/ region, $r\gg \lambda$, we can
write
\begin{equation}
A_z ({\bf r}, t) \simeq \frac{\mu_0}{4\pi} \int_{-l/2}^{l/2} \frac{I(z', t -r/c)}{r}\,dz'.
\end{equation}
This integral is easy to perform, since the current is uniform along the length of the wire. 
So, we get
\begin{equation}\label{e9.10}
A_z({\bf r}, t) \simeq \frac{\mu_0 \,l}{4\pi} \frac{I(t-r/c)}{r}.
\end{equation}

The scalar potential is most conveniently evaluated using the Lorenz gauge condition (see Section~\ref{s4.12})
\begin{equation}
\nabla\cdot {\bf A} = - \epsilon_0 \mu_0\, \frac{\partial\phi}{\partial t}.
\end{equation}
Now,
\begin{equation}
\nabla\cdot{\bf A} =\frac{\partial A_z}{\partial z} \simeq \frac{\mu_0 \,l}{4\pi}
\frac{\partial I(t-r/c)}{\partial t} \left(-\frac{z}{r^2 \,c}\right) +O\left(
\frac{1}{r^2}\right)
\end{equation}
to leading order in $r^{-1}$. Thus, we obtain
\begin{equation}\label{e9.13}
\phi({\bf r}, t) \simeq \frac{l}{4\pi\epsilon_0\, c} \frac{z}{r} \frac{I(t-r/c)}{r}.
\end{equation}

Given the vector and scalar potentials, Equations~(\ref{e9.10}) and (\ref{e9.13}),
 respectively, we can 
evaluate the associated electric and magnetic fields using (see Section~\ref{s4.12})
\begin{eqnarray}
{\bf E} &=& - \frac{\partial {\bf A}}{\partial t} - \nabla \phi,\\[0.5ex]
{\bf B} &=& \nabla\times {\bf A}.
\end{eqnarray}
Note that we are only interested in {\em radiation fields}, which fall off like $r^{-1}$
with increasing distance from the source. It is easily demonstrated that
\begin{equation}
{\bf E} \simeq - \frac{\omega\, l\, I_0}{4\pi\epsilon_0 \,c^2}\, \sin\theta \,
\frac{\sin[\omega\, (t-r/c)]}{r} \,{\bf e}_\theta,
\end{equation}
and 
\begin{equation}
{\bf B} \simeq -\frac{\omega \,l \,I_0 }{4\pi \epsilon_0 \,c^3} \,\sin \theta \,\frac{\sin
[\omega \,(t-r/c)]}{r}\, {\bf e}_\varphi.
\end{equation}
Here, ($r$, $\theta$, $\varphi$) are standard spherical polar coordinates aligned along
the $z$-axis. The above expressions for the far field ({\em i.e.}, $r\gg \lambda$)
electromagnetic fields generated by a localized oscillating current are also
easily derived from Equations~(\ref{e4.183}) and (\ref{e4.184}). Note that the fields are symmetric in
the azimuthal angle $\varphi$. Moreover, there is no radiation along the axis of the oscillating
dipole ({\em i.e.}, $\theta =0$), and the maximum emission is in the plane perpendicular
to this axis ({\em i.e.}, $\theta = \pi/2$)---see Figure~\ref{f91}. 

\begin{figure}
\epsfysize=3.in
\centerline{\epsffile{chapter9/fig9.1.eps}}
\caption{\em Equally-spaced contours of the normalized poloidal electric field $E_\theta/E_0$ generated by
a Hertzian dipole in the $x$-$z$ plane at a fixed instant in time. Here, $E_0=\omega\,l\,I_0/4\pi\epsilon_0\,c^2$. }\label{f91}
\end{figure}

The average power crossing a spherical surface $S$ (whose radius is much greater than
$\lambda$), centered on the dipole,  is
\begin{equation}
P_{\rm rad} = \oint_S \langle {\bf u}\rangle\cdot d{\bf S},
\end{equation}
where the average is over a single period of oscillation of the wave, and
the Poynting flux is given by (see Section~\ref{s8.2})
\begin{equation}
{\bf u} = \frac{ {\bf E} \times{\bf B}}{\mu_0} = \frac{\omega^2 \,l^2 \,I_0^{\,2}}
{16\pi^2 \epsilon_0\, c^3}\, \sin^2[\omega\,(t-r/c)] \,\frac{\sin^2\theta}{r^2}\,
{\bf e}_r.
\end{equation}
It follows that
\begin{equation}\label{e9.20}
\langle {\bf u}\rangle = \frac{\omega^2\, l^2\, I_0^{\,2}}{32\pi^2\epsilon_0\, c^3}\, \frac{\sin^2\theta}
{r^2}\, {\bf e}_r.
\end{equation}
Note that the energy flux is {\em radially outwards}\/ from the source. The total power 
flux across $S$ is given by
\begin{equation}
P_{\rm rad} = \frac{\omega^2 \,l^2\, I_0^{\,2}}{32\pi^2 \epsilon_0\, c^3}
\int_0^{2\pi} d\phi \int_0^\pi \frac{\sin^2\theta}{r^2} \,\, r^2\sin\theta\,\,
d\theta,
\end{equation}
yielding
\begin{equation}
P_{\rm rad} = \frac{\omega^2\, l^2\, I_0^{\,2}}{12\pi \epsilon_0\, c^3}.
\end{equation}
This total flux is independent of the radius of $S$, as is to be expected if
energy is conserved. 

Recall that for a resistor of resistance $R$ the average ohmic heating power is
\begin{equation}
P_{\rm heat} = \langle I^2 \,R\rangle = \frac{1}{2} I_0^{\,2}\, R,
\end{equation}
assuming that $I= I_0 \,\cos(\omega\, t)$. It is convenient to define the {\em radiation
resistance}\/ of a Hertzian dipole antenna:
\begin{equation}
R_{\rm rad} = \frac{P_{\rm rad}}{I_0^{\,2}/2},
\end{equation}
so that
\begin{equation}
R_{\rm rad} = \frac{2\pi}{3\epsilon_0 \,c} \left(\frac{l}{\lambda}\right)^2,
\end{equation}
where $\lambda = 2\pi\,c/\omega$ is the wavelength of the radiation.
In fact,
\begin{equation}\label{e9.25}
R_{\rm rad} = 789 \left(\frac{l}{\lambda}\right)^2\,{\rm ohms}.
\end{equation}
Now, in the theory 
of electrical circuits, an antenna is conventionally represented as
a resistor whose resistance is equal
to the characteristic radiation resistance of the antenna plus its real
resistance. The power loss $I_0^{\,2}\,R_{\rm rad}/2$
associated with the radiation resistance
 is due to the emission of electromagnetic radiation, whereas the power loss
$ I_0^{\,2}\,R/2$ associated with
 the real resistance is
due to ohmic heating of the antenna.

Note that the formula (\ref{e9.25})  is only valid for $l\ll \lambda$. This suggests
that $R_{\rm rad} \ll R$ for most Hertzian 
dipole antennas: {\em i.e.}, the radiated power is
swamped by  the ohmic losses. Thus, antennas whose lengths are much less than
that of the emitted radiation tend to be {\em extremely inefficient}.
In fact,  it is necessary
to have $l\sim \lambda$ in order to obtain an efficient antenna. The simplest
practical antenna is the {\em half-wave antenna}, for which $l = \lambda/2$. This
can be analyzed as a series of Hertzian dipole antennas
 stacked on top of one another, each
slightly out of phase with its neighbours. The characteristic radiation resistance
of a half-wave antenna is
\begin{equation}
R_{\rm rad} = \frac{ 2.44}{4\pi\epsilon_0 \,c} = 73 \,\,{\rm ohms}.
\end{equation}

Antennas can also be used to receive electromagnetic radiation. The incoming wave
induces a voltage in the antenna, which can be detected in an electrical
 circuit
connected to the antenna. In fact, this process is equivalent to the emission
of electromagnetic waves by the antenna viewed in reverse. It
is easily demonstrated that antennas most readily detect electromagnetic radiation
incident from those directions in which they preferentially emit radiation. 
Thus, a Hertzian dipole antenna is unable to detect radiation incident along
its axis, and most efficiently detects radiation incident in the plane perpendicular
to this axis. In the theory of electrical circuits, a receiving antenna is represented
as a voltage source in series with a resistor. The voltage source, $V_0\,\cos(\omega \,t)$, represents
the voltage  induced in the antenna by the incoming wave. The resistor,
$R_{\rm rad}$, represents the power re-radiated by the antenna (here, 
the real resistance
of the antenna is neglected). Let us represent the detector circuit as a single
load resistor, $R_{\rm load}$, connected in series with the antenna. So, 
what value of $R_{\rm load}$  ensures that the maximum power is extracted from the
wave and transmitted to the detector circuit? 

According to Ohm's law,
\begin{equation}
V=V_0 \,\cos(\omega\, t) = I_0\, \cos(\omega\, t)\, (R_{\rm rad} + R_{\rm load}),
\end{equation}
where $I= I_0\,\cos(\omega\, t)$ is the current induced in the circuit. 
The power input to the circuit is
\begin{equation}
P_{\rm in} = \langle V\,I\rangle = \frac{V_0^{\,2}}{2\,(R_{\rm rad} + R_{\rm load})}.
\end{equation}
The power transferred to the load is
\begin{equation}
P_{\rm load} = \langle I^2 \,R_{\rm load}\rangle = \frac{R_{\rm load} \,V_0^{\,2}}
{2\,(R_{\rm rad} + R_{\rm load})^2}.
\end{equation}
Finally, the power re-radiated by the antenna is 
\begin{equation}
P_{\rm rad} = \langle I^2 \,R_{\rm rad}\rangle = \frac{R_{\rm rad} \,V_0^{\,2}}
{2\,(R_{\rm rad} + R_{\rm load})^2}.
\end{equation}
Note that $P_{\rm in} = P_{\rm load} + P_{\rm rad}$.
The maximum power transfer to the load occurs when
\begin{equation}
\frac{\partial P_{\rm load}}{\partial R_{\rm load}} = \frac{V_0^{\,2}}{2}\left[
\frac{R_{\rm rad} - R_{\rm load}}{(R_{\rm rad} + R_{\rm load})^3}\right] = 0.
\end{equation}
Thus, the maximum transfer rate corresponds to
\begin{equation}
R_{\rm load} = R_{\rm rad}.
\end{equation}
In other words, the resistance of the load circuit must match the radiation
resistance of the antenna. 
For this optimum case,
\begin{equation}
P_{\rm load} = P_{\rm rad} = \frac{V_0^{\,2}}{8 \,R_{\rm rad}} = \frac{P_{\rm in}}{2}.
\end{equation}
So, in the optimum case {\em  half}
 of the power absorbed by the antenna is immediately
re-radiated. Clearly,  an antenna which
is receiving electromagnetic radiation is also  emitting it. 
This (allegedly) is how the BBC catch people who do not pay their television license fee in
the United Kingdom. They have vans which can detect the radiation emitted by
a TV aerial whilst it is in use (they can even tell which channel you are watching!).

For a Hertzian dipole antenna interacting with an incoming wave whose electric
field has an amplitude $E_0$, we expect
\begin{equation}
V_0 = E_0\, l.
\end{equation}
Here, we have used the fact that the wavelength of the radiation is much longer
than the length of the antenna. We have also assumed that the antenna is
properly aligned ({\em i.e.}, the radiation is incident perpendicular to the axis of the
antenna). The Poynting flux of the incoming wave is [see Equation~(\ref{e8.35})]
\begin{equation}
\langle u_{\rm in}\rangle = \frac{\epsilon_0\, c\, E_0^{\,2}}{2},
\end{equation} 
whereas the power transferred to a properly matched detector circuit is
\begin{equation}
P_{\rm load} = \frac{E_0^{\,2} \,l^2}{8\, R_{\rm rad}}.
\end{equation}
Consider an idealized antenna in which all 
incoming radiation incident on some area $A_{\rm eff}$ is  absorbed, and then
magically transferred to the detector circuit, with no re-radiation. 
Suppose  that the power absorbed from the idealized antenna
 matches that absorbed from
 the
real antenna. This implies that
\begin{equation}
P_{\rm load} = \langle u_{\rm in}\rangle\, A_{\rm eff}.
\end{equation}
 The quantity $A_{\rm eff}$ is called the {\em effective area}\/ of the antenna: it is
the area of the idealized antenna which absorbs as much net power from the incoming 
wave as the actual antenna. 
Thus,
\begin{equation}
P_{\rm load} =\frac{ E_0^{\,2}\, l^2}{8 \,R_{\rm rad} } = \frac{\epsilon_0\, c\, E_0^{\,2}}{2}
\,A_{\rm eff},
\end{equation}
giving
\begin{equation}
A_{\rm eff} = \frac{l^2}{4\epsilon_0 \,c\, R_{\rm rad}} = \frac{3}{8\pi}\,\lambda^2.
\end{equation}
So, it is clear that the effective area of a Hertzian dipole antenna is of
order the {\em wavelength squared}\/ of the incoming radiation. 

For a properly aligned half-wave antenna,
\begin{equation}
A_{\rm eff} = 0.13\,\lambda^2.
\end{equation}
Thus, the antenna, which is essentially one-dimensional with length $\lambda/2$,
acts as if it is two-dimensional, with width $0.26\,\lambda$, as far as its
absorption of incoming electromagnetic radiation is concerned.  

\section{Electric Dipole Radiation}
In the previous section, we examined the radiation emitted by a short electric dipole of oscillating dipole moment
\begin{equation}
{\bf p}(t) = p_0\,\sin(\omega\,t)\,{\bf e}_z,
\end{equation}
where $p_0= q_0\,l=I_0\,l/\omega$. We found that, in the far field region, the mean
electromagnetic energy flux takes the form [see Equation~(\ref{e9.20})]
\begin{equation}
\langle {\bf u}\rangle = \frac{\omega^4\,p_0^{\,2}}{32\pi^2\epsilon_0\,c^3}\frac{\sin^2\theta}{r^2}\,{\bf e}_r,
\end{equation}
assuming that the dipole is centered on the origin of our spherical polar coordinate system.
The mean power radiated into the element of solid angle $d{\mit\Omega} = \sin\theta\,d\theta\,d\varphi$, centered on the angular coordinates ($\theta$, $\varphi$),
is 
\begin{equation}
dP = \langle {\bf u}(r, \theta, \varphi)\rangle\!\cdot\!{{\bf e}_r}\,\,r^2\,d{\mit\Omega}.
\end{equation}
Hence, the differential power radiated into this element of solid angle is
simply
\begin{equation}\label{e9.45}
\frac{dP}{d{\mit\Omega}} = \frac{\omega^4\,p_0^{\,2}}{32\pi^2\epsilon_0\,c^3}\,\sin^2\theta.
\end{equation}
This formula completely specifies the radiation pattern of an oscillating electric dipole (provided that the dipole is much shorter in length than the
wavelength of the emitted radiation). Of course, the power radiated into
a given element of solid angle is independent of $r$, otherwise energy would not be conserved. Finally, the total radiated power  is the integral of
$dP/d{\mit\Omega}$ over all solid angles.

\section{Thomson Scattering}
Consider a plane electromagnetic wave of angular frequency $\omega$
interacting with a {\em free}\/ electron of mass $m_e$ and charge $-e$. Suppose that
the wave is polarized such that its associated electric field is parallel
to the $z$-axis: {\em i.e.},
\begin{equation}\label{e9.46}
{\bf E} = E_0\,\sin(\omega\,t)\,{\bf e}_z.
\end{equation}
Recall, from Section~\ref{sem}, that as long as the electron remains non-relativistic, the force exerted on it by the electromagnetic wave comes
predominantly from the associated electric field. Hence, the electron's equation of
motion can be written
\begin{equation}
m_e\,\frac{d^2 z}{dt^2} = -e\,E_0\,\sin(\omega\,t),
\end{equation}
which can be solved to give
\begin{equation}
z = \frac{e\,E_0}{m_e\,\omega^2}\,\sin(\omega\,t).
\end{equation}
So, in response to the wave, the electron oscillates backward and forward in the direction of the
wave electric field. It follows that the electron can be thought of as a sort of
oscillating electric dipole, with dipole moment
\begin{equation}
{\bf p} = - e\,z\,\,{\bf e}_z = -p_0\,\sin(\omega\,t)\,{\bf e}_z,
\end{equation}
where $p_0 = e^2\,E_0/(m_e\, \omega^2)$.  (For the moment, let us not worry  about the
positively charged component of the dipole.) Now, we know that an
oscillating electric dipole emits electromagnetic radiation. Hence, it
follows that a free electron placed in the path of a plane
electromagnetic wave will radiate. To be more exact, the electron
{\em scatters}\, electromagnetic radiation from the wave, since the
radiation emitted by the electron is not necessarily in the same direction
as the wave, and any energy radiated by the electron is ultimately extracted
from the wave. This type of scattering is called {\em Thomson scattering}.

It follows from Equation~(\ref{e9.45}) that the differential power scattered
from a plane electromagnetic wave
by a free electron into solid angle $d\,{\mit\Omega}$ takes the form
\begin{equation}\label{e9.50}
\frac{dP}{d{\mit\Omega}} = \frac{e^4\,E_0^{\,2}}{32\pi^2\epsilon_0\,c^3\,m_e^{\,2}}\,\sin^2\theta.
\end{equation}
Now, the mean energy flux of the incident  electromagnetic wave is written
\begin{equation}\label{e9.51}
|\langle {\bf u}\rangle| = \frac{c\,\epsilon_0\,E_0^{\,2}}{2}.
\end{equation}
It is helpful to introduce  a quantity called the {\em differential scattering cross-section}. This is defined
\begin{equation}
\frac{d\sigma}{d{\mit\Omega}} = \frac{dP/d{\mit\Omega}}{|\langle{\bf u}\rangle|},
\end{equation}
and has units of area over solid angle. Somewhat figuratively, we can think
of the electron as offering a target of area $d\sigma/d{\mit\Omega}$ to the
incident wave. Any wave energy which falls on this target is scattered into
the solid angle $d{\mit\Omega}$. Likewise, we can also define the {\em total scattering cross-section},
\begin{equation}
\sigma = \oint \frac{d\sigma}{d{\mit\Omega}}\,d{\mit\Omega},
\end{equation}
which has units of area. Again, the electron effectively offers a target of
area $\sigma$ to the incident wave. Any wave energy which falls on this
target is scattered in some direction or other.
It follows from Equations~(\ref{e9.50}) and (\ref{e9.51}) that the differential
scattering cross-section for Thomson scattering is
\begin{equation}
\frac{d\sigma}{d{\mit\Omega}} = r_e^{\,2}\,\sin^2\theta,
\end{equation}
where the characteristic length
\begin{equation}\label{e9.55}
r_e = \frac{e^2}{4\pi\epsilon_0\,m_e\,c^2} = 2.82\times 10^{-15}\,{\rm m}
\end{equation}
is called the {\em classical electron radius}. An electron effectively acts
like it has a spatial extent $r_e$ as far as its iteration with  electromagnetic
radiation is concerned. As is easily demonstrated, the
total Thomson scattering cross-section is
\begin{equation}\label{e9.56}
\sigma_T = \frac{8\,\pi}{3}\,r_e^{\,2} = 6.65\times 10^{-29}\,{\rm m^2}.
\end{equation}
Note that both the differential and the total Thomson scattering cross-sections
are completely {\em independent}\/ of the frequency (or wavelength) of the incident
radiation.

A scattering cross-section of $10^{-28}\,{\rm m^2}$ does not sound like much. Nevertheless, Thomson scattering is one of the most important types
of scattering in the Universe. Consider the Sun. It turns out that the
mean mass density of the Sun is similar to that of water: {\em i.e.}, 
about $10^3\,{\rm kg\,m}^{-3}$. Hence, assuming that
the Sun is predominantly made up   of ionized Hydrogen, the mean number density of electrons in the Sun (which, of course, is the same as
the number density of protons) is approximately $n_e \sim 10^3/m_p\sim 10^{30}\,{\rm m}^{-3}$, where
$m_p\sim 10^{-27}\,{\rm kg}$ is the mass of a proton. 
Let us consider how far, on average, a photon in the Sun travels before
being scattered by a free electron. If we think of an individual photon
as sweeping out a cylinder of cross-sectional area $\sigma_T$, then the photon will travel an average length $l$, such that a cylinder of area $\sigma_T$ and length $l$ contains about one free electron, before
being scattered. Hence, $\sigma_T\,l \,n_e\sim 1$, or
\begin{equation}
l \sim \frac{1}{n_e\,\sigma_T}\sim 1\,{\rm cm}.
\end{equation}
Given that the radius of the Sun is approximately $10^9\,{\rm m}$, it is
clear that solar photons are very strongly scattered by free electrons.
In fact, it can easily be demonstrated that it takes a photon emitted in the solar
core many thousands of years to fight its way to the surface
because of Thomson scattering.

After the ``Big Bang'', when the Universe was very hot, it consisted
predominately of ionized Hydrogen (and dark matter), and was consequently
{\em opaque}\/ to electromagnetic radiation, due to Thomson scattering. However,
as the Universe expanded, it also cooled, and eventually became sufficiently
cold (when the mean temperature was about $1000^\circ\,{\rm C}$) for any free protons and electrons to combine to form molecular
Hydrogen. It turns out that molecular Hydrogen does not scatter radiation
anything like as effectively as free electrons (see the next section). Hence, as soon as the
Universe became filled with molecular Hydrogen, it effectively became 
{\em transparent}\/ to radiation. Indeed, the so-called  {\em cosmic microwave background}\/  is the remnant of radiation which was last scattered
when the Universe was filled with ionized Hydrogen ({\em i.e.}, when it was about 
$1000^\circ\,{\rm C}$). Astronomers can gain a great deal of information
about the conditions in the early Universe by studying this radiation.

Incidentally, it is clear from Equations~(\ref{e9.55}) and (\ref{e9.56}) that the scattering cross-section of a free particle of charge $q$ and mass $m$ is proportional to
$q^4/m^2$. It follows that the scattering of electromagnetic radiation  by free electrons is generally very much
stronger than the scattering by free protons  (assuming that the
number densities of both species are similar).

\section{Rayleigh Scattering}\label{s9.4}
Let us now consider the scattering of electromagnetic radiation by neutral
atoms. For instance, consider a Hydrogen atom. 
The atom consists of a light electron and a massive proton. As we have seen,
the electron scatters radiation much more strongly than the proton, so
let us concentrate on the  response of the electron to an incident electromagnetic
wave. Suppose that the wave electric field is again polarized in the
$z$-direction, and is given by Equation~(\ref{e9.46}). We can approximate
the electron's equation of motion as
\begin{equation}\label{e9.58}
m_e\,\frac{d^2 z}{dt^2} = - m_e\,\omega_0^{\,2}\,z - e\,E_0\,\sin(\omega\,t).
\end{equation}
Here, the second term on the right-hand side represents the perturbing
force due to the electromagnetic wave, whereas the first term represents
the (linearized) force of electrostatic attraction between the electron and the proton.
Here, we are very crudely modeling our Hydrogen atom as a
{\em simple harmonic oscillator}\/ of natural frequency $\omega_0$. 
We can think of $\omega_0$ as the typical frequency of electromagnetic
radiation emitted by the atom after it is transiently disturbed. In other words,
in our model,
$\omega_0$  should match the frequency of one of the spectral lines of
Hydrogen. More generally, we can extend the above model to deal
with just about any type of atom, provided that we set $\omega_0$
to the frequency of a spectral line.


We can easily solve Equation~(\ref{e9.58}) to give
\begin{equation}
z = \frac{e\,E_0}{m_e\,(\omega^2-\omega_0^{\,2})} \sin(\omega\,t).
\end{equation}
Hence, the dipole moment of the electron takes the form
${\bf p} = - p_0\,\sin(\omega\,t)\,{\bf e}_z$, where
\begin{equation}
p_0 = \frac{e^2\,E_0}{m_e\,(\omega^2-\omega_0^{\,2})}.
\end{equation}
It follows, by analogy with the analysis in the previous section, that the differential
and total scattering cross-sections of our model atom take the form
\begin{equation}\label{e9.61}
\frac{d\sigma}{d{\mit\Omega}} = \frac{\omega^4}{(\omega^2-\omega_0^{\,2})^2}\,r_e^{\,2}\,\sin^2\theta,
\end{equation}
and
\begin{equation}\label{e9.62}
\sigma =  \frac{\omega^4}{(\omega^2-\omega_0^{\,2})^2}\,\sigma_T,
\end{equation}
respectively.

In the limit in which the frequency of the incident radiation is {\em much greater}\/
than the natural frequency of the atom, Equations~(\ref{e9.61}) and (\ref{e9.62})
reduce to the previously obtained expressions for scattering by
a free electron. In other words, an electron in an atom acts very much
like a free electron as far as high frequency radiation is concerned. 
In the opposite limit, in which the frequency of the incident radiation
is {\em much less}\/ than the natural frequency of the atom, Equations~(\ref{e9.61}) and (\ref{e9.62}) yield
\begin{equation}
\frac{d\sigma}{d{\mit\Omega}} = \left(\frac{\omega}{\omega_0}\right)^4\,r_e^{\,2}\,\sin^2\theta,
\end{equation}
and
\begin{equation}\label{e9.64}
\sigma = \left(\frac{\omega}{\omega_0}\right)^4\,\sigma_T,
\end{equation}
respectively. This type of scattering is called {\em Rayleigh scattering}. There
are two features of Rayleigh scattering which are worth noting. First of
all, it is much weaker than Thomson scattering (since $\omega \ll
\omega_0$). Secondly, unlike Thomson scattering, it is highly
{\em frequency dependent}. Indeed, it is clear, from the above formulae,
that high frequency (short wavelength) radiation is scattered far more
effectively than low frequency (long wavelength) radiation.

The most common example of Rayleigh scattering is the scattering
of visible radiation from the Sun by neutral atoms (mostly Nitrogen and Oxygen) in the upper
atmosphere. The frequency of visible radiation is much less than the
typical emission frequencies of a Nitrogen or Oxygen atom (which lie in the
ultra-violet band), so it is certainly the case that $\omega\ll \omega_0$. 
When the Sun is low in the sky, radiation from it has to traverse a
comparatively long path through the atmosphere before reaching us. Under
these circumstances, the scattering of direct solar light by neutral atoms in the atmosphere
becomes noticeable (it is not noticeable when the Sun is high is the sky,
and radiation from it consequently only has to traverse a relatively
short path through the atmosphere before reaching us). 
According to Equation~(\ref{e9.64}), blue light is scattered slightly
more strongly than red light (since blue light has a slightly higher frequency
than red light). Hence,  when the Sun is low in the sky, it appears less bright,
due to atmospheric scattering. However, it also appears {\em redder}\/ than normal, because
more blue light than red light is scattered out of the solar light-rays, leaving an excess of red light.
Likewise, when we look up at the daytime sky, it does not appear black (like the
sky on the Moon) because of light from solar radiation which grazes
the atmosphere being scattered downward towards the surface of the
Earth. Again, since blue light is scattered more effectively than red light,
there is an excess of blue light scattered downward, and so the daytime sky appears
{\em blue}.

Light from the Sun is unpolarized. However, when it is scattered it becomes
polarized, because light is scattered preferentially in some directions
rather than others. Consider a light-ray from the Sun which grazes
the Earth's atmosphere. The light-ray contains light which is
polarized such that the electric field is {\em vertical}\/ to the ground, and
light which is polarized such that the electric field is {\em horizontal}\/ to the ground
(and perpendicular to the path of the light-ray), in equal amounts. However,
due to the $\sin^2\theta$ factor in the dipole emission formula
(\ref{e9.45}) (where, in this case, $\theta$ is the angle between the direction of the wave
electric field and the direction of scattering), very little light is scattered downward from the vertically
polarized light compared to the horizontally polarized light. 
Moreover, the light scattered from the horizontally polarization is
such that its electric field is preferentially perpendicular, rather than parallel, to the
direction of propagation of the solar light-ray ({\em i.e.}, the direction to the Sun).
Consequently, the blue light from the daytime sky is preferentially polarized in a direction
{\em perpendicular}\/ to the direction to the Sun.

\section{Propagation in a Dielectric Medium}
Consider the propagation of an electromagnetic wave through a
uniform dielectric medium of dielectric constant $\epsilon$. 
According to Equations~(\ref{e6.8}) and (\ref{e6.10}), the dipole moment
per unit volume induced in the medium by the wave electric field ${\bf E}$
is
\begin{equation}\label{e9.65}
{\bf P} = \epsilon_0\,(\epsilon -1)\,{\bf E}.
\end{equation}
There are no free charges or free currents in the medium. There is also
no bound charge density (since the medium is uniform), and no magnetization
current density (since the medium is non-magnetic). However, there
is a {\em polarization current}\/ due to the time-variation of the induced
dipole moment per unit volume. According to Equation~(\ref{e6.44x}), this
current is given by
\begin{equation}\label{e9.66}
{\bf j}_p = \frac{\partial{\bf P}}{\partial t}. 
\end{equation}
Hence, Maxwell's equations take the form
\begin{eqnarray}
\nabla\!\cdot\!{\bf E} &=& 0,\\[0.5ex]
 \nabla\!\cdot\!{\bf B} &=&0,\\[0.5ex]
\nabla\times {\bf E} &=&-\frac{\partial {\bf B}}{\partial t},\\[0.5ex]
\nabla\times{\bf B} &=& \mu_0\,{\bf j}_p+ \epsilon_0\,\mu_0\,\frac{\partial {\bf E}}{\partial t}.
\end{eqnarray}
According to Equations~(\ref{e9.65}) and (\ref{e9.66}), the last of the above
equations can be rewritten
\begin{equation}
\nabla\times{\bf B} =\epsilon_0\,\mu_0\,(\epsilon-1)\,\frac{\partial {\bf E}}{\partial t}+ \epsilon_0\,\mu_0\,\frac{\partial {\bf E}}{\partial t}=
\frac{\epsilon}{c^2}\,\frac{\partial {\bf E}}{\partial t},
\end{equation}
since $c= (\epsilon_0\,\mu_0)^{-1/2}$.
Thus, Maxwell's equations for the propagation of electromagnetic waves
through a dielectric medium are the same as Maxwell's equations
for the propagation of waves through a vacuum (see Section~\ref{sem}), except that
$c\rightarrow c/n$,
where
\begin{equation}
n = \sqrt{\epsilon}
\end{equation}
is called the {\em refractive index} of the medium in question. Hence,
we conclude that electromagnetic waves propagate through a dielectric
medium {\em slower}\/ than through a vacuum by a factor $n$ (assuming,
of course, that $n>1$). This conclusion (which was reached
long before Maxwell's equations were invented) is the basis of all
geometric optics involving refraction.

\section{Dielectric Constant of a Gaseous Medium}\label{s9.6}
In Section~\ref{s9.4}, we discussed a rather crude model of an atom interacting with an electromagnetic wave. According to
this model, the dipole moment ${\bf p}$ of the atom induced by the wave electric field ${\bf E}$ is given by
\begin{equation}
{\bf p} = \frac{e^2}{m_e\,(\omega_0^{\,2}-\omega^2)}\,{\bf E},
\end{equation}
where $\omega_0$ is the natural frequency of the atom ({\em i.e.},
the frequency of one of the atom's spectral lines), and $\omega$
 the frequency of the incident radiation. Suppose that there
are $n$ atoms per unit volume. It follows that the induced dipole
moment per unit volume of the assemblage of atoms takes the
form
\begin{equation}
{\bf P} = \frac{n\,e^2}{m_e\,(\omega_0^{\,2}-\omega^2)}\,{\bf E}.
\end{equation}
Finally, a comparison with Equation~(\ref{e9.65}) yields the following expression
for the dielectric constant of the collection of atoms,
\begin{equation}\label{e9.75}
\epsilon = 1 + \frac{n\,e^2}{\epsilon_0\,m_e\,(\omega_0^{\,2}-\omega^2)}.
\end{equation}
The above formula works fairly well for dilute gases, although it is,
of course, necessary to sum over all species and all important
spectral lines. 

Note that, in general, the dielectric ``constant'' of a gaseous medium
(as far as electromagnetic radiation is concerned)
is a function of the wave frequency, $\omega$. Since the effective wave
propagation speed through the medium is $c/\sqrt{\epsilon}$, it follows that
waves of different frequencies travel through
a gaesous medium at {\em different}\/ speeds. This phenomenon
is called {\em dispersion}, since it can be shown to cause short
wave-pulses to spread out as they propagate through the medium.
At low frequencies ($\omega\ll \omega_0$), however, our
expression for $\epsilon$ becomes frequency independent, and so there
is no dispersion of low-frequency waves by a gaseous medium.

\section{Dispersion Relation of a Plasma}\label{s9.7}
A plasma is very similar to a gaseous medium, expect that the electrons
are {\em free}: {\em i.e.}, there is no restoring force due to nearby atomic
nuclii. Hence,  we can obtain an expression for the dielectric constant
of a plasma from Equation~(\ref{e9.75}) by setting $\omega_0$ to zero,
and $n$ to the number density of electrons, $n_e$.
We obtain
\begin{equation}\label{e9.76}
\epsilon = 1 - \frac{\omega_p^{\,2}}{\omega^2},
\end{equation}
where the characteristic frequency
\begin{equation}\label{e9.77}
\omega_p = \sqrt{\frac{n_e\,e^2}{\epsilon_0\,m_e}}
\end{equation}
is called the {\em plasma frequency}.
We can immediately see that formula (\ref{e9.76}) is problematic.
For frequencies above the plasma frequency, the dielectric constant
of a plasma is less than unity. Hence, the refractive
index $n=\sqrt{\epsilon}$ is also less than unity. This would seem to imply that
high-frequency electromagnetic waves can propagate through a plasma with
a velocity $c/n$ which is {\em greater}\/ than the velocity of
light in a vacuum. This appears to violate one of the principles of Relativity. 
On the other hand, for frequencies below the plasma frequency, the dielectric constant
is {\em negative}, which would seem to imply that the refractive
index $n=\sqrt{\epsilon}$ is {\em imaginary}. How should we interpret this?

Consider an infinite plane-wave of frequency $\omega$, which is greater than
the plasma frequency, propagating through a plasma. Suppose that
the wave electric field takes the form
\begin{equation}\label{e9.78}
{\bf E} = E_0\,{\rm e}^{\,{\rm i}\,(k\,x-\omega\,t)}\,{\bf e}_z,
\end{equation}
where it is understood that the physical electric field is the {\em real
part}\/ of the above expression.  A peak or trough of the above wave
travels at the so-called {\em phase-velocity}, which is given by
\begin{equation}\label{e9.79}
v_p = \frac{\omega}{k}. 
\end{equation}
Now, we have also seen that the phase-velocity of electromagnetic waves
in a dielectric medium is $v_p = c/n=c/\sqrt{\epsilon}$, so
\begin{equation}\label{e9.80a}
\omega^2 = \frac{k^2\,c^2}{\epsilon}.
\end{equation}
It follows from Equation~(\ref{e9.76})  that
\begin{equation}\label{e9.80}
\omega^2 = k^2\,c^2+\omega_p^{\,2}
\end{equation}
in a plasma. The above type of expression, which effectively determines the wave
frequency, $\omega$, as a function of the wave-number, $k$, for the
medium in question, is called a {\em dispersion relation} (since,
amongst other things, it determines how fast wave-pulses disperse in
the medium). According to the above dispersion relation, the phase-velocity
of high-frequency waves propagating through a plasma is given by
\begin{equation}\label{e9.81}
v_p = \frac{c}{\sqrt{1-\omega_p^{\,2}/\omega^{2}}},
\end{equation}
which is indeed greater than $c$. However, the Theory of Relativity does not
forbid this. What the Theory of Relativity says is that {\em information}\/
cannot travel at a velocity greater than $c$. However, the peaks and
troughs of an infinite
plane-wave, such as (\ref{e9.78}), {\em do not}\/ carry any information.

We  now need to consider how we could transmit information through
a plasma (or any other dielectric medium) by means of electromagnetic
waves. The easiest way would be to send a series of short discrete wave-pulses
through the plasma, so that we could transmit  information in a sort of
Morse code. We can build up a wave-pulse from a suitable
superposition of infinite plane-waves of different frequencies and
wavelengths: {\em e.g.},
\begin{equation}\label{e9.82}
 E_z(x,t) = \int F(k)\,{\rm e}^{\,{\rm i}\,\phi(k)}\,dk,
\end{equation}
where $\phi(k) = k\,x-\omega(k)\,t$, and 
$\omega(k)$ is determined from the dispersion relation 
(\ref{e9.80}). Now, it turns out that a relatively short wave-pulse can only be built up
from a superposition of plane-waves with a relatively wide range of different $k$
values. Hence, for a short wave-pulse, the integrand in the above formula
consists of the product of a fairly slowly varying function, $F(k)$, and
a rapidly oscillating function, $\exp[{\rm i}\,\phi(k)]$. The latter function
is rapidly oscillating because the phase $\phi(k)$ varies very rapidly with $k$,
relative to $F(k)$.
We expect the net result of integrating the product of a slowly varying function and rapidly oscillating function to be  small, since the oscillations will generally average to zero. It follows that the integral
(\ref{e9.82}) is dominated by those regions of $k$-space for which
$\phi(k)$ varies {\em least rapidly}\/ with $k$. Hence, the peak of the wave-pulse
most likely corresponds to a maximum or minimum of $\phi(k)$:
{\em i.e.},
\begin{equation}
\frac{d\phi}{dk} = x - \frac{d\omega}{dk}\,t = 0.
\end{equation}
Thus, we infer that the velocity of the wave-pulse (which corresponds to the velocity of the peak) is given by
\begin{equation}
v_g = \frac{d\omega}{dk}.
\end{equation}
This velocity is called the {\em group-velocity}, and is different to the
phase-velocity in dispersive media: {\em i.e.}, media for which $\omega$ is not {\em directly
proportional}\/ to $k$. (Of course, in a vacuum, $\omega = k\,c$, so the
phase and group velocities are both equal to $c$.) The upshot of the above discussion is that information ({\em i.e.}, an individual wave-pulse)
travels through a dispersive media at the group-velocity, rather than the
phase-velocity. Hence, Relativity demands that the group-velocity, rather
than the phase-velocity,  must
always be less than $c$. 

What is the group-velocity for high-frequency waves propagating through a plasma?
Well, differentiation of the dispersion relation (\ref{e9.80}) yields
\begin{equation}
\frac{\omega}{k}\,\frac{d\omega}{dk} = v_p\,v_g= c^2.
\end{equation}
Hence, it follows from Equation~(\ref{e9.81}) that
\begin{equation}\label{e9.86}
v_g = c\,\sqrt{1-\frac{\omega_p^{\,2}}{\omega^2}},
\end{equation}
which is less than $c$. We thus conclude that the dispersion relation (\ref{e9.80})
is indeed consistent with Relativity.

Let us now consider the propagation of low-frequency electromagnetic waves through a plasma. We can see, from Equations~(\ref{e9.81}) and (\ref{e9.86}),
that when the wave frequency, $\omega$, falls below the plasma frequency,
$\omega_p$, both the phase and group velocities become imaginary.
This indicates that the wave {\em attenuates}\/ as it propagates. 
Consider, for instance, a plane-wave of frequency $\omega < \omega_p$.
According to the dispersion relation (\ref{e9.80}), the associated
wave-number is given by
\begin{equation}\label{e9.87}
k = \left.{\rm i}\,\sqrt{\omega_p^{\,2}-\omega^2}\right/c = {\rm i}\,|k|.
\end{equation}
Hence, the wave electric field takes the form
\begin{equation}
E_z = E_0\,{\rm e}^{\,{\rm i}\,({\rm  i}\,|k|\,x-\omega\,t)}
= E_0\,{\rm e}^{-|k|\,x}\,{\rm e}^{-{\rm i}\,\omega\,t}.
\end{equation}
So, it can be seen that for $\omega<\omega_p$ electromagnetic
waves in a plasma take the form of decaying {\em standing waves}, rather than
traveling waves. We conclude that an electromagnetic wave, of frequency
less than the plasma frequency, which is incident on a plasma
will not propagate through the plasma. Instead, it will be {\em totally
reflected}. 

We can be sure that the incident wave is reflected by the plasma, rather than absorbed,
 by considering the energy flux of the wave in the plasma.
It is easily demonstrated that the energy flux of an electromagnetic
wave can be written
\begin{equation}
{\bf u} = \frac{{\bf E}\times{\bf B}}{\mu_0} = \frac{E^2}{\mu_0\,\omega}\,{\bf k}.
\end{equation}
For a wave with a real frequency and a complex ${\bf k}$-vector, the
above formula generalizes to
\begin{equation}\label{e9power}
{\bf u} = \frac{|E|^2}{\mu_0\,\omega}\,{\rm Re}({\bf k}).
\end{equation}
However, according to Equation~(\ref{e9.87}), the ${\bf k}$-vector
for a low-frequency electromagnetic wave in a plasma is {\em purely
imaginary}. It follows that the associated energy flux is {\em zero}. 
Hence, any low-frequency wave which is incident on the plasma
must be totally reflected, since if there were any absorption of the wave
energy
 then there would be a net energy flux into the plasma.

The outermost layer of the Earth's atmosphere consists of a partially ionized
zone known as the {\em ionosphere}. The plasma frequency in the
ionosphere is about 1\,MHz, which lies at the upper end of the
medium-wave band of radio frequencies. It follows that low-frequency radio signals ({\em i.e.}, all signals in the long-wave band,
and most in the medium-wave band) are {\em reflected}\/ off the ionosphere. 
For this reason, such signals can be detected over the horizon. Indeed,
long-wave radio signals reflect multiple times off the ionosphere with very little loss (they also
reflect multiple times off the Earth, which is enough of a conductor
to act as a mirror for radio waves), and can consequently be detected all over the world.
On the other hand, high-frequency radio signals ({\em i.e.}, all
signals in the FM band) pass straight through the ionosphere. For this
reason, such signals cannot be detected over the horizon, which accounts
for the relatively local coverage of FM radio stations. Note, from 
Equation~(\ref{e9.77}), that the plasma frequency is proportional to the
square root of the number density of free electrons. Now, the
level of ionization in the ionosphere is maintained by ultra-violet light from the Sun (which effectively knocks electrons out of neutral atoms). Of course, there
is no such light at night, and the number density of free electrons in the
ionosphere consequently drops as electrons and ions gradually recombine.
It follows that the plasma frequency in the ionosphere also drops at night, giving
rise to a marked deterioration in the reception of distant medium-wave radio stations.

\section{Faraday Rotation}
Consider a high-frequency electromagnetic wave propagating, along the $z$-axis,  through
a plasma with a longitudinal equilibrium magnetic field,
${\bf B} = B_0\,{\bf e}_z$.
The equation of motion of an individual electron making up the plasma takes the
form
\begin{equation}
m_e\,\frac{d{\bf v}}{dt} = -e\,({\bf E} + B_0\,{\bf v}\times{\bf e}_z),
\end{equation}
where the first term on the right-hand side is due to the wave electric field,
and the second  to the equilibrium magnetic field.
(As usual, we can neglect the wave magnetic field, provided that the
electron motion remains non-relativistic.)
Of course, ${\bf v} = d{\bf r}/dt$, where ${\bf r}$ is the electron
displacement from its equilibrium position. Suppose that all perturbed quantities vary with time
like $\exp(-{\rm i}\,\omega\,t)$, where $\omega$ is the wave frequency.
It follows that
\begin{eqnarray}\label{e9.93}
m_e\,\omega^2\,x = e\,(E_x - {\rm i}\,\omega\,B_0\,y),\\[0.5ex]
m_e\,\omega^2\,y = e\,(E_y+{\rm i}\,\omega\,B_0\,x).\label{e9.94}
\end{eqnarray}

It is helpful to define
\begin{eqnarray}
s_\pm &=& x\pm {\rm i}\,y,\\[0.5ex]
E_\pm &=& E_x\pm {\rm i}\,E_y.
\end{eqnarray}
Using these new variables, Equations~(\ref{e9.93}) and (\ref{e9.94})
can be rewritten
\begin{equation}
m_e\,\omega^2\,s_{\pm} = e\,(E_{\pm} \mp \omega\,B_0\,s_\pm),
\end{equation}
which can be solved to give
\begin{equation}\label{e9.97}
s_\pm = \frac{e\,E_\pm}{m_e\,\omega\,(\omega\pm{\mit \Omega})},
\end{equation}
where ${\mit \Omega} = e\,B_0/m_e$ is the so-called {\em cyclotron frequency} ({\em i.e.},
the characteristic gyration frequency of free electrons in the
equilibrium magnetic field---see Section~\ref{slorentz}).

In terms of $s_\pm$, the electron displacement can be written
\begin{equation}
{\bf r} = s_+\,{\rm e}^{\,{\rm i}\,(k_+\,z-\omega\,t)}\,{\bf e}_+
+  s_-\,{\rm e}^{\,{\rm i}\,(k_-\,z-\omega\,t)}\,{\bf e}_-,
\end{equation}
where
\begin{equation}
{\bf e}_\pm = \frac{1}{2}\left({\bf e}_x \mp {\rm i}\,{\bf e}_y\right).
\end{equation}
Likewise, in terms of $E_\pm$, the wave electric field takes the form
\begin{equation}\label{e9.101}
{\bf E} = E_+\,{\rm e}^{\,{\rm i}\,(k_+\,z-\omega\,t)}\,{\bf e}_+
+  E_-\,{\rm e}^{\,{\rm i}\,(k_-\,z-\omega\,t)}\,{\bf e}_-.
\end{equation}
Obviously, the actual displacement and electric field are the {\em real
parts} of the above expressions. It follows from Equation~(\ref{e9.101})
that $E_+$ corresponds to a constant amplitude electric field which rotates {\em clockwise}
in the $x$-$y$ plane (looking down the $z$-axis) as the wave propagates in the $+z$-direction, whereas
$E_-$ corresponds to a constant amplitude electric field which rotates {\em anti-clockwise}.
The former type of wave is termed {\em left-hand circularly polarized},
whereas the latter is termed {\em right-hand circularly polarized}. 
Note also that $s_+$ and $s_-$ correspond to {\em circular} electron motion
in opposite senses.
With these insights, we conclude that Equation~(\ref{e9.97}) indicates that
individual electrons in the plasma have a slightly different response to left- and right-hand circularly polarized waves in the presence of a
longitudinal magnetic field.

Following the analysis of Section~\ref{s9.6}, we can deduce from Equation~(\ref{e9.97}) that the dielectric constant  of the plasma for left- and 
right-hand circularly
polarized waves is
\begin{equation}
\epsilon_{\pm} = 1 - \frac{\omega_p^{\,2}}{\omega\,(\omega\pm{\mit \Omega})},
\end{equation}
respectively.   Hence, according to Equation~(\ref{e9.80a}), the  dispersion relation
for left- and right-hand circularly polarized waves becomes
\begin{equation}
k_{\pm}^{\,2}\,c^2 = \omega^2\left[1- \frac{\omega_p^{\,2}}{\omega\,(\omega\pm{\mit \Omega})}\right],
\end{equation}
respectively.
In the limit $\omega \gg \omega_p, {\mit \Omega}$, we obtain
\begin{equation}\label{e9.104}
k_{\pm}\simeq k\pm {\mit\Delta} k,
\end{equation}
where $k= \omega\,[1-(1/2)\,\omega_p^{\,2}/\omega^2]/c$ and
${\mit\Delta} k = (1/2)\,(\omega_p^{\,2}/\omega^2)\,{\mit \Omega}/c$. In other words, in a magnetized plasma,
left- and right-hand circularly polarized waves of the same frequency have
slightly different wave-numbers

Let us now consider the propagation of a {\em linearly polarized}\/ electromagnetic
wave through the plasma. Such a wave can be constructed via a superposition
of left- and  right-hand circularly polarized waves of {\em equal}\/
amplitudes. So, the wave electric field can be written
\begin{equation}\label{e9.105}
{\bf E} = E_0 \left[{\rm e}^{\,{\rm i}\,(k_+\,z-\omega\,t)}\,{\bf e}_+
+ {\rm e}^{\,{\rm i}\,(k_-\,z-\omega\,t)}\,{\bf e}_- \right].
\end{equation}
It can easily be seen that at $z=0$ the wave electric field is aligned
along the $x$-axis. If left- and right-hand circularly polarized waves
of the same frequency have the same wave-number ({\em i.e.}, if
$k_+=k_-$) then the wave electric field will continue to be aligned
along the $x$-axis as the wave propagates  in the $+z$-direction:
{\em i.e.}, we will obtain a standard linearly polarized wave. However, we have just
demonstrated
that, in the presence of a longitudinal magnetic field, the wave-numbers $k_+$ and $k_-$
are slightly different. What effect does this have on the polarization of the
wave?

Taking the real part of Equation~(\ref{e9.105}), and making use of Equation
 (\ref{e9.104}), and some standard trigonometrical identities,
we obtain
\begin{equation}
{\bf E} =E_0 \left[\cos(k\,z-\omega\,t)\,\cos({\mit\Delta} k\,z),\,
\cos(k\,z-\omega\,t)\,\sin({\mit\Delta} k\,z),\, 0\right].
\end{equation}
The polarization angle of the wave (which is a convenient measure of
its plane of polarization) is given by
\begin{equation}
\varphi = \tan^{-1}(E_y/E_x) = {\mit\Delta}k\,z.
\end{equation}
Thus, we conclude that in the presence of a longitudinal magnetic field
the polarization angle  {\em rotates} as the wave propagates through the plasma. This effect is
known as {\em Faraday rotation}. It is clear, from the above expression,
that the rate of advance of the polarization angle with distance
travelled by the wave is given by
\begin{equation}
\frac{d\varphi}{dz} = {\mit\Delta}k = \frac{\omega_p^{\,2}\,{\mit\Omega}}{2\,\omega^2\,c} = \frac{e^3}{2\epsilon_0\,m_e^{\,2}\,c}\frac{n_e\,B_0}{\omega^2}.
\end{equation}
Hence, a linearly polarized electromagnetic wave which propagates through a
plasma with a (slowly) varying electron number density, $n_e(z)$, 
and longitudinal magnetic field, $B_0(z)$, has its plane of
polarization rotated through a total angle
\begin{equation}\label{e9.108}
{\mit\Delta}\varphi = \varphi - \varphi_0=\frac{e^3}{2\epsilon_0\,m_e^{\,2}\,c}\,\frac{1}{\omega^2}
\int n_e(z)\,B_0(z)\,dz.
\end{equation}
Note the very strong inverse variation of ${\mit\Delta}\varphi$ with $\omega$.

{\em Pulsars} are rapidly rotating neutron stars which emit regular blips of
highly polarized radio waves. Hundreds of such objects have been found
in our galaxy since the first was discovered in 1967. By measuring
the variation of the angle of polarization, $\varphi$, of radio emission from a pulsar with frequency, $\omega$, astronomers can effectively determine
the line integral of $n_e\,B_0$ along the straight-line joining the
pulsar to the Earth using formula (\ref{e9.108}). Here, $n_e$ is the number
density of free electrons in the interstellar medium, whereas $B_0$
is the parallel component of the galactic magnetic field. Obviously, in order
to achieve this, astronomers must make the reasonable assumption that the
radiation was emitted by the pulsar with a common angle of polarization, $\varphi_0$,
over a wide range of different frequencies. By fitting Equation~(\ref{e9.108})
to the data, and then extrapolating to large $\omega$, it is then possible to
determine  $\varphi_0$, and, hence, the amount, ${\mit\Delta}\varphi(\omega)$, 
through which the polarization angle of the radiation has rotated, at a given frequency, during its passage to Earth.

\section{Propagation in a Conductor}\label{s9.9}
Consider the propagation of an electromagnetic wave through a conducting
medium which obeys Ohm's law:
\begin{equation}
{\bf j} = \sigma\,{\bf E}.
\end{equation}
Here, $\sigma$ is the {\em conductivity} of the medium in question. 
Maxwell's equations for the wave take the form:
\begin{eqnarray}
\nabla\cdot{\bf E} &=& 0,\\[0.5ex]
\nabla\cdot{\bf B} &=& 0,\\[0.5ex]
\nabla\times{\bf E} &=& - \frac{\partial {\bf B}}{\partial t},\\[0.5ex]
\nabla\times{\bf B} &=&\mu_0\,{\bf j} + \epsilon\,\epsilon_0\mu_0\,\frac{\partial {\bf E}}{\partial t},
\end{eqnarray}
where $\epsilon$ is the dielectric constant of the medium.
It follows, from the above equations, that
\begin{equation}
\nabla\times\nabla\times{\bf E} = -\nabla^2{\bf E} = -\frac{\partial
\nabla\times{\bf B}}{\partial t} =-\frac{\partial}{\partial t}\!\left[
\mu_0\,\sigma\,{\bf E} + \epsilon\,\epsilon_0\mu_0\,\frac{\partial {\bf E}}{\partial t}\right].
\end{equation}
Looking for a wave-like solution of the form
\begin{equation}\label{e9.116}
{\bf E} = {\bf E}_0\,{\rm e}^{\,{\rm i}\,(k\,z-\omega\,t)},
\end{equation}
we obtain the dispersion relation
\begin{equation}\label{e9.117}
k^2 = \mu_0\,\omega\,(\epsilon\,\epsilon_0\,\omega + {\rm i}\,\sigma).
\end{equation}

Consider a ``poor'' conductor for which $\sigma\ll \epsilon\,\epsilon_0\,\omega$. In this limit, the dispersion relation 
(\ref{e9.117}) yields
\begin{equation}
k\simeq n\,\frac{\omega}{c} + {\rm i}\,\frac{\sigma}{2}\sqrt{\frac{\mu_0}
{\epsilon\,\epsilon_0}},
\end{equation}
where $n=\sqrt{\epsilon}$ is the refractive index.
Substitution into Equation~(\ref{e9.116}) gives
\begin{equation}\label{e9.118}
{\bf E} = {\bf E}_0\,{\rm e}^{-z/d}\,{\rm e}^{\,{\rm i}\,(k_r\,z-\omega\,t)},
\end{equation}
where
\begin{equation}\label{e9.119}
d = \frac{2}{\sigma}\sqrt{\frac{\epsilon\,\epsilon_0}{\mu_0}},
\end{equation}
and $k_r = n\,\omega/c$. Thus, we conclude that the amplitude
of an electromagnetic wave propagating through a conductor
{\em decays exponentially} on some length-scale, $d$, which
is termed the {\em skin-depth}.  Note, from Equation~(\ref{e9.119}),
that the skin-depth for a poor conductor is {\em independent} of the frequency
of the wave. Note, also, that $k_r\,d\gg 1$ for a poor conductor,
indicating that the wave penetrates many wavelengths into the
conductor before decaying away. 

Consider a ``good'' conductor for which $\sigma\gg \epsilon\,\epsilon_0\,\omega$. In this limit, the dispersion relation 
(\ref{e9.117}) yields
\begin{equation}\label{ekgood}
k\simeq \sqrt{{\rm i}\,\mu_0\,\sigma\,\omega}.
\end{equation}
Substitution into Equation~(\ref{e9.116}) again gives Equation~(\ref{e9.118}),
with
\begin{equation}
d = \frac{1}{k_r} = \sqrt{\frac{2}{\mu_0\,\sigma\,\omega}}.
\end{equation}
It can be seen that the skin-depth for a good conductor {\em decreases}
with increasing wave frequency. The fact that $k_r\,d= 1$ indicates
that the wave only penetrates a few wavelengths into the conductor
before decaying away.

Now the power per unit volume dissipated via ohmic heating in
a conducting medium is
\begin{equation}
P = {\bf j}\cdot{\bf E} = \sigma\,E^2.
\end{equation}
Consider an electromagnetic wave of the form (\ref{e9.118}). The mean power dissipated 
per unit area in the region $z>0$ is written
\begin{equation}
\langle P\rangle = \frac{1}{2}\int_{0}^\infty
\sigma\,E_0^{\,2}\,e^{-2\,z/d}\,dz = \frac{d\,\sigma}{4}\,E_0^{\,2} =
\sqrt{\frac{\sigma}{8\,\mu_0\,\omega}}\,E_0^{\,2},
\end{equation}
for a good conductor.
Now, according to Equation~(\ref{e9power}), the mean electromagnetic
power flux into the region $z>0$ takes the form
\begin{equation}
\langle u\rangle = \left\langle \frac{{\bf E}\times{\bf B}\cdot{\bf e}_z}{\mu_0}\right\rangle_{z=0} = \frac{1}{2}\frac{E_0^{\,2}\,k_r}{\mu_0\,\omega} = \sqrt{\frac{\sigma}{8\,\mu_0\,\omega}}\,E_0^{\,2}.
\end{equation}
It is clear, from a comparison of the previous two equations, that
all of the wave energy which flows into the region $z>0$ is dissipated via ohmic
heating. We thus conclude that the attenuation of an electromagnetic
wave propagating through a conductor is a direct consequence of  ohmic power losses.

Consider a typical metallic conductor such as Copper, whose electrical
conductivity at room temperature  is about $6\times
10^{7}\,(\Omega\,{\rm m})^{-1}$. Copper, therefore, acts as a good
conductor for all electromagnetic waves of frequency below about
$10^{18}\,{\rm Hz}$. The skin-depth in Copper for such waves is thus
\begin{equation}
d = \sqrt{\frac{2}{\mu_0\,\sigma\,\omega}} \simeq \frac{6}{\sqrt{f({\rm Hz})}}\,{\rm cm}.
\end{equation}
It follows that the skin-depth is about $6\,{\rm cm}$ at 1\,Hz, but only about
2\,mm at 1\,kHz. This gives rise to the so-called {\em skin-effect}\/ in copper wires, by which an oscillating electromagnetic
signal of increasing frequency, transmitted along such a wire,  is confined
to an increasingly narrow layer (whose thickness is of order the skin-depth)
on the surface of the wire.

The conductivity of sea-water is only about $\sigma\simeq 5\,(\Omega\,{\rm m})^{-1}$. However, this is still sufficiently high for sea-water to act as
a good conductor for all radio frequency electromagnetic waves ({\em i.e.}, $f=\omega/2\pi < 10^9$\,Hz). The skin-depth at 1\,MHz ($\lambda\sim 2$\,km)
is about $0.2$\,m, whereas that at 1\,kHz ($\lambda\sim 2000$\,km)
is still only about 7\,m. This obviously poses quite severe restrictions for
radio communication with submerged submarines. Either the submarines
have to come quite close to the surface to communicate (which is dangerous), or the communication must be performed with extremely low frequency (ELF) waves ({\em i.e.}, $f< 100$\,Hz). Unfortunately, such waves have very large wavelengths ($\lambda > 20,000\,{\rm km}$), which means
that they can only be efficiently generated by gigantic
antennas. 

\section{Dispersion Relation of a Collisional Plasma}
We have now investigated electromagnetic wave propagation
through two different media possessing free electrons: {\em i.e.}, plasmas (see Section~\ref{s9.7}), and
 ohmic conductors (see Section~\ref{s9.9}). In the first case, we
obtained the dispersion relation (\ref{e9.80}), whereas in the second
we obtained the quite different dispersion relation (\ref{e9.117}).
This leads us, quite naturally, to ask what the essential distinction is between
the response of  free electrons in a plasma to an electromagnetic
wave, and that of  free electrons in an ohmic conductor.
It turns out that the main distinction is the relative strength of electron-ion
{\em collisions}. 

In the presence of electron-ion collisions, we can model the equation
of motion of an individual electron in a plasma or a conductor as
\begin{equation}\label{e9.127}
m_e\,\frac{d{\bf v}}{dt} + m_e\,\nu\,{\bf v} = -e\,{\bf E},
\end{equation}
where ${\bf E}$ is the wave electric field. The collision term ({\em i.e.}, the second term on the left-hand side) takes the form of 
a drag force proportional to $-{\bf v}$. In the absence of the wave electric
field, this force damps out any electron motion on the typical time-scale
$\nu^{-1}$. Since, in reality, an electron loses virtually all of its directed momentum during a collision with a much more massive ion, we can regard
$\nu$ as the effective electron-ion collision frequency.

Assuming the usual $\exp(-{\rm i}\,\omega\,t)$ time-dependence of
perturbed quantities, we can solve Equation~(\ref{e9.127})
to give
\begin{equation}\label{e9.128}
{\bf v} = -{\rm i}\,\omega\,{\bf r} = - \frac{{\rm i}\,\omega\,e\,{\bf E}}{m_e\,\omega\,(\omega+{\rm i}\,\nu)}.
\end{equation}
Hence, the perturbed current density can be written
\begin{equation}
{\bf j} = - e\,n_e\,{\bf v} = \frac{{\rm i}\,n_e\,e^2 \,{\bf E}}{m_e\,(\omega+{\rm i}\,\nu)},
\end{equation}
where $n_e$ is the number density of free electrons.
It follows that the effective conductivity of the medium takes the form
\begin{equation}\label{e9.130}
\sigma = \frac{{\bf j}}{{\bf E}} = \frac{{\rm i}\,n_e\,e^2}{m_e\,(\omega+{\rm i}\,\nu)}.
\end{equation}

Now, the mean rate of ohmic heating per unit volume in the medium is
written
\begin{equation}
\langle P\rangle = \frac{1}{2} \,{\rm Re}(\sigma)\,E_0^{\,2},
\end{equation}
where $E_0$ is the amplitude of the wave electric field. Note that
only the {\em real part}\/ of $\sigma$ contributes to ohmic heating, because the
perturbed current must be {\em in phase} with the wave electric field
in order for there to be a net heating effect. An imaginary $\sigma$ gives a
perturbed current which is in phase quadrature with the wave electric field.
In this case, there is zero net transfer of power between the wave and the plasma over a wave period. We can see from Equation~(\ref{e9.130})
that in the limit in which the wave frequency is much larger than the
collision frequency ({\em i.e.}, $\omega\gg \nu$), the effective
conductivity of the medium becomes purely imaginary:
\begin{equation}
\sigma \simeq  \frac{{\rm i}\,n_e\,e^2}{m_e\,\omega}.
\end{equation}
In this limit, there is no loss of wave energy due to ohmic heating, and
the medium acts like a conventional plasma. In the opposite limit, in which
the wave frequency is much less than the collision frequency ({\em i.e.}, $\omega\ll\nu$), the effective conductivity becomes purely real:
\begin{equation}\label{e9.133}
\sigma \simeq  \frac{n_e\,e^2}{m_e\,\nu}.
\end{equation}
In this limit, ohmic heating losses are significant, and the medium acts like
a conventional ohmic conductor.

Repeating the analysis of Section~\ref{s9.6}, we can derive the following
dispersion relation from Equation~(\ref{e9.128}):
\begin{equation}\label{e9.134}
k^2\,c^2 = \omega^2 - \frac{\omega_p^{\,2}\,\omega}{\omega + {\rm i}\,\nu}.
\end{equation}
It can be seen that, in the limit $\omega\gg \nu$, the above dispersion relation
reduces to the dispersion relation (\ref{e9.80}) for a conventional
({\em i.e.}, collisionless) plasma. In the opposite limit, we obtain
\begin{equation}
k^2 = \frac{\omega^2}{c^2} + {\rm i}\,\frac{\omega_p^{\,2}\,\omega}{\nu\,c^2} = \mu_0\,\omega\,(\epsilon_0\,\omega + {\rm i}\,\sigma).
\end{equation}
where use has been made of Eq~(\ref{e9.133}). Of course, the
above dispersion relation is identical to the dispersion relation (\ref{e9.117})
(with $\epsilon=1$) which we  previously derived for an ohmic
conductor.

Our main conclusion from this section is that the dispersion relation
(\ref{e9.134})
can be used to describe electromagnetic wave propagation through both a collisional plasma and an ohmic
conductor. We can also deduce that in the low frequency limit, $\omega\ll\nu$, a collisional
plasma acts very much like an ohmic conductor, whereas in the
high frequency limit, $\omega\gg \nu$, an ohmic conductor acts very much
like a collisionless plasma.

\section{Normal Reflection at a Dielectric Boundary}\label{snorm}
An electromagnetic wave of real (positive) frequency $\omega$ can be written
\begin{eqnarray}
{\bf E}({\bf r}, t) &=& {\bf E}_0\,{\rm e}^{\,{\rm i}\,({\bf k}\cdot{\bf r}-\omega\,t)},\\[0.5ex]
{\bf B}({\bf r},t) &=& {\bf B}_0\,{\rm e}^{\,{\rm i}\,({\bf k}\cdot{\bf r}-\omega\,t)}.
\end{eqnarray}
The wave-vector, ${\bf k}$, indicates the direction of propagation of the
wave, and also its phase-velocity, $v$, via
\begin{equation}
v = \frac{\omega}{k}.
\end{equation}
Since the wave is transverse in nature, we must have
${\bf E}_0\cdot{\bf k} = {\bf B}_0\cdot{\bf k} = 0$. Finally, the familiar
 Maxwell
equation
\begin{equation}
\nabla\times {\bf E} = -\frac{\partial {\bf B}}{\partial t}
\end{equation}
leads us to the following relation between the constant vectors ${\bf E}_0$ and ${\bf B}_0$:
\begin{equation}\label{e9.140}
{\bf B}_0 = \frac{\hat{\bf k}\times {\bf E}_0}{v}.
\end{equation}
Here, $\hat{\bf k} = {\bf k}/k$ is a unit vector pointing in the
direction of wave propagation.

Suppose that the plane $z=0$ forms the boundary between two different dielectric
media. Let medium 1, of refractive index $n_1$, occupy the region $z<0$,
whilst medium 2, of refractive index $n_2$, occupies the region $z>0$.
Let us investigate what happens when an electromagnetic wave is incident
on this boundary from medium 1.

\begin{figure}
\epsfysize=2.5in
\centerline{\epsffile{chapter9/fig9.2.eps}}
\caption{\em Reflection at a dielectric boundary for the case of normal incidence.}\label{f48}
\end{figure}

Consider, first of all, the simple case of incidence {\em normal}\/
to the boundary---see Figure~\ref{f48}. In this case, $\hat{\bf k} = +{\bf e}_z$ for the
incident and transmitted waves, and $\hat{\bf k} = -{\bf e}_z$ 
for the reflected wave. Without loss of generality, we can assume that
the incident wave is polarized in the $x$-direction.
Hence, using Equation~(\ref{e9.140}), the incident
wave can be written
\begin{eqnarray}
{\bf E}(z,t) &=& E_i\,{\rm e}^{\,{\rm i}\,(k_1\,z-\omega\,t)}
\,{\bf e}_x,\\[0.5ex]
{\bf B}(z,t)&=&\frac{E_i}{v_1}\,{\rm e}^{\,{\rm i}\,(k_1\,z-\omega\,t)}\,{\bf e}_y,
\end{eqnarray}
where $v_1=c/n_1$ is the phase-velocity in medium 1, and $k_1=\omega/v_1$. Likewise, the reflected wave takes the form
\begin{eqnarray}
{\bf E}(z,t) &=& E_r\,{\rm e}^{\,{\rm i}\,(-k_1\,z-\omega\,t)}\,{\bf e}_x,\\[0.5ex]
{\bf B}(z,t) &=&-\frac{E_r}{v_1}\,{\rm e}^{\,{\rm i}\,(-k_1\,z-\omega\,t)}\,{\bf e}_y.
\end{eqnarray}
Finally, the transmitted wave can be written
\begin{eqnarray}
{\bf E}(z,t) &=& E_t\,{\rm e}^{\,{\rm i}\,(k_2\,z-\omega\,t)}\,{\bf e}_x,\\[0.5ex]
{\bf B}(z,t) &=&\frac{E_t}{v_2}\,{\rm e}^{\,{\rm i}\,(k_2\,z-\omega\,t)}\,{\bf e}_y,
\end{eqnarray}
where $v_2=c/n_2$ is the phase-velocity in medium 2, and $k_2=\omega/v_2$.

For the case of normal incidence, the electric and magnetic
components of all three waves are {\em parallel} to the boundary between
the two dielectric media. Hence, the appropriate boundary conditions
to apply at $z=0$ are
\begin{eqnarray}\label{e9.147}
E_{\parallel\,1} &=& E_{\parallel\,2},\\[0.5ex]
B_{\parallel\,1}&=& B_{\parallel\,2}.\label{e9.148}
\end{eqnarray}
The latter condition derives from the general boundary condition
$H_{\parallel\,1} = H_{\parallel\,2}$, and the fact that ${\bf B}=\mu_0\,
{\bf H}$ in both media (which are assumed to be non-magnetic).

Application of the boundary condition $(\ref{e9.147})$ yields
\begin{equation}\label{e9.149}
E_i + E_r = E_t.
\end{equation}
Likewise, application of the boundary condition (\ref{e9.148}) gives
\begin{equation}
\frac{E_i - E_r}{v_1} = \frac{E_t}{v_2},
\end{equation}
or
\begin{equation}\label{e9.151}
E_i - E_r = \frac{v_1}{v_2}\,E_t = \frac{n_2}{n_1}\,E_t,
\end{equation}
since $v_1/v_2=n_2/n_1$.
Equations (\ref{e9.149}) and (\ref{e9.151}) can be solved to give
\begin{eqnarray}\label{e9.152}
E_r &=& \left(\frac{n_1-n_2}{n_1+n_2}\right) E_i,\\[0.5ex]
E_t &=& \left(\frac{2\, n_1}{n_1+n_2}\right) E_t.\label{e9.153}
\end{eqnarray}
Thus, we have determined the amplitudes of the reflected and transmitted
waves in terms of the amplitude of the incident wave.

It can be seen, first of all, that if $n_1=n_2$ then $E_r=0$ and $E_t=E_i$.
In other words, if the two media have the same indices of refraction then
there is no reflection  at the boundary between them, and the transmitted
wave is consequently equal in amplitude to the incident wave. On the other
hand, if $n_1\neq n_2$ then there is some reflection at the boundary. Indeed,
the amplitude of the reflected wave is roughly proportional to the difference between $n_1$ and $n_2$. This has  important practical consequences.
We can only see a clean pane of glass in a window because some of the light incident
on an air/glass boundary is reflected, due to the different refractive indicies
of air and glass. As is well-known, it is a lot more difficult to see glass when it is submerged in water. This is because the refractive indices of glass and water are quite similar, and so there is very little reflection of light
incident on a water/glass boundary.

According to Equation~(\ref{e9.152}), $E_r/E_i<0$ when $n_2> n_1$. 
The negative sign indicates a $180^\circ$ phase-shift of the reflected wave, with
respect to the incident wave. We conclude that there is a $180^\circ$ phase-shift of the reflected wave, relative to the incident wave, on reflection from a boundary with a
medium of {\em greater} refractive index. Conversely, there is no 
 phase-shift
on reflection from a boundary with a medium of {\em lesser} refractive index.

The mean electromagnetic energy flux, or {\em intensity}, in the $z$-direction is simply
\begin{equation}
I =\frac{\langle {\bf E}\times{\bf B}\cdot{\bf e}_z\rangle}{\mu_0}= \frac{E_0\,B_0}{2\,\mu_0} = \frac{E_0^{\,2}}{2\,\mu_0\,v}.
\end{equation}
The {\em coefficient of reflection}, $R$, is defined as the ratio
of the intensities of the reflected and incident waves:
\begin{equation}\label{e9.155}
R = \frac{I_r}{I_i} = \left(\frac{E_r}{E_i}\right)^2.
\end{equation}
Likewise, the {\em coefficient of transmission}, $T$, is the ratio of
the intensities of the transmitted and incident waves:
\begin{equation}\label{e9.156}
T = \frac{I_t}{I_i} =\frac{v_1}{v_2}\left(\frac{E_t}{E_i}\right)^2=\frac{n_2}{n_1}\left(\frac{E_t}{E_i}\right)^2.
\end{equation}
Equations~(\ref{e9.152}), (\ref{e9.153}), (\ref{e9.155}), and (\ref{e9.156})
yield
\begin{eqnarray}
R &=& \left(\frac{n_1-n_2}{n_1+n_2}\right)^2,\\[0.5ex]
T&=&\frac{n_2}{n_1}\left(\frac{2\,n_1}{n_1+n_2}\right)^2.
\end{eqnarray}
Note that $R+T=1$. In other words, any wave energy which is not reflected
at the boundary is transmitted, and {\em vice versa}.


\section{Oblique Reflection at a Dielectric Boundary}
Let us now consider the case of incidence {\em oblique} to the boundary---see Figure~\ref{f49}.
Suppose that the incident wave subtends an angle $\theta_i$ with the
normal to the boundary, whereas the reflected and transmitted
waves subtend angles $\theta_r$ and $\theta_t$, respectively.

\begin{figure}
\epsfysize=2.5in
\centerline{\epsffile{chapter9/fig9.3.eps}}
\caption{\em Reflection at a dielectric boundary for the case of oblique incidence.}\label{f49}
\end{figure}


The incident wave can be written
\begin{eqnarray}
{\bf E}({\bf r},t) &=& {\bf E}_i\,{\rm e}^{\,{\rm i}\,({\bf k}_i\cdot{\bf r}-\omega\,t)},\\[0.5ex]
{\bf B}({\bf r},t) &=& {\bf B}_i\,{\rm e}^{\,{\rm i}\,({\bf k}_i\cdot{\bf r}-\omega\,t)},
\end{eqnarray}
with analogous expressions for the reflected and transmitted waves.
Since, in the case of oblique incidence, the electric and magnetic
components of the wave fields are no longer necessarily parallel to the
boundary, the boundary conditions (\ref{e9.147}) and (\ref{e9.148}) at $z=0$ must
be supplemented by the additional boundary conditions
\begin{eqnarray}\label{e9.161}
\epsilon_1\,E_{\perp\,1} &=& \epsilon_2\,E_{\perp\,2},\\[0.5ex]
B_{\perp\,1} &=& B_{\perp\,2}.\label{e9.162}
\end{eqnarray}
Equation (\ref{e9.161}) derives from the general
boundary condition $D_{\perp\,1} = D_{\perp\,2}$.

It follows from Equations~(\ref{e9.148}) and (\ref{e9.162}) that both components
of the magnetic field are continuous at the boundary. Hence, we can write
\begin{equation}
 {\bf B}_i\,{\rm e}^{\,{\rm i}\,({\bf k}_i\cdot{\bf r}-\omega\,t)}
 + {\bf B}_r\,{\rm e}^{\,{\rm i}\,({\bf k}_r\cdot{\bf r}-\omega\,t)}= {\bf B}_t\,{\rm e}^{\,{\rm i}\,({\bf k}_t\cdot{\bf r}-\omega\,t)}
\end{equation}
at $z=0$. Given that ${\bf B}_i$, ${\bf B}_r$, and ${\bf B}_t$ are
constant vectors, the only way in which the above equation can be satisfied for all
values of $x$ and $y$ is if
\begin{equation}
{\bf k}_i\cdot{\bf r} = {\bf k}_r\cdot{\bf r} = {\bf k}_t\cdot{\bf r} 
\end{equation}
throughout the $z=0$ plane. This, in turn, implies that
\begin{equation}\label{e9.165}
k_{i\,x} =k_{r\,x}= k_{t\,x}
\end{equation}
and
\begin{equation}
k_{i\,y} = k_{r\,y} = k_{t\,y}.
\end{equation}
It immediately follows that if $k_{i\,y}=0$ then $k_{r\,y} =k_{t\,y}=0$.
In other words, if the incident wave lies in the $x$-$z$ plane then the
reflected and transmitted waves also lie in the $x$-$z$ plane. Another
way of putting this is that the incident, reflected, and transmitted
waves all lie in the {\em same} plane, know as the {\em plane
of incidence}. This, of course, is one of the laws of geometric optics.
From now on, we shall assume that the plane of incidence is the $x$-$z$
plane.

Now, $k_i=k_r = \omega/v_1$ and $k_t=\omega/v_2$. Moreover,
\begin{equation}
\sin\theta_i = \frac{k_{x\,i}}{k_i},
\end{equation}
with similar expressions for $\theta_r$ and $\theta_t$. Hence, according
to Equation~(\ref{e9.165}), 
\begin{equation}
\sin\theta_r = \sin\theta_i,
\end{equation}
which implies that $\theta_r=\theta_i$. Moreover,
\begin{equation}\label{e9.169}
\frac{\sin\theta_t}{\sin\theta_i} = \frac{v_2}{v_1} = \frac{n_1}{n_2}.
\end{equation}
Of course, the above expressions correspond to the {\em law of reflection}
and {\em Snell's law of refraction}, respectively.

For the case of oblique incidence, we need to consider {\em two} independent
wave polarizations  separately. The first polarization
has all the wave electric fields  parallel to the boundary whilst
the second has all  the wave magnetic fields parallel to the boundary 

Let us consider the first wave polarization. We can write unit vectors
in the directions of propagation of the incident, reflected, and transmitted
waves likso:
\begin{eqnarray}
\hat{\bf k}_i &=&\left(\sin\theta_i,\,0,\,\cos\theta_i\right),\\[0.5ex]
\hat{\bf k}_r &=& \left(\sin\theta_i,\,0,\,-\cos\theta_i\right),\\[0.5ex]
\hat{\bf k}_t &=&\left(\sin\theta_t,\,0,\,\cos\theta_t\right).\label{e9.172x}
\end{eqnarray}
The constant vectors associated with the incident wave are written
\begin{eqnarray}
{\bf E}_i &=& E_i\,{\bf e}_y,\\[0.5ex]
{\bf B}_i &=& \frac{E_i}{v_1}\,\left(-\cos\theta_i,\,0,\,\sin\theta_i\right),
\end{eqnarray}
where use has been made of Equation~(\ref{e9.140}). Likewise, the constant
vectors associated with the reflected and transmitted waves are
\begin{eqnarray}
{\bf E}_r &=& E_r\,{\bf e}_y,\\[0.5ex]
{\bf B}_r &=& \frac{E_r}{v_1}\,\left(\cos\theta_i,\,0,\,\sin\theta_i\right),
\end{eqnarray}
and
\begin{eqnarray}
{\bf E}_t &=& E_t\,{\bf e}_y,\\[0.5ex]
{\bf B}_t &=& \frac{E_t}{v_2}\,\left(-\cos\theta_t,\,0,\,\sin\theta_t\right),
\end{eqnarray}
respectively.

Now, the boundary condition (\ref{e9.147}) yields $E_{y\,1} = E_{y\,2}$,
or
\begin{equation}\label{e9.179}
E_i + E_r = E_t.
\end{equation}
Likewise, the boundary condition (\ref{e9.162}) gives $B_{z\,1} = B_{z\,2}$,
or
\begin{equation}
(E_i + E_r)\,\frac{\sin\theta_i}{v_1} = E_t\,\frac{\sin\theta_t}{v_2}.
\end{equation}
However, using Snell's law, (\ref{e9.169}), the above expression reduces to Equation~(\ref{e9.179}). Finally, the boundary condition (\ref{e9.148}) yields
$B_{x\,1} = B_{x\,2}$, or
\begin{equation}\label{e9.181}
(E_i -E_r)\,\frac{\cos\theta_i}{v_1} = E_t\,\frac{\cos\theta_t}{v_2}.
\end{equation}

It is convenient to define the parameters
\begin{equation}\label{e9.182x}
\alpha =\frac{\cos\theta_t}{\cos\theta_i},
\end{equation}
and
\begin{equation}\label{e9.183x}
\beta = \frac{v_1}{v_2}= \frac{n_2}{n_1}.
\end{equation}
Equations (\ref{e9.179}) and (\ref{e9.181}) can be solved in terms
of these parameters to give
\begin{eqnarray}\label{e9.184}
E_r = \left(\frac{1-\alpha\,\beta}{1+\alpha\,\beta}\right) E_i,\\[0.5ex]
E_t = \left(\frac{2}{1+\alpha\,\beta}\right) E_i.\label{e9.185}
\end{eqnarray}
These relations are known as {\em Fresnel equations}.

The wave intensity in the $z$-direction is given by
\begin{equation}
I_z =\frac{\langle {\bf E}\times{\bf B}\cdot{\bf e}_z\rangle}{\mu_0}= \frac{E_0\,B_0\,\cos\theta}{2\,\mu_0} = \frac{E_0^{\,2}\,\cos\theta}{2\,\mu_0\,v}.
\end{equation}
Hence, the coefficient of reflection is written
\begin{equation}
R = \left(\frac{E_r}{E_i}\right)^2 = \left(\frac{1-\alpha\,\beta}{1+\alpha\,\beta}\right)^2,
\end{equation}
whereas the coefficient of transmission takes the form
\begin{equation}
T = \frac{\cos\theta_t}{\cos\theta_i}\,\frac{v_1}{v_2}\left(
\frac{E_t}{E_i}\right)^2 = \alpha\,\beta \left(\frac{2}{1+\alpha\,\beta}\right)^2.
\end{equation}
Note that it is again the case that $R+T=1$. 

Let us now consider the second wave polarization. In this case, the
constant vectors associated with the incident, reflected, and transmitted
waves are written
\begin{eqnarray}
{\bf E}_i &=& E_i\,(\cos\theta_i,\, 0,\,-\sin\theta_i),\\[0.5ex]
{\bf B}_i &=& \frac{E_i}{v_1}\,{\bf e}_y,
\end{eqnarray}
and
\begin{eqnarray}
{\bf E}_r&=& E_r\,(\cos\theta_i,\, 0,\,\sin\theta_i),\\[0.5ex]
{\bf B}_r&=&- \frac{E_r}{v_1}\,{\bf e}_y,
\end{eqnarray}
and
\begin{eqnarray}
{\bf E}_t &=& E_t\,(\cos\theta_t,\, 0,\,-\sin\theta_t),\\[0.5ex]
{\bf B}_t &=& \frac{E_t}{v_2}\,{\bf e}_y,
\end{eqnarray}
respectively.
The boundary condition (\ref{e9.148}) yields $B_{y\,1}= B_{y\,2}$, or
\begin{equation}\label{e9.195}
\frac{E_i-E_r}{v_1} = \frac{E_t}{v_2}.
\end{equation}
Likewise, the boundary condition (\ref{e9.147}) gives $E_{x\,1}=E_{x\,2}$, or
\begin{equation}\label{e9.196}
(E_i+E_r)\,\cos\theta_i = E_t\,\cos\theta_t.
\end{equation}
Finally, the boundary condition (\ref{e9.161}) yields $\epsilon_1\,E_{z\,1} = \epsilon_2\,E_{z\,2}$, or
\begin{equation}
\epsilon_1\,(E_i-E_r)\,\sin\theta_i = \epsilon_2\,E_i\,\sin\theta_t.
\end{equation}
Making use of Snell's law, and the fact that $\epsilon = n^2$, the
above expression reduces to Equation~(\ref{e9.195}).

Solving Equations~(\ref{e9.165}) and (\ref{e9.196}), we obtain
\begin{eqnarray}\label{e9.198}
E_r &=&\left(\frac{\alpha-\beta}{\alpha+\beta}\right) E_i,\\[0.5ex]
E_t &=& \left(\frac{2}{\alpha+\beta}\right) E_i.\label{e9.199}
\end{eqnarray}
The associated coefficients of reflection and transmission take the
form 
\begin{eqnarray}
R &=&\left(\frac{\alpha-\beta}{\alpha+\beta}\right)^2,\\[0.5ex]
T &=& \alpha\,\beta\left(\frac{2}{\alpha+\beta}\right)^2,
\end{eqnarray}
respectively. As usual, $R+T=1$. 

Note that at oblique incidence the Fresnel equations, (\ref{e9.184}) and
(\ref{e9.185}), for the wave polarization in which the electric
field is parallel to the boundary are {\em different} to the Fresnel equations,
(\ref{e9.198}) and (\ref{e9.199}), for the wave polarization
in which the magnetic field is parallel to the boundary. This implies that
the coefficients of reflection and transmission for these two wave polarizations
are, in general, {\em different}.

\begin{figure}
\epsfysize=2.25in
\centerline{\epsffile{chapter9/fig9.4.eps}}
\caption{\em Coefficients of reflection (solid curves) and transmission (dashed curves) for oblique incidence from air ($n=1.0$) to
glass ($n=1.5$). The left-hand panel shows the wave polarization
for which the electric field is parallel to the boundary, whereas the
right-hand panel shows the wave polarization for which the
magnetic field is parallel to the boundary.}\label{ffresnel}
\end{figure}

Figure~\ref{ffresnel} shows the coefficients of reflection and transmission
for oblique incidence from air ($n_1=1.0$) to
glass ($n_2=1.5$).  In general, it can
be seen that the coefficient of reflection rises, and the coefficient of
transmission falls, as the angle of incidence increases. Note, however,
that for the wave polarization in which the magnetic field is parallel to the boundary there is a particular angle of incidence,
know as the {\em Brewster angle},
at which the reflected intensity is {\em zero}. There is no similar behaviour for
the wave polarization in which the electric field is parallel to the boundary. 

It follows from Equation~(\ref{e9.198}) that the Brewster angle corresponds
to the condition
\begin{equation}
\alpha=\beta,
\end{equation}
or
\begin{equation}
\beta^2 = \frac{\cos^2\theta_t}{\cos^2\theta_i} = \frac{1-\sin^2\theta_t}{1-\sin^2\theta_i}=\frac{1-\sin^2\theta_i/\beta^2}{1-\sin^2\theta_i},
\end{equation}
where use has been made of Snell's law. The above expression
reduces to
\begin{equation}
\sin\theta_i = \frac{\beta}{\sqrt{1+\beta^2}},
\end{equation}
or $\tan\theta_i = \beta = n_2/n_1$. Hence, the Brewster angle satisfies
\begin{equation}
\theta_B = \tan^{-1}\left(\frac{n_2}{n_1}\right).
\end{equation}
If unpolarized light is incident on an air/glass (say) boundary at the Brewster angle
then the reflected light is $100\%$ plane polarized.

\section{Total Internal Reflection}
Let us again consider an electromagnetic wave obliquely incident on a dielectric boundary.  According to Equation~(\ref{e9.169}), the
angle of refraction $\theta_t$ is related to the angle of incidence $\theta_i$
via
\begin{equation}
\sin\theta_t = \frac{n_1}{n_2}\,\sin\theta_i.
\end{equation}
This formula presents no problems when $n_1< n_2$. However, if
$n_1> n_2$ then the formula predicts that $\sin\theta_t$ is greater than
unity when the angle of incidence exceeds some critical angle given by
\begin{equation}
\theta_c = \sin^{-1}(n_2/n_1).
\end{equation}
Obviously, in this situation, we can no longer interpret $\sin\theta_t$ as the sine of
an angle. Moreover, $\cos\theta_t \equiv (1-\sin^2\theta_t)^{1/2}$ can no longer be interpreted as the cosine of an angle. However, these quantities still specify the wave-vector of the
transmitted wave. In fact, from Equation~(\ref{e9.172x}), 
\begin{equation}
{\bf k}_t = k_t\,(\sin\hat{\theta}_t,\, 0,\, {\rm i}\,\cos\hat{\theta}_t),
\end{equation}
where $k_t=n_2\,\omega/c$, $\cos\hat{\theta}_t =(\sin^2\hat{\theta}_t-1)^{1/2}$, and
\begin{equation}
\sin\hat{\theta}_t = \frac{\sin\theta_i}{\sin\theta_c}.
\end{equation}
Here, the hat on $\hat{\theta}_t$ is to remind us that this  quantity is not a real angle.
Now, the transmitted wave varies as
\begin{equation}
{\rm e}^{\,{\rm i}\,({\bf k}_t\cdot{\bf r} - \omega\,t)} = {\rm e}^{-k_t\,\cos\hat{\theta}_t}\,{\rm e}^{\,{\rm i}\,(k_t\,\sin\hat{\theta}_t-\omega\,t)}.
\end{equation}
Hence, we conclude that when $\theta_i>\theta_c$ the transmitted wave
is {\em evanescent}: {\em i.e.}, it decays exponentially, rather than propagating, in medium 2.

When $\theta_i>\theta_c$ the parameter $\alpha$, defined in Equation~(\ref{e9.182x}), becomes {\em complex}. In fact, $\alpha\rightarrow {\rm i}\,\hat{\alpha}$, where
\begin{equation}
\hat{\alpha} = \frac{\cos\hat{\theta}_t}{\cos\theta_i}.
\end{equation}
Note that the parameter $\beta$, defined in Eq.~(\ref{e9.183x}), remains
real. Hence, from Equations~(\ref{e9.184}) and (\ref{e9.198}), the
relationship between $E_r$ and $E_i$ for the two previously discussed wave polarizations, in which
either the electric field or the magnetic field
is parallel to the boundary, are
\begin{eqnarray}
E_r &=& \left(\frac{1-{\rm i}\,\hat{\alpha}\,\beta}{1+{\rm i}\,\hat{\alpha}\,\beta}\right)E_i,\\[0.5ex]
E_r &=& \left(\frac{{\rm i}\,\hat{\alpha}-\beta}{{\rm i}\,\hat{\alpha}+\beta}\right)E_i,
\end{eqnarray}
respectively. In both cases, the associated coefficients of reflection are 
{\em unity}: {\em i.e.}, 
\begin{equation}
R = \left|\frac{E_r}{E_i}\right|^2 = 1.
\end{equation}
In other words, the incident wave undergoes  complete reflection at the boundary.
This phenomenon is called {\em total internal reflection}, and occurs
whenever a wave is incident on a boundary separating a medium of high
refractive index from  a medium of low refractive index, and the angle of
incidence exceeds the critical angle, $\theta_c$. 

\begin{figure}
\epsfysize=2.25in
\centerline{\epsffile{chapter9/fig9.5.eps}}
\caption{\em Coefficients of reflection (solid curves) and transmission (dashed curves) for oblique incidence from water ($n=1.33$) to
air ($n=1.0$). The left-hand panel shows the wave polarization
for which the electric field is parallel to the boundary, whereas the
right-hand panel shows the wave polarization for which the
magnetic field is parallel to the boundary.}\label{ffresnel1}
\end{figure}

Figure~\ref{ffresnel1} shows the coefficients of reflection and transmission
for oblique incidence from water  ($n_1=1.33$) to
air ($n_2=1.0$). In this case, the critical angle is $\theta_c= 48.8^\circ$. 

Note that when total internal reflection takes place the evanescent transmitted
wave penetrates a few wavelengths into the lower refractive index medium, since (as is easily demonstrated) the amplitude of this wave is non-zero.
The existence of the evanescent wave can be demonstrated using the
apparatus pictured in Figure~\ref{fprism}. Here, we have two right-angled
glass prisms separated by a small air gap of width $d$. Light incident
on the internal surface of the first prism is internally reflected (assuming that $\theta_c<45^\circ$). However,
if the spacing $d$ is not too much larger than the  wavelength of the light (in air) then the
evanescent wave in the air gap still has a finite amplitude when it reaches the second prism. In this case, a detectable transmitted
wave is excited in the second prism. Obviously, the amplitude of this
wave has an inverse exponential dependance on the width of the gap. 
This effect is called {\em frustrated total internal reflection}\/ and
is analogous to the {\em tunneling}\/ of wave-functions through
potential barriers in Quantum Mechanics.

\begin{figure}
\epsfysize=2.25in
\centerline{\epsffile{chapter9/fig9.6.eps}}
\caption{\em Frustrated total internal reflection.}\label{fprism}
\end{figure}

\section{Optical Coatings}
Consider an optical instrument, such as a refracting telescope, which makes
use of multiple glass lenses.  Let us examine the light-ray running along the optical
axis of the instrument. This ray is normally incident on all of the lenses. 
However, according to the analysis of Section~\ref{snorm}, whenever
the ray enters or leaves a lens it is partly reflected. In fact, for glass of
refractive index 1.5 the transmission coefficient across an air/glass or
a glass/air 
boundary is about $96\%$. Hence, approximately $8\%$ of the light is lost each time the
ray passes completely through a lens. Clearly, this level of attenuation is unacceptable in an instrument which contains many lens,
especially if it is being used to view faint objects.

It turns out that the above mentioned problem can be alleviated by coating
all  the lenses of the instrument in question with a thin layer of dielectric whose
refractive index is intermediate between that of air and glass. Consider the situation shown in Figure~\ref{fcoat}. Here, a
light-ray is normally incident on a boundary between medium 1,
of refractive index $n_1$, and medium 3, of refractive index $n_3$. 
Here, medium 1 represents air, and medium 3 represents glass, or
{\em vice versa}. Suppose that the glass is covered with a thin optical
coating, referred to as medium 2, of thickness $d$, and refractive
index $n_2$. 

\begin{figure}
\epsfysize=2.25in
\centerline{\epsffile{chapter9/fig9.7.eps}}
\caption{\em An optical coating.}\label{fcoat}
\end{figure}

In the notation of Section~\ref{snorm}, the incident wave is written
\begin{eqnarray}
{\bf E}(z,t) &=& E_i\,{\rm e}^{\,{\rm i}\,(k_1\,z-\omega\,t)}
\,{\bf e}_x,\\[0.5ex]
{\bf B}(z,t)&=&\frac{E_i}{v_1}\,{\rm e}^{\,{\rm i}\,(k_1\,z-\omega\,t)}\,{\bf e}_y,
\end{eqnarray}
where $v_1=c/n_1$ is the phase-velocity in medium 1, and $k_1=\omega/v_1$. Likewise, the reflected wave takes the form
\begin{eqnarray}
{\bf E}(z,t) &=& E_r\,{\rm e}^{\,{\rm i}\,(-k_1\,z-\omega\,t)}\,{\bf e}_x,\\[0.5ex]
{\bf B}(z,t) &=&-\frac{E_r}{v_1}\,{\rm e}^{\,{\rm i}\,(-k_1\,z-\omega\,t)}\,{\bf e}_y.
\end{eqnarray}
The wave traveling to the right in medium 2 is written
\begin{eqnarray}
{\bf E}(z,t) &=& E_a\,{\rm e}^{\,{\rm i}\,(k_2\,z-\omega\,t)}
\,{\bf e}_x,\\[0.5ex]
{\bf B}(z,t)&=&\frac{E_a}{v_2}\,{\rm e}^{\,{\rm i}\,(k_2\,z-\omega\,t)}\,{\bf e}_y,
\end{eqnarray}
where $v_2=c/n_2$ is the phase-velocity in medium 2, and $k_2=\omega/v_2$. 
Likewise, the wave traveling to the left takes the form
\begin{eqnarray}
{\bf E}(z,t) &=& E_b\,{\rm e}^{\,{\rm i}\,(-k_2\,z-\omega\,t)}\,{\bf e}_x,\\[0.5ex]
{\bf B}(z,t) &=&-\frac{E_b}{v_2}\,{\rm e}^{\,{\rm i}\,(-k_2\,z-\omega\,t)}\,{\bf e}_y.
\end{eqnarray}
Finally, the transmitted wave is  written
\begin{eqnarray}
{\bf E}(z,t) &=& E_t\,{\rm e}^{\,{\rm i}\,[k_3\,(z-d)-\omega\,t]}\,{\bf e}_x,\\[0.5ex]
{\bf B}(z,t) &=&\frac{E_t}{v_3}\,{\rm e}^{\,{\rm i}\,[k_2\,(z-d)-\omega\,t]}\,{\bf e}_y,
\end{eqnarray}
where $v_3=c/n_3$ is the phase-velocity in medium 3, and $k_3=\omega/v_3$.

Continuity of $E_x$  and $B_y$ at $z=0$ yield
\begin{eqnarray}
E_i+E_r &=& E_a+E_b,\\[0.5ex]
\frac{E_i-E_r}{v_1} &=& = \frac{E_a-E_b}{v_2},
\end{eqnarray}
respectively, whereas continuity of $E_x$ and $B_y$ at $z=d$ give
\begin{eqnarray}
E_a\,{\rm e}^{\,{\rm i}\,k_2\,d}+ E_b\,{\rm e}^{-{\rm i}\,k_2\,d} &=& E_t,\\[0.5ex]
\frac{E_a\,{\rm e}^{\,{\rm i}\,k_2\,d}- E_b\,{\rm e}^{-{\rm i}\,k_2\,d}}{v_2} &=& \frac{E_t}{v_3},
\end{eqnarray}
respectively. At this point, it is convenient to make the special choice
$k_2\,d=\pi/2$. This corresponds to the optical coating being exactly {\em one-quarter}\/ of a wavelength thick. It follows that
\begin{eqnarray}
E_a - E_b &=& - {\rm i}\,E_t,\\[0.5ex]
E_a + E_b &=& -{\rm i}\,\frac{n_3}{n_2}\,E_t.
\end{eqnarray}
The above equations can be solved to give
\begin{eqnarray}
E_r &=& -\left(\frac{1-\alpha}{1+\alpha}\right) E_i,\\[0.5ex]
E_t &=&  \frac{n_1}{n_2}\,\frac{2\,{\rm i}}{1+\alpha}\,E_i,
\end{eqnarray}
where
\begin{equation}
\alpha = \frac{n_1\,n_3}{n_2^{\,2}}.
\end{equation}
Thus, the overall coefficient of reflection is 
\begin{equation}\label{e9.234x}
R = \left|\frac{E_r}{E_i}\right|^2 = \left(\frac{1-\alpha}{1+\alpha}\right)^2,
\end{equation}
whereas the overall coefficient of transmission is
\begin{equation}\label{e9.235x}
T = \frac{n_3}{n_1} \left|\frac{E_t}{E_i}\right|^2  = \frac{4\,\alpha}{(1+\alpha)^2}.
\end{equation}
Suppose finally that
\begin{equation}
n_2 = \sqrt{n_1\,n_3}:
\end{equation}
{\em i.e.}, the refractive index of the coating is the {\em geometric mean}\/
of that of air and glass. In this case, $\alpha=1$, and it follows from
Equations~(\ref{e9.234x}) and (\ref{e9.235x}) that there is zero
reflection, and 100\% transmission, at the boundary. Hence, by
coating lenses with a one-quarter wavelength thickness of a substance
whose refractive index is (approximately) the geometric mean between those of air and glass ({\em e.g.}, Magnesium fluoride, whose refractive index is  $1.38$), we can
completely eliminate unwanted reflections. This technique is
widely used in high-quality optical instruments. Note that
the physics of quarter-wavelength optical coatings is analogous to
that of quarter-wave transformers in transmission lines---see Section~\ref{strans}.

\section{Reflection at a Metallic Boundary}
Let us now consider the reflection of electromagnetic radiation by
a metallic surface. This investigation is obviously relevant to
optical instruments, such as reflecting telescopes, which make use of
mirrors. For the sake of simplicity, we shall restrict our investigation
to the case of normal incidence. 

A metal is, by definition, a good electrical conductor. According to Equation~(\ref{ekgood}), the wave-number of an electromagnetic wave of frequency $\omega$ in a good conductor of
conductivity $\sigma$ (and true dielectric constant unity) is 
\begin{equation}
k\simeq \sqrt{\,{\rm i}\,\mu_0\,\sigma\,\omega}.
\end{equation}
Hence, it follows that the effective refractive index of the conductor is
\begin{equation}
n = \frac{k\,c}{\omega}\simeq \sqrt{\frac{{\rm i}\,\sigma}{\epsilon_0\,\omega}}.
\end{equation}
Note that the good conductor ordering $\sigma\gg \epsilon_0\,\omega$
ensures that $|n|\gg 1$.

For the case of a light-ray in air reflecting at normal incidence off a metal mirror, we can employ the previously derived formula (\ref{e9.152})
with $n_1=1$ and $n_2=n$, where $n$ is specified above. We obtain
\begin{equation}
\frac{E_r}{E_i} = \frac{1-n}{1+n} \simeq -1 + \frac{2}{n},
\end{equation}
where we have made use of the fact that $|n|\gg 1$. Hence, the coefficient
of reflection of the mirror takes the form
\begin{equation}
R = \left|\frac{E_r}{E_i}\right|^2\simeq 1 - {\rm Re}\left(\frac{4}{n}\right),
\end{equation}
or
\begin{equation}
R \simeq 1 - \sqrt{\frac{8\,\epsilon_0\,\omega}{\sigma}}.
\end{equation}

High-quality metallic mirrors are generally coated in Silver, whose conductivity
is $6.3\times 10^7\,(\Omega\,{\rm m})^{-1}$. It follows, from the above
formula, that  at optical
frequencies ($\omega = 4\times 10^{15}\,{\rm rad./s}$) the coefficient
of reflection of a silvered mirror is $R\simeq 93.3\%$. This implies that
about $7\%$ of the light incident on a silvered mirror is absorbed, rather than being reflected. This rather severe light loss can be
problematic in instruments, such as astronomical telescopes, which are used to
view faint objects. 

\section{Wave-Guides}
A wave-guide is a hollow conducting pipe, of uniform cross-section, used to transport high-frequency
electromagnetic waves (generally, in the microwave band) from one
point to another. The main advantage of wave-guides is their relatively
low level of radiation losses (since the electric and
magnetic fields are completely enclosed by a conducting wall) compared to transmission lines.

Consider a vacuum-filled wave-guide which runs parallel to the $z$-axis.
An electromagnetic wave trapped inside the wave-guide satisfies Maxwell's equations for free space:
\begin{eqnarray}
\nabla\cdot{\bf E} &=& 0,\\[0.5ex]
\nabla\cdot{\bf B} &=&0,\\[0.5ex]
\nabla\times{\bf E} &=&-\frac{\partial {\bf B}}{\partial t},\\[0.5ex]
\nabla\times{\bf B} &=& \frac{1}{c^2}\frac{\partial {\bf E}}{\partial t}.
\end{eqnarray}
Let $\partial/\partial t\equiv -{\rm i}\,\omega$, and $\partial/\partial z\equiv
{\rm i}\,k$, where $\omega$ is the wave frequency, and $k$ the wave-number parallel to the axis of the wave-guide. 
It follows that
\begin{eqnarray}\label{e9.210}
\frac{\partial E_x}{\partial x} + \frac{\partial E_y}{\partial y} + {\rm i}\,k\,E_z &=& 0,\\[0.5ex]\label{e9.211}
\frac{\partial B_x}{\partial x} + \frac{\partial B_y}{\partial y} + {\rm i}\,k\,B_z &=& 0,\\[0.5ex]\label{e9.212}
{\rm i}\,\omega\, B_x&=& \frac{\partial E_z}{\partial y} - {\rm i}\,k\,E_y,\\[0.5ex]\label{e9.213}
{\rm i}\,\omega\,B_y &=& -\frac{\partial E_z}{\partial x} + {\rm i}\,k\,E_x,\\[0.5ex]\label{e9.214}
{\rm i}\,\omega\,B_z&=&\frac{\partial E_y}{\partial x} - \frac{\partial E_x}{\partial y},\\[0.5ex]\label{e9.215}
{\rm i}\,\frac{\omega}{c^2}\,E_x &=&
- \frac{\partial B_z}{\partial y} + {\rm i}\,k\,B_y,\\[0.5ex]\label{e9.216}
{\rm i}\,\frac{\omega}{c^2}\,E_y &=& \frac{\partial B_z}{\partial x}
- {\rm i}\,k\,B_x,\\[0.5ex]\label{e9.217}
{\rm i}\,\frac{\omega}{c^2}\,E_z &=& -\frac{\partial B_y}{\partial x}+
\frac{\partial B_x}{\partial y}.
\end{eqnarray}
Equations (\ref{e9.213}) and (\ref{e9.215}) yield
\begin{equation}
E_x = {\rm i}\left(\omega\,\frac{\partial B_z}{\partial y} + k\,\frac{\partial E_z}{\partial x}\right)\left(\frac{\omega^2}{c^2}-k^2\right)^{-1},
\end{equation}
and
\begin{equation}
B_y = {\rm i}\,\left(\frac{\omega}{c^2}\frac{\partial E_z}{\partial x}
+ k\,\frac{\partial B_z}{\partial y}\right)\left(\frac{\omega^2}{c^2}-k^2\right)^{-1}.
\end{equation}
Likewise, Equations~(\ref{e9.212}) and (\ref{e9.216}) yield
\begin{equation}
E_y = {\rm i}\left(-\omega\,\frac{\partial B_z}{\partial x} + k\,\frac{\partial E_z}{\partial y}\right)\left(\frac{\omega^2}{c^2}-k^2\right)^{-1},
\end{equation}
and
\begin{equation}
B_x = {\rm i}\,\left(-\frac{\omega}{c^2}\frac{\partial E_z}{\partial y}
+ k\,\frac{\partial B_z}{\partial x}\right)\left(\frac{\omega^2}{c^2}-k^2\right)^{-1}.
\end{equation}
These equations can be combined to give
\begin{eqnarray}\label{e9.222}
{\bf E}_t&=& {\rm i}\left(\omega\,\nabla B_z\times{\bf e}_z + k\,\nabla
E_z\right)\left(\frac{\omega^2}{c^2}-k^2\right)^{-1},\\[0.5ex]\label{e9.223}
{\bf B}_t &=& {\rm i}\left(-\frac{\omega}{c^2}\,\nabla E_z\times{\bf e}_z + k\,\nabla
B_z\right)\left(\frac{\omega^2}{c^2}-k^2\right)^{-1}.
\end{eqnarray}
Here, ${\bf E}_t$ and ${\bf B}_t$ are the {\em transverse} electric
and magnetic fields: {\em i.e.}, the electric and
magnetic fields in the $x$-$y$ plane. It is clear, from Equations~(\ref{e9.222}) and (\ref{e9.223}), that the transverse fields are fully determined once the
longitudinal fields, $E_z$ and $B_z$, are known.

Substitution of Equations~(\ref{e9.222}) and (\ref{e9.223}) into Equations~(\ref{e9.214}) and (\ref{e9.217}) yields the equations satisfied by
the longitudinal fields:
\begin{eqnarray}\label{e9.224}
\left(\frac{\partial^2}{\partial x^2} + \frac{\partial^2}{\partial y^2}\right)\!
E_z + \left(\frac{\omega^2}{c^2}-k^2\right)\!E_z&=&0,\\[0.5ex]\label{e9.225}
\left(\frac{\partial^2}{\partial x^2} + \frac{\partial^2}{\partial y^2}\right)\!
B_z + \left(\frac{\omega^2}{c^2}-k^2\right)\!B_z&=&0.
\end{eqnarray}
The remaining equations, (\ref{e9.210}) and (\ref{e9.211}), are automatically
satisfied provided Equations~(\ref{e9.222})--(\ref{e9.225}) are satisfied.

We expect ${\bf E} = {\bf B} = {\bf 0}$ inside the walls of the wave-guide,
assuming that they are perfectly conducting. Hence, the appropriate
boundary conditions at the walls are
\begin{eqnarray}
E_{\parallel} &=& 0,\\[0.5ex]
B_{\perp} &=& 0.
\end{eqnarray}
It follows, by inspection of Equations~(\ref{e9.222}) and (\ref{e9.223}), that
these boundary conditions are  satisfied provided
\begin{eqnarray}\label{e9.228}
E_z &=& 0,\\[0.5ex]\label{e9.229}
\hat{\bf n}\cdot\nabla B_z &=& 0,
\end{eqnarray}
at the walls. Here, $\hat{\bf n}$ is a unit vector normal to the walls.
Hence, the electromagnetic fields inside the wave-guide are fully
specified by solving Equations~(\ref{e9.224}) and (\ref{e9.225}), subject to
the boundary conditions (\ref{e9.228}) and (\ref{e9.229}), respectively.

Equations~(\ref{e9.224}) and (\ref{e9.225}) support two independent types
of solution. The first type has $E_z=0$, and is consequently called a {\em transverse electric}, or TE,  mode. Conversely, the
second type  has $B_z=0$, and is called a {\em transverse
magnetic}, or TM, mode.

	Consider the specific example of a {\em rectangular}\/ wave-guide, with conducting walls
at $x=0, a$,  and $y=0, b$. For a TE mode, the longitudinal
magnetic field can be written
\begin{equation}
B_z(x,y) = B_0\,\cos(k_x\,x)\,\cos(k_y\,y),
\end{equation}
The boundary condition (\ref{e9.229}) requires that
$\partial B_z/\partial x =0$ at $x=0, a$, and $\partial B_z/\partial y=0$
at $y=0, b$. It follows that
\begin{eqnarray}
k_x &=& \frac{m\,\pi}{a},\\[0.5ex]
k_y&=& \frac{n\,\pi}{b},
\end{eqnarray}
where $m=0, 1, 2, \cdots$, and $n=0, 1, 2, \cdots$. Clearly, there are
many different kinds of TE mode, corresponding to the many different
choices of $m$ and $n$. Let us refer to a mode corresponding to
a particular choice of $m, n$ as a ${\rm TE}_{mn}$ mode. Note, however, that there
is no ${\rm TE}_{00}$ mode, since $B_z(x,y)$ is uniform in this case.
 According to
Equation~(\ref{e9.225}), the dispersion relation for the ${\rm TE}_{mn}$ mode is
given by
\begin{equation}\label{e9.233}
k^2\,c^2 = \omega^2 - \omega_{mn}^{\,2},
\end{equation}
where
\begin{equation}\label{e9.234}
\omega_{mn} = c\,\pi\,\sqrt{\frac{m^2}{a^2} + \frac{n^2}{b^2}}.
\end{equation}

According to the dispersion relation (\ref{e9.233}), $k$ is imaginary for
$\omega < \omega_{mn}$. In other words, for
wave frequencies below $\omega_{mn}$, the ${\rm TE}_{mn}$ mode
fails to propagate down the wave-guide, and is instead attenuated.  Hence, $\omega_{mn}$
is termed the {\em cut-off frequency} for the ${\rm TE}_{mn}$ mode.
Assuming that $a>b$, the TE mode with the lowest cut-off frequency is
the ${\rm TE}_{10}$ mode, where
\begin{equation}
\omega_{10} = \frac{c\,\pi}{a}.
\end{equation}

For frequencies above the cut-off frequency, the phase-velocity of the
${\rm TE}_{mn}$ mode is given by
\begin{equation}\label{e9.236}
v_p = \frac{\omega}{k} = \frac{c}{\sqrt{1-\omega_{mn}^{\,2}/\omega^2}},
\end{equation}
which is greater than $c$. However, the group-velocity takes the form
\begin{equation}\label{e9.237}
v_g = \frac{d\omega}{d k} = c\,\sqrt{1-\omega_{mn}^{\,2}/\omega^2},
\end{equation}
which is always less than $c$. Of course, energy is transmitted down the wave-guide
at the group-velocity, rather than the phase-velocity. Note that the group-velocity goes to zero as the
wave frequency approaches the cut-off frequency.

For a TM mode, the longitudinal electric field  can be written
\begin{equation}
E_z(x,y) = E_0\,\sin(k_x\,x)\,\sin(k_y\,y),
\end{equation}
The boundary condition (\ref{e9.228}) requires that
$E_z=0$ at $x=0, a$, and $y=0, b$. It follows that
\begin{eqnarray}
k_x &=& \frac{m\,\pi}{a},\\[0.5ex]
k_y&=& \frac{n\,\pi}{b},
\end{eqnarray}
where $m=1, 2, \cdots$, and $n=1, 2, \cdots$. The dispersion relation
for the ${\rm TM}_{mn}$ mode is also given by Equation~(\ref{e9.233}). 
Hence, Equations~(\ref{e9.236}) and (\ref{e9.237}) also apply to TM modes.
However, the TM mode with the lowest cut-off frequency is the
${\rm TM}_{11}$ mode, where
\begin{equation}
\omega_{11} = c\,\pi\,\sqrt{\frac{1}{a^2}+\frac{1}{b^2}}>\omega_{10}.
\end{equation}
It follows that the mode with the lowest cut-off frequency is always
a TE mode.

There is, in principle, a third type of mode which can propagate down
a wave-guide. This third mode type is characterized by $E_z=B_z=0$,
and is consequently called a {\em transverse electromagnetic}, or
TEM, mode. It is easily seen, from an inspection of
Equations~(\ref{e9.212})--(\ref{e9.217}), that a TEM mode satisfies
\begin{equation}\label{e9.242}
\omega^2= k^2\,c^2,
\end{equation}
and
\begin{eqnarray}
{\bf E}_t &=& - \nabla\phi,\\[0.5ex]
{\bf B}_t &=& c^{-1}\,\nabla\phi\times{\bf e}_z,
\end{eqnarray}
where $\phi(x,y)$ satisfies
\begin{equation}\label{e9.245}
\nabla^2\phi = 0.
\end{equation}
The boundary conditions (\ref{e9.228}) and (\ref{e9.229}) imply
that
\begin{equation}\label{e9.246}
\phi = {\rm constant}
\end{equation}
at the walls. However, there is no non-trivial solution of Equations~(\ref{e9.245})
and (\ref{e9.246}) for a conventional wave-guide. In other words,
conventional wave-guides {\em do not} support TEM modes. 
It turns out that only wave-guides with {\em central conductors}
support TEM modes. Consider, for instance, a co-axial wave-guide
in which the electric and magnetic fields are trapped between two
co-axial cylindrical conductors of radius $a$ and $b$ (with $b>a$).
In this case, $\phi=\phi(r)$, and Equation~(\ref{e9.245}) reduces to
\begin{equation}
\frac{1}{r}\frac{\partial}{\partial r}\left( r\,\frac{\partial \phi}{\partial r}\right) = 0,
\end{equation}
where $r$ is a standard cylindrical polar coordinate.
The boundary condition (\ref{e9.246}) is automatically satisfied at $r=a$ and
$r=b$. 
The above equation has the following non-trivial solution:
\begin{equation}
\phi(r) = \phi_b\,\ln(r/b).
\end{equation}
Note, however, that the inner conductor {\em must} be present, otherwise
$\phi\rightarrow\infty$ as $r\rightarrow 0$, which is unphysical.
According to the dispersion relation (\ref{e9.242}), TEM modes have
no cut-off frequency, and have the phase-velocity (and group-velocity) $c$.
Indeed, this type of mode is the same as that supported by a transmission line
(see Section~\ref{strans}).

{\small
\section{Exercises}
\renewcommand{\theenumi}{9.\arabic{enumi}}
\begin{enumerate}
\item Consider an electromagnetic wave propagating through a
non-dielectric, non-magnetic medium containing free charge
density $\rho$ and free current density ${\bf j}$. Demonstrate
from Maxwell's equations that the associated wave equations
take the form
\begin{eqnarray}
\nabla^2{\bf E} - \frac{1}{c^2}\,\frac{\partial^2 {\bf E}}{\partial t^2} &=& \frac{\nabla \rho}{\epsilon_0} + \mu_0\,\frac{\partial {\bf j}}{\partial t},\nonumber\\[0.5ex]
\nabla^2{\bf B} - \frac{1}{c^2}\,\frac{\partial^2 {\bf B}}{\partial t^2} &=& 
- \mu_0\,\nabla\times{\bf j}.\nonumber
\end{eqnarray}
\item A spherically symmetric charge distribution undergoes purely
radial oscillations. Show that no electromagnetic waves are emitted. [Hint: Show that there is no magnetic field.]
\item A general electromagnetic wave-pulse propagating in the $z$-direction at velocity $u$
is written
\begin{eqnarray}
{\bf E} &= &P(z-u\,t)\,{\bf e}_x + Q(z-u\,t)\,{\bf e}_y+ R(z-u\,t)\,{\bf e}_z,\nonumber\\[0.5ex]
{\bf B} &= &\frac{S(z-u\,t)}{u}\,{\bf e}_x + \frac{T(z-u\,t)}{u}\,{\bf e}_y+ \frac{U(z-u\,t)}{u}\,{\bf e}_z,\nonumber
\end{eqnarray}
where $P$, $Q$, $R$, $S$, $T$, and $U$ are arbitrary functions. In order
to exclude electrostatic and magnetostatic fields, these functions are subject to
the constraint that $\langle P\rangle=\langle Q\rangle=\langle R\rangle=\langle S\rangle=\langle T\rangle=\langle U\rangle =0$, where
$$
\langle P\rangle = \int_{-\infty}^{\infty}P(x)\,dx.
$$
Suppose that the pulse propagates through a uniform dielectric medium of
dielectric constant $\epsilon$. Demonstrate from Maxwell's equation that
$u = c/\sqrt{\epsilon}$, $R=U=0$, $S=-Q$, and $T=P$. Incidentally, this result implies that a general
wave-pulse is characterized by {\em two}\/ arbitrary functions, corresponding
to the two possible independent polarizations of the pulse.
\item A medium is such that the product of the phase and group
velocities of electromagnetic waves is equal to $c^2$ at all wave
frequencies. Demonstrate that the dispersion relation for
electromagnetic waves takes the form
$$
\omega^2 = k^2\,c^2+\omega_0^{\,2},
$$
where $\omega_0$ is a constant.
\item Consider a uniform plasma of plasma frequency $\omega_p$ containing a uniform magnetic field $B_0\,{\bf e}_z$. Show that left-hand
circularly polarized electromagnetic waves can only propagate parallel to the
magnetic field provided that $\omega > -\Omega/2 + \sqrt{\Omega^2/4+\omega_p^2}$, where $\Omega=e\,B_0/m_e$ is the electron cyclotron frequency.
Demonstrate that right-hand circularly polarized electromagnetic waves can only propagate
parallel to the magnetic field provided that their frequencies do
not lie in the range $\Omega\leq  \omega\leq\Omega/2 + \sqrt{\Omega^2/4+\omega_p^2}$. You may neglect the finite mass of the ions.
\item Consider an electromagnetic wave propagating through a
nonuniform dielectric medium whose dielectric constant $\epsilon$ is a
function of ${\bf r}$. Demonstrate that the associated
wave equations take the form
\begin{eqnarray}
\nabla^2{\bf E} - \frac{\epsilon}{c^2}\,\frac{\partial^2{\bf E}}{\partial t^2}
&=& - \nabla\left(\frac{\nabla\epsilon\cdot{\bf E}}{\epsilon}\right),\nonumber\\[0.5ex]
\nabla^2{\bf B} - \frac{\epsilon}{c^2}\,\frac{\partial^2{\bf B}}{\partial t^2}
&=& - \frac{\nabla\epsilon\times (\nabla\times {\bf B})}{\epsilon}.\nonumber
\end{eqnarray}
\item Consider a light-wave normally incident on a uniform
pane of glass of thickness $d$ and refractive index $n$. Show
that the coefficient of transmission though the pane takes the form
$$
T^{-1} = 1 + \left[\frac{n^2-1}{2\,n}\,\sin (k\,d)\right]^2,
$$
where $k$ is the wave-number within the glass.
\item Consider an electromagnetic wave obliquely incident on a plane
boundary between two transparent magnetic media of relative permeabilities 
$\mu_1$ and $\mu_2$. Find the coefficients of reflection and transmission
as functions of the angle of incidence for the wave polarizations in
which all electric fields are parallel to the boundary and all magnetic
fields are parallel to the boundary. Is there a Brewster angle? If so, what is it?
Is it possible to obtain total reflection? If so, what is the critical angle of
incidence required to obtain total reflection?
\item Suppose that a light-ray is incident on the front (air/glass) interface of a uniform pane
of glass of refractive index $n$ at the Brewster angle. Demonstrate that the refracted ray
is also incident on the rear (glass/air) interface of the pane at the Brewster
angle.
\item Consider an electromagnetic wave propagating through a good conductor. Demonstrate that the energy density of the wave's magnetic component dominates that of its electric component. In addition, show that the phase
of the wave's magnetic component  lags that of its electric component by
$45^\circ$. 
\item Demonstrate that the electric and magnetic fields inside a wave-guide are mutually orthogonal.
\item Consider a $TE_{mn}$ mode in a rectangular wave-guide of dimensions
$a$ and $b$. Calculate the mean electromagnetic energy per unit length, as
well as
the mean electromagnetic energy flux down the wave-guide. Demonstrate
that the ratio of the mean energy flux to the mean energy per unit length
is equal to the group-velocity of the mode.

\end{enumerate}
\renewcommand{\theenumi}{arabic{enumi}}
}
