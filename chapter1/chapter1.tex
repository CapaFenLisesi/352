\chapter{Introduction}
The main topic of this book is {\em Maxwell's Equations}. These are a set of
{\em eight},  scalar, first-order partial differential equations which constitute a 
{\em complete}\/ description of classical electric and magnetic phenomena. To be more
exact, Maxwell's equations constitute a complete description of the
classical  behaviour of electric and magnetic {\em fields}.  

Electric and magnetic fields were first introduced into electromagnetic theory merely as mathematical constructs designed to facilitate the calculation
of the forces exerted between electric charges and between current carrying wires.
However, physicists soon came to realize that  the physical existence of these fields is  key
to making Classical Electromagnetism  consistent with
Einstein's Special Theory of Relativity. In fact, Classical Electromagnetism was the first example of a so-called {\em field theory}\/ to be discovered in Physics. Other, subsequently discovered,  field theories  include General Relativity, Quantum Electrodynamics, and Quantum Chromodynamics.

At any given point in space, an electric or magnetic field possesses two
pro\-perties---a {\em magnitude}\/ and a {\em direction}. In
general, these properties
 vary (continuously) from
point to point. It is conventional to represent such a field in terms
of its components measured with respect to some conveniently chosen set of
Cartesian axes  ({\em i.e.}, the standard $x$-, $y$-, and $z$-axes). Of course, the
orientation of these axes is {\em arbitrary}. In other words, different observers
may well choose differently aligned coordinate axes  to describe the same field. 
Consequently, the same electric and magnetic fields may have different components
according to different observers. It can be seen that any description of
 electric and magnetic fields is going to depend on two seperate things. 
Firstly, the nature of the fields themselves, and, secondly, the arbitrary choice
of  the coordinate axes  with respect to which these fields are measured. 
Likewise, Maxwell's equations---the equations which describe the behaviour
of electric and magnetic fields---depend on two separate things. Firstly, the
fundamental laws of Physics which govern the behaviour of electric and magnetic
fields, and, secondly, the arbitrary choice
of  coordinate axes. It would be helpful to be able to easily distinguish between those
elements of Maxwell's equations which depend on Physics from those which
only depend on coordinates.  In fact, this goal can  be achieved  by employing a branch of mathematics called {\em vector field theory}. This formalism enables  Maxwell's
equations to be written in a manner which is {\em completely independent}\/ of the choice
of coordinate axes. As an added bonus, Maxwell's equations look a
lot simpler when written in a coordinate-free fashion. Indeed, instead of
{\em eight}\/ first-order partial differential equations, there are only
{\em four}\/ such equations within the context of vector field theory. 

Electric and magnetic fields are  useful and interesting  because they interact {\em strongly}\/  with ordinary matter.
Hence,
the primary application of Maxwell's equations is the study of this interaction.
In order to facilitate this study, materials are generally divided into three
broad classes: {\em conductors}, {\em dielectrics}, and {\em magnetic materials}. Conductors contain free charges which drift in response to
an applied electric field. Dielectrics are made up of atoms and molecules which
develop electric dipole moments in the presence of an applied electric field.
Finally, magnetic materials are made up of atoms and molecules which develop
magnetic dipole moments in response to an applied magnetic field.
Generally speaking, the interaction of electric and magnetic fields with these three classes
of materials is usually investigated in two limits. Firstly, the {\em low frequency limit}, which
is appropriate to the study of the electric and magnetic fields found in conventional electrical circuits. Secondly, the {\em high frequency limit},  which is appropriate to the
study of the electric and magnetic fields which occur in electromagnetic waves.
In the low frequency  limit, the interaction of  a conducting body with electric and magnetic fields  is conveniently parameterized in terms of its {\em resistance}, its {\em capacitance}, and its {\em inductance}. Resistance measures the resistance
of the body to the passage of electric currents. Capacitance measures its
capacity to store charge. Finally, inductance measures the magnetic
field generated by the body when a current flows through it. Conventional
electric circuits are can be  represented as networks of pure resistors, capacitors, and inductors.

This book commences in Chapter~1 with a review of vector field theory. In
Chapters~2 and 3, vector field theory is employed to transform the
familiar laws of electromagnetism ({\em i.e.}, Coulomb's law, Amp\`{e}re's law, Faraday's
law, {\em etc.}) into Maxwell's equations. The general properties of these
equations and their solutions are then discussed. In particular, it is explained why it is necessary to use fields, rather than forces alone, to fully describe electric and magnetic phenomena. It is also demonstrated that Maxwell's equations are
soluble, and that their solutions are unique.
In Chapters~4 to 6, Maxwell's equations are  used to investigate the interaction
of low frequency electric and magnetic fields with conducting, dielectric, and
magnetic media. The related concepts of resistance, capacitance, and inductance are also examined. The interaction of high frequency radiation fields with various different types of media is
discussed  in Chapter~8. In particular, the emission, absorption, scattering, reflection, and refraction of electromagnetic waves is investigated in detail.
Chapter~7 contains a demonstration that Maxwell's equations
conserve both energy and momentum. Finally, in Chapter~9 it is shown that
Maxwell's equations are fully consistent with Einstein's Special Theory
of Relativity, and can, moreover,  be written in a manifestly Lorentz invariant manner. The relativistic form of Maxwell's equations is then used to examine
radiation by accelerating charges.

This book is primarily intended to accompany a single-semester upper-division Classical
Electromagnetism course for physics majors. It assumes a knowledge of
elementary physics, advanced calculus, partial differential equations, vector algebra, vector calculus, and
complex analysis.

Much of the material appearing in this book was gleaned from the excellent 
references listed in Appendix~D. Furthermore, the contents of Chapter~2 are partly
based on my recollection of a series of lectures given by Dr.~Stephen Gull
at the University of Cambridge.
